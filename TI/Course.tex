\documentclass{article}

\usepackage[utf8]{inputenc}
\usepackage[english]{babel}
 
\usepackage{caption}
%\usepackage{pgfplots}
\usepackage{listings}
\usepackage{graphicx}
\usepackage{footnote}
\usepackage{amsmath}
\usepackage{amsthm}
\usepackage{graphicx}
\usepackage{url}
\usepackage{amssymb}
\usepackage{mathrsfs}
\usepackage{amsfonts}
\usepackage{bbold}
\usepackage[boxed,linesnumbered,noend]{algorithm2e}
\usepackage{qcircuit}

\newtheorem{thm}{Theorem}
\newtheorem{prop}{Propriety}
\newtheorem{coro}{Corollary}
\newtheorem{lemma}{Lemma}
\newtheorem{defi}{Definition}



\setlength{\oddsidemargin}{0pt}
% Marge gauche sur pages impaires
\setlength{\evensidemargin}{0pt}
% Marge gauche sur pages paires
\setlength{\textwidth}{470pt}
% Largeur de la zone de texte 
\setlength{\topmargin}{0pt}
% Pas de marge en haut
\setlength{\headheight}{13pt}
% Haut de page
\setlength{\headsep}{10pt}
% Entre le haut de page et le texte
\setlength{\footskip}{40pt}
% Bas de page + séparation
\setlength{\textheight}{630pt}
% Hauteur de la zone de texte 
%\setlength\parindent{0pt}


\title{Information Theory}
\author{Omar Fawzi\footnote{omar.fawzi@ens-lyon.fr}\\
http://perso.ens-lyon.fr/omar.fawzi/teaching/it/index.html\\
\small Master 1, ENS de Lyon}
\date{}

\newcommand{\note}{\medskip\noindent\underline}
\SetKwInOut{Input}{input}\SetKwInOut{Output}{output}

\begin{document}

\maketitle

\tableofcontents
\newpage

\section{Introduction}
\textit{Midterm exam: Friday 4th November, 10:00 a.m.}


\textit{This course will mainly be a mathematical one, with only a few practical courses.}\medskip\\
\underline{Background needed:}
\begin{itemize}
\item A bit of Probability theory
\item Linear algebra (finite field)
\end{itemize}

\bigskip

"Information theory" comes from \textit{Shannon} in 1948 as "Communication theory": \quote{"The fundamental problem of communication is that of reproducing at one point either exactly or approximately a message selected at another point."} where "point" is to be taken at the broad sense.

\underline{Ex:}
\begin{itemize}
\item Point 1: memory at $t_1$, Point 2: memory at $t_2$.
\item Point 1: DNA of the parent cell, Point 2: DNA of the daughter cell.
\end{itemize}


\begin{align*}
\Qcircuit @C=2em @R=0.8em {
& \text{Point 1} & & \gate{\text{Channel}} & \qw & \text{Point 2} 
}
\end{align*}


\bigskip
Two fields of solution:
\begin{itemize}
\item Improve the channel
\item Accept an error model as given and build a system on top of it to transform it into a reliable one.
\end{itemize}

\begin{align*}
\Qcircuit @C=2em @R=0.8em {
& s & & \qw & t & & \gate{\text{Noise}} & \qw & r  & & \qw & \hat{s} 
}
\end{align*}



The goal is to achieve $s=\hat{s}$.  


As a designer: Find "good" encoding and decoding function. We want $\mathbb{P}(s\neq \hat{s} )$ small.

\underline{Example:} Suppose $I$ have memory cells storing 1 bit suffers from noise. After one year, the bits flip with probability $f\in [0,1]$.\\
We model this channel:
\begin{equation*}
W(y|x)=\text{prob that output}=y\text{ for input }x
\end{equation*}
For this channel, $W(0|0)=1-f$, $W(1|1)=1-f$, $W(1|0)=f$, $W(0|1)=f$.

Think that $f=0.1$, and that we want to tore a file with $n=10^6$ bits.
\bigskip

\subsection{Encoding 1: the Trivial encoding}

\begin{align*}
\Qcircuit @C=2em @R=0.8em {
& s_1 & & \qw_{(=)} & t_1 & & \gate{\text{Noise}} & \qw & r_1 \\
& s_2 & & \qw_{(=)} & t_2 & & \gate{\text{Noise}} & \qw & r_2 \\
& \vdots & & & \vdots & & & & \vdots \\
& s_n & & \qw_{(=)} & t_n & & \gate{\text{Noise}} & \qw & r_n \\
}
\end{align*}
Decoding $\hat{s}=r_i$.

\underline{Bit error:}
$\mathbb{P}(s_i\neq\hat{s}_i)=f$



\underline{Aside:} How different are $s$ and $\hat{s}$ distance between $s$ and $\hat{s}$ follows a Binomial$(n,f)$ distribution.

\begin{equation}
\mathbb{E}(\#flips)=nf,\qquad Var(\#flips)=nf(1-f)
\end{equation}

With high probability, $\#flips \in [nf-10\sqrt{nf(1-f)}, \; nf+10\sqrt{nf(1-f)]}$

\underline{Block error:}
\begin{align*}
\mathbb{P}(s\neq\hat{s}) &=1-\mathbb{P}(s\neq\hat{s})\\
&= 1-\mathbb{P}(\forall i\in \{1,...,n\}, s_i=\hat{s}_i)\\
&=1-(1-f)^n
\end{align*}

For this to be small, need $nf\ll 1$. For $n=10^6$, need $f=10^-8$.

\underline{Rate:} $\frac{\# bits\; in \; file}{\# cells\; used}=\frac{n}{n}=1$.

\bigskip
\subsection{Encoding 2: the Repetition code}

Encode each bit of file in 3 different cells.
\begin{align*}
\tag{$R_3$}
0\to000\\
1\to111
\end{align*}

\underline{Rate:} $\frac{1}{3}$
\bigskip

\begin{center}
\begin{tabular}{c c c c}
s & 0 & 1 & 0 \\
t & 000 & 111 & 000\\
r & 001 & 111 & 010\\
$\hat{s}$ & 0 & 1 & 0
\end{tabular}
\textit{Decoding: Majority vote.}
\end{center}

\underline{Bit error}
\begin{align*}
\mathbb{P}(s_1\neq \hat{s}_1) & =\mathbb{P}(\geq 2\; flips)\\
& = 3f^2(1-f)+f^3\\
&= 3f^2-2f^3\\
&<f(for\; f<1/2)
\end{align*}
Better than trivial encoding. For $f=0.1, \: \mathbb{P}(s_1\neq\hat{s}_1)=0.028$.


\underline{Block Error:}
\begin{align*}
\mathbb{P}(s\neq\hat{s}) & = 1 - \mathbb{P}(\forall i\in \{1,...,n\}, s_i=\hat{s}_i\\
& 1- (1-3f^2+2f^3)^n
\end{align*}
Slightly better but not so good.

\underline{HW:} Generalize to $N$ repetitions.

%TODO
%graph bit error (f=0.1) =f( rate )

\subsection{Encoding 3: the Block code}

Make block of size 4 and encode each one: (7,4)-Hamming code.

\begin{align*}
    \Qcircuit @C=2em @R=0.8em { 
&     & & & & t_1 & & \gate{\text{Noise}} & \qw & r_1\\    
& s_1 & & & & t_2 & & ... \\
& s_2 & & \gate{\text{Encode}} & \qw & t_3 & & ...\\
& s_3 & & & & t_4 & & ...\\
& s_4 & & & & t_5 & &  ...\\
& & & & & t_6 & & ...\\
& & & & & t_7 & & ...\\
}
\end{align*}
\begin{align*}
t_1&=s_1\\
t_2&=s_2\\
t_3&=s_3\\
t_4&=s_4\\
t_5&=s_1\oplus s_2 \oplus s_3\\
t_6&=s_2\oplus s_3 \oplus s_4\\
t_7&=s_1\oplus s_3 \oplus s_4\\
\end{align*}


\underline{Rate:} $\frac{4}{7}$

\underline{Decode:} $r_1,\; r_2,\; ... \; r_7$

%TODO scheme with circles

\underline{Ex:} 
\begin{equation*}
\overbrace{1000}^s \to \overbrace{1000101}^t
\end{equation*}

\underline{Decoding:} Flip the bit that is in all violated circles and \underline{not} in good circle.

If $\leq 1$ error, recover $T_1,...,t_7$ from $r_1, ... r_7$.

\underline{Bit error:} One can show that 
\begin{align*}
\mathbb{P}(s_i\neq\hat{s_i})\leq 9f^2 + O(f^3)
\end{align*} 

\underline{Block error:}
\begin{align*}
\mathbb{P}(s\neq\hat{s}) & = 1 - \mathbb{P}( \forall i\in \{1,...,\frac{n}{4}\}, \forall j \in \{0,1,2,3\}, s_{4i+j}=\hat{s}_{4i+j}) \\
& = 1 - \prod_{i=1}^{n/4} \mathbb{P} (\forall j \in {0,1,2,3,4}, s_{4i+j} = \hat{s}_{4i+j} )\\
& \leq 1- \mathbb{P}(\leq \text{1 error in a block})^{n/4}\\
& = 1 - (1-\left(\binom{7}{2} f^2 (n-f)^5 - ... \right)^{n/4}\\
& = 1 - (1 -21f^2 - O(f^3))^{n/4}
\end{align*}

The conventional wisdom was: "to decrease error probability, we need to decrease the rate to 0. But Shannon showed that we can do much better. \textbf{We can make the error probability go arbitrary close to 0 with a constant rate $\mathbf{> 0}$.} Even more, we can make the block error tate arbitrary close to zero at positive rate.

\underline{Ex:} $f=0.1$ File $n=10^6$ bits.

Use $\simeq 2. 10^6$ cells with very small block error.

\section{Information measure}
There are many approaches to define entropy, which mainly depends on the the question we ask.

\underline{Ex:} Given data $X$, determine the minimum space needed to store $X$.

\begin{itemize}
\item Find the shortest description of $X$\\
Solution: a description is an algorithm that computes $X$.
This is the \textit{Algorithmic complexity}, also called \textit{Kolmogorov complexity}.
\begin{align*}
X & = 0 ... 0 & \text{"small"}\\
X & = \pi & \text{"small"}\\
X & = "random" & \text{"large"}\\
\end{align*}
Problem: This in not computable.
\item More useful approach of Shannon\\
Entropy = measure of likelihood of $X$ (Thus we need a probability model).
\end{itemize}

\subsection{Probability notations}
All system are finite $(\Omega, \mathcal{E}, \mathbb{P})$. $X$ random variable in $\mathcal{X}$. We note $P_X(x)=\mathbb{P}(X=x)$.

For joint random variables, we note:
\begin{align*}
P_{XY}(x,y) & =\mathbb{P}(X=x,Y=y)\\
P_{X|Y=y}(x) & =\mathbb{P}(X=x | Y=y)\\
P_X^{\times n}&= P_X \times P_X \times ... \times P_X \quad \text{$n$ times} \\
\mathbb{E}(X) & = \sum_{x\in \mathcal{X}} xP_Y(x)\\
\end{align*}

\subsection{Entropy of event}

\begin{equation*}
h_X: \mathcal{E} \to \mathcal{R}_+ \cup \{\infty\}
\end{equation*}

\begin{enumerate}
\item Independence of representation: $h(E)$ only depends on $\mathbb{P}(E)$
\item Continuity with respect $\mathbb{P}$: $h$ continuity in $\mathbb{P}$
\item Additivity: $h(E\cap E')=h(E) + h(E')$ if $E$ and $E'$ are independent
\item Normalization: $h(E)= 1$ if $ \mathbb{P}(E)=\frac{1}{2}$
\end{enumerate}

\begin{prop}
$h_X$ satisfies 1, 2, 3, 4 $\Leftrightarrow h(E)=-\log_2 \mathbb{P}(E)$
\end{prop}

\begin{proof}
Skipped.
\end{proof}

$h$ is also called \textit{surprisal}.\\
If $X$ is a random variable, we define:
\begin{align*}
h_X(x) & = h_X(\{X=x\})\\
& = -\log_2 P_X(x)\\
\\
h_X: & \; \mathcal{X}\to \mathbb{R_+} \cup \{ \infty \}\\
& x \mapsto - \log_2 P(x)\\
\end{align*}

$h(X)$ is a random variable. It's distribution is 


\begin{defi}[Shannon entropy]
The Shannon Entropy of X is:
\begin{align*}
H(X)&=\mathbb{E}(h_X(X))\\
& =  - \sum_{x\in \mathcal{X}} P_X(x) \log_2 P_X(x)\\
\end{align*}

\end{defi}

\paragraph{Remarks}
\begin{itemize}
\item Only depends on $P_X$ and not on the values taken
\item Units is "bits"
\item $0\log_2 0 = 0$
\end{itemize}

\paragraph{Remark on notation}
$P_X(X)$ this is $P_X:\mathcal{X} \to \mathbb{R}_+$ applied to the random variable $X$. It is \underline{NOT} $\mathbb{P}(X=X)=1$.

\begin{prop}
For any $x\in \mathcal{X}$:
\begin{align*}
0\leq H(X) \leq \log |\mathcal{X}|\\
\end{align*}
With the equality cases $H(X)=0$ if and only if $X$ is constant and $H(X)=\log |\mathcal{X}|$ if and only if $X$ is uniform on $\mathcal{X}$.
\end{prop}

\begin{proof}
\begin{itemize}
\item First inequality: easy
\item 
\begin{align*}
H(X) & =\mathbb{E}\left( \log_2 \frac{1}{P_X(X)}\right) \\
\text{As log is concave :}\\
& \leq \log_2 \mathbb{E}_X\left( \frac{1}{P_X(X)}\right)\\
& = \log_2 \sum_{x\in \mathcal{X}} P_X(x)-\frac{1}{P_X(x)}=\log_2 |\mathcal{X}|\\
\end{align*}
Equality condition: all $P_X(x)$ are equal so $P_X$ is the uniform distribution.
\end{itemize}
\end{proof}


\paragraph{Remark} Expectation $\mathbb{E}(h_X(X))$ is not the only interesting quantity. For example
\begin{align*}
H_{\min}(X) & = \min_{x\in \mathcal{X}} h_X(x)\\
& = - \log_2 \max_x P_X(x)\\
\end{align*}

\paragraph{Ex} If $X\in \{0,1\}$ $P_X(0)=1-p$ and $P_X(1)=p$:
\begin{equation*}
H(X)=-p\log_2 p - (1-p)\log(1-p)
\end{equation*}

\subsection{Joint entropy and conditional entropy}

\begin{defi}[Joint entropy]
Let $X\in \mathcal{X}, \; Y\in \mathcal{Y}$. The joint entropy $H(X,Y)$ is defined as:
\[
\underbrace{H(X,Y)}_{H(XY)} = - \sum_{\substack{x\in \mathcal{X} \\ y \in \mathcal{Y}}} P_{XY}(x,y) \log_2 P_{XY}(x,y)\\
\]
\end{defi}

\begin{defi}[Conditional entropy]
The conditional entropy $H(X|Y)$ is defined as:
\begin{align*}
H(X|Y) & =\sum_{y\in \mathcal{Y}} P_Y(y).\underbrace{H(P_{X|Y=y})}_{H(X|Y=y)}\\
\end{align*}
\end{defi}

\paragraph{Ex}
\begin{itemize}
\item $X=Y$, then $H(X|Y)=\sum_{y\in \mathcal{Y}}P_Y(y)H(P_{X|Y=y})=0$
\item $Y$ and $X$ are independent, then $H(X|Y)=\sum_{y\in \mathcal{Y}} P_Y(y)\underbrace{H(P_{X|Y=y})}_{=H(P_X)}=H(X)$
\end{itemize}

\begin{prop}
\begin{equation*}
H(X|Y)=H(XY)-H(Y)
\end{equation*}
\end{prop}

\begin{proof}
\begin{align*}
P_{XY}(x,y) & = P_Y(y)P_{X|Y=y}(x)\\
H(XY) & = - \sum_{x,y} P_{XY}(x,y) \log_2 P_Y (y) P_{X|Y=y}(x)\\
& = - \sum_{x,y} P_{XY} (x,y) \log_2 P_Y(y)\\
& \quad - \sum_{x,y} P_{XY}(x,y) \log_2 P_{X|Y=y}(x)\\
&=H(Y) \qquad \qquad \left(\text{as } \sum_x P_{XY}(x,y)=P_Y(y) \right)\\
& \quad - \sum_y P_Y(y)\underbrace{\sum_x P_{X|Y=y}(x)\log P_{X|Y=y}(x)}_{-H(X|Y=y)}\\
& =H(Y) + H(X|Y)
\end{align*}
\end{proof}

\begin{defi}[The mutual information]
\begin{align*}
I(X:Y) & = H(X) - H(X|Y)\\
&= H(X) + H(Y) - H(XY)\\
I(X:Y) & = \sum_{x,y}P_{XY}\log_2 \frac{P_{XY} (xy)}{P_X(x)P_Y(y)}
\end{align*}
\end{defi}

\paragraph{Examples}
\begin{itemize}
\item If $X=Y$, $I(Y:Y)=H(X)$
\item If $X$ and $Y$ are independent, $I(X:Y)=0$
\end{itemize}

%TODO scheme

\begin{defi}
Let $P$ and $Q$ be distributed on $\mathcal{X}$.
The relative entropy
\begin{equation*}
D(P||Q)=\sum_{x\in \mathcal{X}} P(x). \log_2 \frac{P(x)}{Q(x)}
\end{equation*}
\end{defi}

\paragraph{Remark}
\begin{itemize}
\item Common name Kullback-Leibler divergence.
\item If $P(x)=0$, $P(x)\log \frac{P(x)}{Q(x)}=0$.
\item If for some $X\in\mathcal{X}$, $P(x)>0$ but $Q(x)=0$, $D(P||Q)=\infty$.
\item Not symmetric between $P$ and $Q$
\item $D(P||P)=0$
\item $I(X,Y)=D(P_{XY}||P_X\times P_Y)$
\end{itemize}

\begin{prop}
For any dist $P,Q$
\begin{equation*}
D(P||Q)\geq 0
\end{equation*}
with equality if and only of $P=Q$
\end{prop}

\begin{proof}
Let $S=\{x:P(x)>0\}$
\begin{align*}
D(P||Q) & = - \sum_{x\in S} P(x) \log_2 \frac{Q(x)}{P(x)}\\
& -\log_2 \text{ is convex}\\
\tag{1}
& \geq -\log_2 \sum_{x\in S} P(x) \frac{Q(x)}{P(x)}\\
& = -\log_2 \sum_{x\in S} Q(x)\\
\tag{2} 
& \geq 0
\end{align*}
Equality condition:
\begin{enumerate}
\item Strict convexity: $\frac{Q(x)}{P(x)}=C$. $\forall X \in S$
\item $\sum_{x\in S} Q(x)=1$\\
This implies that $Q=P$
\end{enumerate}
\end{proof}

\begin{coro}
For any $X,Y$
\begin{equation}
\tag{*}\label{*}
I(X:Y)\geq 0
\end{equation}
with equality if and only if $X$ and $Y$ are independent
\end{coro}

\begin{proof}
Just write $I(X:Y)=D(P_{XY}||P_X\times P_Y)$
\end{proof}

Another way of writing \eqref{*}
\begin{align*}
H(X) & \geq H(X|Y)\\
H(X)+H(Y) & \geq H(XY)
\end{align*}

\section{Data compression}
\subsection{Settings}
Also called \emph{source coding}.

In interesting data: not all possible sequences are expected.

\paragraph{Setting} Source $X\in \mathcal{X}$ with distribution $P_X$
\begin{align*}
C:\mathcal{X}\to \{0,1\}^*
\end{align*}

Two variants:
\begin{itemize}
\item Variable length compression\\
$|C(x)|$ might be different from $|C(x')|$.
Want to minimize, e.g., \emph{expected} length $\mathbb{E}(|C(X)|)$
\item Fixed-length compression, allow a probability of error $\delta$ and minimize the length.
\end{itemize}

\subsection{Variable length compression}
\subsubsection{General compressors}

\begin{defi}
A variable length lossless compressor  is a function $C:\mathcal{X}\to \{0,1\}^*$ such that there is a decompressor $D: \{0,1\}^* \to \mathcal{X}$ with $D\circ C(x)=x$ for all $x\in \mathcal{X}$
\end{defi}

\paragraph{Note}
\begin{itemize}
\item Equivalent condition: $C$ is injective
\item For $x\in \mathcal{X}$, $C(x)$ is called a code word
\\$\{C(x):x\in \mathcal{X}\}$ is called code or codebook.
\end{itemize}

\paragraph{Objective} Find $C$ that minimizes $\mathbb{E}(|C(X)|)$

\begin{thm}
Let $P_X$ be a distribution on $\mathcal{X}$ and $x_1,...,x_{|\mathcal{X}|}$ such that $P_X(x_1)\geq P_X(x_2)\geq ... \geq P_X(x_{|\mathcal{X}|})$

Then define $C^*(x_i)=w_i$ (the i-th bitstring in shortlex order).

$C^*$ is an optimal compressor i.e.,
\begin{equation*}
\mathbb{E}(|C^*(X)|) \leq \mathbb{E}(C(X))
\end{equation*}
for any lossless compressor $C$. We have
\begin{equation*}
H(X) - \log_2(1+ \lfloor \log_2 |X| \rfloor) \leq \mathbb{E}(|C^*(X)|) \leq H(X)
\end{equation*}
\end{thm}

\begin{proof}
Let $C$ be a lossless compressor, $C$ is injective
\begin{align*}
|\{x\in \mathcal{X}:|C(x)|\leq k \}| & \leq \sum_{l=0}^{k} 2^l\\
& = |\{x\in \mathcal{X}: |C^*(x)|\leq k\}|
\end{align*}
as $C^*$ uses all the possible strings of length $k$.
Because in addition, these codewords (bitstrings length $\leq k$) are assigned to the $2^{k+1} - 1$ elements with largest probability, we have
\begin{align*}
\sum_{x\in \mathcal{X} : |C(X)|\leq k} P_X(x) & \leq \sum_{x\in \mathcal{X} : |C^* (x)|\leq k} P_X(x)\\
\mathbb{E}(|C^*(X)|) & = \sum_{k=0}^\infty \mathbb{P}(|C^*(X)|>k)\\
\Big( \text{\underline{Aside:} Ex:}\quad \mathbb{E}X &= \sum_{n\geq 1} \mathbb{P}(Y\geq n)\Big)\\
& = \sum_{k=0}^\infty \sum_{x\in \mathcal{X}:|C^*(x)|>k}P_X(x)\\
& = \sum_{k=0}^\infty \left( 1- \sum_{x : |C*(x)|\leq k} P_X(x) \right)\\
& \leq \sum_{k=0}^\infty \left( 1 - \sum_{x: |C(x)|\leq k} P_X(x) \right)\\
& = \mathbb{E} (|C(X)|)
\end{align*}
\begin{itemize}
\item To relate to entropy:\\
Observe that $|C^*(x_i)|=\lfloor \log_2 (i) \rfloor$\\
Note also that $P_X(x_i)\leq 1-\sum_{j=1}^{i-1} P_X(x_j)\leq 1-(i-1)P_X(x_i)$\\
We get $P_X(x_i)\leq \frac{1}{i}$

\begin{align*}
\mathbb{E}(|C^*(X)|) & = \sum_{i=1}^{|\mathcal{X}|} P_X(x_i) \lfloor \log (i) \rfloor\\
 & \leq - \sum_{i=1}^{|\mathcal{X}|}P_X(x_i) \log \left( \frac{1}{i} \right)\\
 & \leq - \sum_{i=1}^{|\mathcal{X}|} P_X(x_1) \log P_X(x_j)\\
 &\leq H(X)
\end{align*}

\item Lower bound\\
Let $L=|C^*(X)|\in \{0,1,...,\lfloor \log |\mathcal{X} | \rfloor\}$
\begin{align*}
H(X,L) & =H(X)+\underbrace{H(L|X)}_{\substack{= \sum_x P_X(x)H(P_{L|X=x}) \\ =0}}\\
& = H(X)\\
\text{As a result: } H(X)=H(X,L) & =H(L)+ H(X\; | \;L)\\
& \leq \log_2 (1+ \lfloor \log|\mathcal{X}|\rfloor) + \sum_{k=0}^{\lfloor \log |\mathcal{X}|\rfloor} P_L(k)H(X|L=k)\\
& \leq \log_2(1+\log_2 |\mathcal{X}|)+ \underbrace{\sum_{k=0}^{\lfloor \log |\mathcal{X}|\rfloor} P_L(k).k}_{=\mathbb{E}L=\mathbb{E}(|C^*(X)|)}\\
\end{align*}

\end{itemize}
\end{proof}


\subsubsection{Uniquely decodable and prefix-free compression}

Let $C:\mathcal{A}\to \{0,1\}^*$, we can naturally define its extension on $\mathcal{A}^*=\bigcup\limits_{n\leq 1} \mathcal{A}^n$ by $C^+(a_1...a_n)=C(a_1).C(a_2)...C(a_n)$.

For $C^+$ to be lossless, we need $C$ to be lossless, but it is not sufficient in general. Let $\mathcal{A}=\{a,b,c\}$
\begin{align*}
C(a) & =0\\
C(b)&=010\\
C(c) &= 01\\
C^+b &=010\\
C^+(ca)&=010
\end{align*}

\begin{defi}[Uniquely decodable compressor]
A compressor $C$ is uniquely decodable if its extension $C^+$ is injective.
\end{defi}

\begin{defi}[Prefix-free compressor]
$C$ is a prefix-free compressor if no codeword is a prefix of any other.
\end{defi}

$Code=\{C(a):a\in \mathcal{A}\}$

\paragraph{Ex} $\mathcal{A}=\{a,b,c\}, C(a)=0, C(b)=10, C(c)=110$

\begin{prop}
Prefix-free $\Rightarrow$ uniquely decodable.
\end{prop}

\begin{proof}
To decompose $C(a_1)...C(a_n)$, $C(a_1)$ is the unique prefix of $C(a_1)...C(a_n)$ which is a codeword.
\end{proof}

\paragraph{Remark} $C$ might be uniquely decodable without being prefix-free.
\begin{align*}
C(a) & = 10\\
C(b) & = 11\\
C(c) & = 110
\end{align*}

Uniquely decodable is more general than prefix-free, but not very useful, because there is a correspondence between prefix-free code and uniquely decodable code.

\begin{thm}
Let $A\in \mathcal{A}$ be a random variable. The Huffman algorithm computes $O(|\mathcal{A}|\log|\mathcal{A}|)$ a prefix-free compressor $C_H:\mathcal{A}\to \{0,1\}^*$ with minimum expected length $\mathbb{E}|C_H(A)|$ among all possible prefix-free compressor.

Moreover,
\begin{equation*}
\mathbb{E}(|C_H(A)|) < H(A)+1
\end{equation*}
\end{thm}

The key observation is to find a correspondence between prefix-free code and a binary tree.

%TODO figure of tree


Bitstrings labelling leaves form a prefix-free code. In this representation, the expected length is $\sum_{a\in \mathcal{A}}P_A(a).depth(C(A))$.

\begin{proof}
Huffman: in tutorial and HW
\end{proof}


\begin{lemma}[Kraft's inequality]
\begin{itemize}
\item For any prefix-free compressor $C$ with codeword lengths $l_a=|C(a)|$, we have \begin{equation*}
\tag{*}
\label{*low}
\sum_{a\in \mathcal{A}} 2^{-l_a}\leq 1
\end{equation*}
\item Conversely, given a set of length $\{l_a\}$ satisfying \eqref{*low}, we can construct a prefix compressor with $|C(a)|=l_a$.
\end{itemize}
\end{lemma}

\begin{proof}
$\Rightarrow$ In terms of binary tree $T$, $l_a=depth(\underbrace{C(A)}_{\text{leaf}})$. I have exactly $\sum_{a\in \mathcal{A}} 2^{l_{max}-l_a}$ nodes at level $l_{max}$, but at most $2^{l_{max}}$ nodes at depth $l_{max}$, so 
\begin{align*}
\sum_{a \in \mathcal{A}} 2^{l_{max}-l_a} & \leq 2^{l_{max}}\\
\sum_{a\in \mathcal{A}} 2^{-l_a} & \leq 1\\
\end{align*}
$\Leftarrow$ Let $\{l_a\}$ satisfy $\sum_a 2^{-l_a}\leq 1$. We order the elements of $\mathcal{A}, \{a_1,...,a_n\}$ so that $l_{a1}\leq...\leq l_{a_{|\mathcal{A}|}}$

\begin{align*}
C(a_i) & =\text{binary expansion of length $l_{a_i}$ of } \sum_{j=1}^{i-1} 2^{-l_{a_i}} <1\\
& = 0.\underbrace{01...0}_{l_{a_i}}
\end{align*}
Want to show that $C$ is prefix-free. Consider $C(a_i)=b_i$ and $C(a_k)=b_k$ for $k>i$.
\begin{align*}
b_k-b_i=\sum_{j=i}^{k-1}2^{-l_{a_j}}\geq 2^{-l_{a_i}}
\end{align*}
Any codeword that has $C(a_i)$ as a prefix is a binary expression of a number that is at most
\begin{align*}
b_i+\sum_{p=l_{a_i}+1}^{l_{max}}2^{-p}<b_i + 2^{-l_{a_i}}
\end{align*}
So $C(a_i)$ cannot be a prefix of $C(a_k)$. So $C$ is prefix-free.
\end{proof}

Using lemma, we can write the minimum expected length as the following optimization program:

\begin{align*}
OPT=\text{ minimize } & \sum_{a\in \mathcal{A}}P_A(a)l_a\\
\text{subject to } & l_a\in \mathbb{N}_+\\
& \sum_{A\in \mathcal{A}} 2^{-l_a}\leq 1
\end{align*}

\begin{proof} of the theorem.\\
$\mathbb{E}(|C_H(A)|)=OPT$. We start by proving that $H(A)\leq OPT$. For that, we relax the condition $l_a \in \mathbb{N_+}$ to $l_a \in \mathbb{R}$. We change variables: $Q(a)=2^{-l_a}$.

The program becomes:
\begin{align*}
OPT=\text{ minimize } & \sum_{a\in \mathcal{A}}P_A(a). (-\log_2 Q(a) ) \\
\text{subject to } & \sum_{A\in \mathcal{A}} Q(a)\leq 1
\end{align*}

Recall that $D(P||Q)=\sum_a P(a)\log_2 P(a) - \sum_a P(a)\log_2 Q(a)$. So the objective function can be written as:
\begin{equation*}
-\sum_a P_A(a)\log_2 P_A(a) + \sum_a P_A(a)\log_2 P_A(a) - \sum_a P_A(a)\log_2 Q(a) = H(P_A) + D(P_A||Q)
\end{equation*}

To show that $D(P_A||Q)\geq 0$, we consider $Q'(a)=\frac{Q(a)}{\sum_{a'} Q(a')}$ be the normalized version of $Q$.

\begin{align*}
D(P_A||Q) & = \sum_a P_A(a)\log_2 P_A(a) - \sum_a P_A(a)\log_2 (Q'(a).\sum_{a'} Q(a'))\\
& = \sum_a P_A(a)\log_2 P_A(a) - \sum_a \Big( P_A(a)\log_2 Q'(a) \Big) - \log_2(\sum_{a'} Q(a') )\\
& \geq D(P_A||Q')\\
& \geq 0 \text{ by propriety of the relative entropy}
\end{align*}
So the value for the relaxed program is exactly $H(P_A)$.

Now we want to show that $OPT<H(A)+1$.

From the lower bound proof, we choose $l_a=-\log_2 P_A(a)$, but might not be an integer. Let's choose $l_a=\lceil - \log_2 P_A(a) \rceil$. We have $\sum_a 2^{-l_a}\leq \sum_{a} P_A(a)=1$, and the objective function has thus the value 
\[
\sum_a P_A(a).l_a = \sum_a P_A(a) \lceil - \log_2 P_A(a) \rceil < \sum_a P_A(a)\big( -\log_2P_A(a) + 1 \big) =H(P_A)+1
\]
\end{proof}

\begin{prop}
For any uniquely decodable compressor $C$ with codeword lengths $\{l_a\}_{a\in \mathcal{A}}$, we have 
\[\sum_{a\in \mathcal{A}} 2^{-l_a}\leq 1\]
\end{prop}

\begin{proof}
For a string $a^n=a_1...a_n$, define 
\[C(a^n)=C(a_1)...C(a_n)\]
and length \[l_{a_n}=l_{a_1}+...+l_{a_n}\]

\begin{align*}
\Big( \sum_{a\in \mathcal{A}} 2^{-l_a}\Big)^n & = \sum_{a\in \mathcal{A}} 2^{-l_{a^n}}\\
& = \sum_{m=1}^{n.l_{max}} N_m.2^{-m}\\
\text{where } N_m & = |\{a^n\in \mathcal{A}^n:L_{a^n}=m\}|\\ 
\end{align*}
By unique decodability, $N_m\leq 2^m \leq n.l_{max}$.

So for all $n\geq 1$,
\[\sum_{a\in \mathcal{A}} 2^{-l_a} \leq \underbrace{(n.l_{max})^{1/n}}_{\xrightarrow [n \rightarrow \infty]{} 1}\]
\end{proof}

\subsection{Fixed-length almost lossless compression}
\begin{defi}
A fixed-length compressor for some $x\in \mathcal{X}$ of length $l$ is a function $C:\mathcal{X}\to \{0,1\}^l$. It has an error probability $\leq \delta$ if there exists $D:\{0,1\}^l \to \mathcal{X}$ such that $\mathbb{P}(D \circ C(X)=X)\geq 1-\delta$.
\end{defi}

We would like to determine
\begin{defi}
\[ l^{OPT}(X,\delta)=\min\{l:\text{there is a length $l$ compressor with error probability $\leq \delta$}\}\]
\end{defi}

\newpage

One natural encoding strategy is:
\begin{algorithm}
Sort elements in $\mathcal{X}=\{x_1,x_2,...,x_{|\mathcal{X}|}\}$\\
so that $P_X(x_1)\geq P_X(x_2)\geq ... \geq P_X(x_{|\mathcal{X}|})$\\
$S_\delta^*=\emptyset$\\
\For{i=1 to $|\mathcal{X}|$}{
	$S_\delta^*\leftarrow S_\delta^*\cup \{x_i\}$\\
	\If{$\sum_{x\in S_\delta^*} P_X(x) \geq 1-\delta$}{
		stop
	}
}
\end{algorithm}

\begin{thm}
\[ l^{OPT}=\lceil \log_2 |S_\delta^* | \rceil\]
\end{thm}

\begin{proof}
$\bullet$ Start with "$\leq$"\\
$S_\delta^*=\{x_1,...,x_k\}$ for some $k$.\\
$l=\lceil \log_2 |S_\delta^* | \rceil$, we have $k=|S_\delta^*|\leq 2^e$.
\[C(x_i)=
\begin{cases}
\underbrace{bin^l(i-1)}_{\substack{\text{binary representation}\\ \text{of $i-1$ with $l$ bits}}} & \text{if $i\leq k$}\\
\qquad \quad \; \; 0^l & \text{if $i>k$}
\end{cases}\]

Define $D(y)=x_i$ if $i-1$ is the number in $\{0,1,...,2^l-1\}$ with binary representation $y$ ($y$ is a bitstring of length $l$).
\begin{align*}
\mathbb{P}(D\circ C(X)=X) & \geq \sum_{i=1}^{l}P_X(x_i)\\
& \geq 1-\delta
\end{align*}

$\bullet$ Ineq "$\geq$"\\
Let $C$ be a compressor with length $l$ and error probability $\leq \delta$.\\
Define $S=\{ x\in \mathcal{X}:D(C(X))=x \}$\\
$\bullet S\subset D(\{0,1\}^l)$ so $|S|\leq 2^l$\\
$\sum_{x\in S} P_X(x)=\mathbb{P}(D\circ C(X)=X)=1-\delta$

So $|S_\delta^*|\leq |S| \leq 2^l $ and so $\log |S_\delta^*|\leq l$ using the fact that $S_\delta^*$ is the smallest set with probability $\geq 1 - \delta$.
\end{proof}

\paragraph{Remark} $\lceil \log_2 |S_\delta^* | \rceil$ can be very different from $H(X)$.

As an example: $\mathcal{X}=\{0,1,...,m\}$
\[
X=\begin{cases}
0 & \text{with prob $1-\epsilon$}\\
i & \text{with prob $\frac{\epsilon}{m}$}
\end{cases}
\]

For $\delta=0$, $\log_2 |S_\delta^* | = \log_2 (m+1)$

But \begin{align*}
H(X) & = -(1-\epsilon)\log_2 (1-\epsilon) - \sum_{i=1}^{m}\frac{\epsilon}{m}\log_2\frac{\epsilon}{m}\\
& = \underbrace{ -(1-\epsilon )\log_2(1-\epsilon )-\epsilon\log_2 \epsilon}_{h_2(\epsilon)}+ \epsilon \log_2 m
\end{align*}

But $\log_2 |S_\delta^* |$ is an entropic quantity in itself.\\
Define
\[\tag{Hartley entropy}
H_0(X)=\log_2(|\sup P_X|) \]

Note that $H_0$ has shared some proprieties with $H$:
\begin{itemize}
\item $H_0=0$ iff $X=x_0$ wp $1$
\item $H_0=\log |\mathcal{X}|$ if $X$ is uniform
\item $H_0(X)\geq H(X)$ (Ex)
\end{itemize}

If allow "error probability" $\delta$, we define \emph{another} version
\[H_0^\delta = \min_{\sum_{x\in s_\delta}P_X(x)\geq 1-\delta} \log_2 |S_\delta|\]

Smoothing can have a strange effect on entropy. For $X$ defined before,
\[H_0(X)=\log(1+m) \qquad H_0^\epsilon=0\]

We also saw in homework an example:
$X_i\hookrightarrow \mathcal{B}(p) \quad X^n=X_1...X_n \in \{0,1\}^n: H_0 (X^n)=n$, but $H_0^\delta (X^n)\leq n\log p$

Recall that
\[H(X)=\mathbb{E}(\underbrace{h_X(X)}_{surprisal})\]
\[ \text{with }  h_X(X)=-\log_2(P_X(X))\]

\paragraph{Examples of $h_X(X)$}
\begin{itemize}
\item $X\hookrightarrow \mathcal{U}(\mathcal{X})$\\
$h_X(X)=\log |\mathcal{X}|$ with probability $1$,\\
In particular $H(X)=\mathbb{E(h_X(X))=\log |\mathcal{X}|}$
\item $\mathcal{X}=\{1,2,...,2t\}$
\[
P_X(X)=
\begin{cases}
\frac{3}{4t} & \text{for } X\in\{1,...,t\}\\
\frac{1}{4t} & \text{for } X\in\{t+1,...,2t\}
\end{cases}
\]
\begin{align*}
\mathbb{P}(-\log_2P_X(X)=\log_2 \frac{4t}{3}) &  = \sum_{\log\frac{1}{P_X(x)}=\log\frac{4t}{3} \Leftrightarrow P_X(x)=\frac{3}{4t}} P_X(x)=\frac{3}{4}\\
\mathbb{P}(-\log_2P_X(X)=\log_2 4t) &  = \sum_{x:P_X(x)=\frac{1}{4t}} P_X(x)=\frac{1}{4}\\
H(X) & = t \frac{3}{4t}\log_2 \frac{4t}{3} + t\frac{1}{4t}\log 4t\\
& = \frac{3}{4} \log \frac{4}{3}t + \frac{1}{4}\log 4t
\end{align*}
\end{itemize}

%missing graph

%In general

%missing graph

\begin{prop}
\[l^{OPT}(X,\delta)\leq \min \{ l\in \mathbb{N}_+ : \mathbb{P}(h_X(X)>l)\leq \delta\}\]

\emph{"achievability"}: There is a compressor with length $l$ and error probability $\leq \mathbb{P}(h_X(X)>l)$

Moreover, for any $\tau > 0$
\[l^{OPT}(X,\delta)\geq \min \{ l\in \mathbb{N}_+: \mathbb{P}(h_X(X)>l+\tau) - 2^{-\tau}\leq \delta\}\]

\emph{"converse"}: For any compressor and any $\tau > 0$, the probability of error is at least $P(h_X(X)>l+\tau)-2^{-\tau}$
\end{prop}

\begin{proof}
$\bullet$ Let $l$ satisfy $\mathbb{P}(h_X(X)>l)\leq \delta $. Take $S=\{x\in \mathcal{X}: P_X(x)\geq 2^{-l})\}$.\\
Note that $|S|\leq 2^l$\\
Moreover
\begin{align*}
\mathbb{P}(X\in S) & = \mathbb{P}(P_X(X)\geq 2^{-l})\\
& = \mathbb{P}(-\log_2 P_X(X) \leq l)\\
& = 1 - \mathbb{P}(h_X(X)>l)\\
& \geq 1-\delta
\end{align*}


$\bullet$ Converse

Given $C$ with length $l$ and error probability $\leq \delta$.
\begin{align*}
S & = \{ x\in \mathcal{X}: D(C(X))=x \}\\
S & \subset D(\{0,1\}^l) \text{ so } |S|\leq 2^l\\
\text{We have } \mathbb{P}(X\in S) & = \mathbb{P} (D(C(X))=X)\\
& \geq 1 - \delta
\end{align*}

\begin{align*}
1-\delta & \leq \sum_{x\in S} P_X(x)\\
& = \sum_{x\in S: \underbrace{P_X(x)\geq 2^{-l-\tau}}_{-\log_2 P_X(x)\leq l+\tau}} P_X(x)+\sum_{x\in S:P_X(x)<2^{-l-\tau}} P_X(x)\\
& \leq \sum_{x:-\log_2 P_X(x)\leq l+\tau}P_X(x) + 2^{l}.2^{-l-\tau}\\
& = \underbrace{\mathbb{P}(h_X(X)\leq l+\tau)}_{=1-\mathbb{P}(h_X(X)>l+\tau)} + 2^{-\tau}
\end{align*}


So $l$ satisfies
\[\mathbb{P} ( h_X(X)>l+\tau) - 2^{-\tau} \leq \delta\]
\end{proof}

\paragraph{Important special case}
\begin{align*}
X^n & =X_1X_2...X_n\\
& \text{with $X_i$ independent and identically distributed}\\
& \text{with same distribution as $X$}
\end{align*}


\begin{thm}[Shanon's source coding theorem]
For any $\delta \in (0,1), 0<\delta<1$
\[\lim_{n\to \infty} \frac{l^{OPT}(X^n,\delta)}{n} = H(X)\]
\end{thm}

\begin{proof}
Need to get a handle on $\mathbb{P}(h_{X^n}>l)$

\begin{align*}
h_{X^n}(X^n)=-\log_2 P_{X^n}(X^n) & = -\log_2 P_X(X_1)P_X(X_2)...P_X(X_n)\\
& = \sum _{i=1}^n \log_2 P_X(X_i)\\
\mathbb{E} \big(h_{X^n}\big) & = n \mathbb{E}\big( h_X(X_i)\big)\\
& = nH(X)\\
\mathbb{P} (|h_{X^n}(X^n)-nH(X)|\geq t) & \leq \frac{\mathbb{V}\big(h_{X^n}(X^n)\big)}{t^2}\\
\mathbb{V}(h_{X^n}(X^n)) & = \mathbb{V} \left( -\sum_{i=1}^n \log_2 P_X(X_i) \right)\\
& = n \mathbb{V}(h_X(X))\\
\text{So } \mathbb{P} (|h_{X^n}(X^n)-nH(X)|\geq t) & = \frac{n\mathbb{V} (h_X(X))}{t^2}
\end{align*}

Set $t=\sqrt{\frac{n\mathbb{V}(h_X(X))}{\delta}}$, we get:
\begin{align*}
\mathbb{P}(h_{X^n}(X^n)) & > nH(X)+ \sqrt{\frac{n\mathbb{V}(h_X(X))}{\delta}} \leq \delta\\
\text{So } l^{OPT}(X^n,\delta) & \leq nH(X)+\sqrt{\frac{n\mathbb{V}(h_X(X))}{\delta}}\\
\text{So } \lim_{n\to\infty} \frac{l^{OPT}(X^n,\delta)}{n} & \leq H(X)
\end{align*}
For the lower bound, let $\alpha > 0$ take $t=\sqrt{\frac{n\mathbb{V}(h_X(X))}{\alpha}}$
\[\mathbb{P}(h_{X^n}(X^n))\leq \underbrace{nH(X)-\sqrt{\frac{n\mathbb{V}(h_X(X))}{\alpha}}}_{l+\tau} \leq \alpha\]

Now take $l=nH(X)-2\sqrt{\frac{n\mathbb{V}(h_X(X))}{\alpha}}$ and $\tau = \sqrt{\frac{n\mathbb{V}(h_X(X))}{\alpha}}$.

Choose $\alpha$ small enough such that $1-\delta> \alpha + 2^{-\tau}$

\begin{align*}
\mathbb{P}(h_{X^n}(X^n)>l+\tau) & \leq 1 - \alpha > \delta + 2^{-\tau}\\
\text{So } l^{OPT}(X,\delta) &\geq l = nH(X)-2\sqrt{\frac{n\mathbb{V}(h_X(X))}{\alpha}}
\end{align*}
\end{proof}

In tutorial, we showed that $S$, chosen here as $\{x\in \mathcal{X} : P_X(X)\geq 2^{-l}\}$ can be picked at random, but gives therefore a good code (the bound is almost the same).
In practice:
\begin{itemize}
\item Stream of symbols
\item But not independent (ex: informati\textbullet\textbullet $\rightarrow$ information) so the rest of the word depends on the context\\
Hoffman with larger blocks may be inefficient
\item Do not even know usually the distribution $\rightarrow$ universal compressor
\end{itemize}

\subsection{Universal compression}
Consider a stream $X^n$ with $n$ symbols in $\mathcal{X}$. We do not have access to $P_{X^n}$.


\subsubsection{Arithmetic code}
Idea: learn a model for data.

\paragraph{Ex} $P_1(a)=\frac{1}{|\mathcal{X}|}$. Then, when we see $x_1,...,x_{i-1}$, 
\[P_i(a|x_1,...,x_{i-1})=\frac{1+|\{ j\in \{ 1,..., i-1 \}, x_j=a \}|}{|\mathcal{X}|+(i-1)}\]

\paragraph{Remark} Simple to compute, only need to keep $|\mathcal{X}|$ counters.

It is useful to interpretate a bitstring as an interval in $[0,1]$
\begin{align*}
01 & \mapsto [0.01;0.1]\\
y & \mapsto [0.y;0.y+0.\underbrace{0...01}_{|y|}]
\end{align*}

\paragraph{Idea} encode stream as intervals. Each new symbol: choose a subinterval of current interval with length proportional to the probability given by model.

$\mathcal{X}=\{ a_1, a_2,...,a_n\}$

\paragraph{Algorithm} For a new symbol \emph{$x_i=a_k$}, chose subinterval given by
\[ \left[ \underbrace{\sum_{p=1}^{k-1} P_i(a_p | x_1 ... x_{i-1})}_{\alpha}, \underbrace{\sum_{p=1}^{k} P_i(a_p|x_1 ... x_{i-1})}_{\beta}\right]\]


In absolute terms: if current interval is $[u_{i-1}, v_{i-1}]$:
\begin{align*}
u_i & = u_{i-1} + (v_{i-1}-u_{i-1})\alpha\\
v_{i} & = u_{i-1} + (v_{i-1}-u_{i-1})\beta
\end{align*}

\paragraph{Problem} From interval to bitstrings?

\paragraph{Solution} Find largest dyadic interval included in it.

Overall,
\[x_1,...,x_n \to I_{x_1,...,x_n} \to \text{Find } \underbrace{I_y}_{\text{dyadic interval}} \subset I_{x_1,...,x_n} \to \text{output } \underbrace{y}_{\text{bitstring}} \]

\paragraph{Remark} Decoding is easy if agree on model $P_i$.


\subsubsection{Lempel-Ziv coding}
No probabilistic model, based on dictionary of words that appeared. Read sequence into words:
\[m_0,m_1,...,m_L\]

\begin{itemize}
\item $m_0=\emptyset$
\item $m_i=m_{j}.x$ for some $x\in \mathcal{X}$
\item $m_i$ are distinct for $0\leq i < L$
\item $m_L=m_j$ for some $j<L$
\end{itemize}

\[\underbrace{\;}_{m_0}
\underbrace{a}_{\substack{m_1 \\ j=0\\x=a}}
\underbrace{b}_{\substack{m_2 \\ j=0\\x=b}}
\underbrace{aa}_{\substack{m_3 \\ j=1\\x=a}}
\underbrace{aaa}_{\substack{m_4 \\ j=3\\x=a}}
\underbrace{b}_{\substack{m_1 \\ j=2}}\]


Each word encoded in a pair (pointer to $j$, additional letter $x$). Encoding word $m_i$ cost $\lceil \log i \rceil + 1$.

\paragraph{Lossy compression} \textit{Will not talk about it.}\\
For images, audio, we don't need exact recovery $\rightarrow$ rate \emph{distortion}.


\section{Noisy channel coding}
\subsection{Setting}
\paragraph{Channel}
\begin{itemize}
\item Input alphabet $\mathcal{X}$
\item Output alphabet $\mathcal{Y}$
\end{itemize}

\[W_{Y|X}(y|x)=\text{ probability of outputting $y$ when the input is $x$}\]

\paragraph{Our task}
Find $E,D$ to send messages with small error probability.

\begin{align*}
\Qcircuit @C=2em @R=0.8em {
& s \in \{1,...,M\} \qquad \quad & & \gate{E} & \qw & x & & \gate{W} & \qw & y & & \gate{D} & \qw & \qquad \quad \hat{s}\in \{1,...,M\}
}
\end{align*}

\begin{defi}
An $M$-code for $W$ is a pair of functions
\begin{align*}
E &: [M] \to \mathcal{X}\\
D &: \mathcal{Y} \to [M]
\end{align*}
\end{defi}

\begin{itemize}
\item $E(s)$ is called codeword and $\{ E(1),...,E(M) \}$ is codebook.
\item Decoding region $D_S=D^{-1}(\{s\})=\{y:D(y)=s\}$
\end{itemize}
%Scheme here

We will talk about random variables 
\[\underbrace{S}_{\substack{\text{original}\\ \text{message}}},
\underbrace{X}_{\substack{\text{encoded}\\ \text{message}\\ \text{or}\\ \text{channel input}}},
\underbrace{Y}_{\substack{\text{channel}\\ \text{output}}},
\underbrace{\hat{S}}_{\substack{\text{decoded}\\ \text{message}}}\]

For the rest of the section, we assume that the distribution on messages is uniform, i.e.:

\[P_{SXY\hat{S}}(s,x,y,\hat{s})= \frac{1}{M} \mathbb{1}_{x=E(s)}.W(y|x).\mathbb{1}_{\hat{s}=D(y)}\]

\begin{defi}[Error Probability]
In that case, we define the error probability as follow:
\begin{align*}
P_{err} &= \mathbb{P}(S\neq\hat{S})\\
& = 1 - \frac{1}{M}\sum_{s=1}^M \sum_{y\in \mathcal{Y}} W(y|E(s)).\mathbb{1}_{D(y)=s}
\end{align*}
\end{defi}

\paragraph{Note} $P_{err}$ was called \emph{block error probability} in the first lectures, this is the \emph{average} error probability.

\paragraph{Two others}
\begin{itemize}
\item $P_{err,max}=\max_{s\in[M]} \mathbb{P}(\hat{S}\neq s|S=s)$\\
Maximum error probability
\item If $M=2^k$, we can see $[M]=\{0,1\}^k$, and thus define $P_{bit}=\frac{1}{k}\sum_{i=1}^{k}\mathbb{P}(S_i \neq \hat{S}_i)$
\end{itemize}

\paragraph{Question} Trade-off between $M$ and $P_{err}$.

\begin{defi}
\[M^{OPT}=\max \{ M: \text{there is an $M$-code with }P_{err} \leq \delta \}\]
\end{defi}

\paragraph{Remark} $\log_2 M^{OPT}(W,\delta)=$ number of bit used

\paragraph{Important thing to keep in mind} Let $W^n$ be $n$ independent copies of $W$.
\[W^n(y_1...y_n|x_1...x_n)=W(y_1|x_1)...W(y_n|x_n)\]
With $n\to \infty$ and want $\delta \to 0$. So we are interested in $\frac{\log M^{OPT}(W^n,\delta)}{n}$, which is the number of bits that can be send per channel use.

\paragraph{Examples}
\begin{enumerate}
\item $\mathcal{X}=\{1,...,N\}$\\
$\mathcal{Y}=\{1,...,N\}$\\
$W(y|x)=\mathbb{1}_{x=y}$
\begin{itemize}
\item If $M\leq N$, $E=id, D=id$ so $P_{err}=0$.
\end{itemize} If $M>N$, take $E, D$ and $M$-code for $\mathbb{N}.$
\begin{align*}
P_{err} & = 1 - \frac{1}{M}\sum_{s=1}^M \sum_{y\in \mathcal{Y}} W(y|E(s)).\mathbb{1}_{D(y)=s}\\
& = 1 - \frac{1}{M}\sum_{y\in \mathcal{Y}} \underbrace{\sum_{s=1}^M W(y|E(s)).\mathbb{1}_{D(y)=s}}_{\underbrace{W(y|E(D(Y)))}_{\leq 1}}\\
& \geq 1 - \frac{|y|}{M}=1-\frac{N}{M}
\end{align*}

\item Binary symmetric channel.
%scheme 0 -> 0
%         X
%       1 -> 1
$f<1/2$

$BSC_f^{\times n}$: $n$ independent copies of $BSC_f$.

\paragraph{Repetition code}
Message set $=\{0,1\}^{n/3}$

\note{Encoder:} $E(s_1...s_{\frac{n}{3}})=s_1s_1s_1...s_{\frac{n}{3}}s_{\frac{n}{3}}s_{\frac{n}{3}}$

\note{Decoder:} $D(y_1...y_n)=maj(y_1y_2y_3)maj(y_4y_5y_6) ... maj(y_{n-2}y_{n-1}y_{n})$\\
$P_{bit}=3f^2-2f^3, P_{err}=1-(1-3f^2+2f^3)^{n/3}$
\end{enumerate}

\begin{defi}
The information capacity of a channel $W$ with input alphabet $\mathcal{X}$ and output alphabet $\mathcal{Y}$ is
\[ C(W)=\max\limits_{\substack{P_X \\ \text{distribution over $\mathcal{X}$}}} I(X:Y) \qquad \text{with } P_{XY}(x,y)=P_X(x)W_{Y|X}(y|x)\]
\end{defi}

\paragraph{Ex}
\begin{enumerate}
\item $W(y|x)=\mathbb{1}_{y=x}\qquad \mathcal{X}=\{1,...,N\}, \mathcal{Y}=\{1,...,N\}$.
For any $P_X$:
\begin{align*}
I(X:Y) &= H(X) - \underbrace{H(X|Y)}_{=0}\\
&= H(X)
\end{align*}

So $C(W)=\max_{P_X}I(X:Y) = \max_{P_X} H(X)=\log N$

\item 
%scheme a -> 0
%         /
%       b
%         \
%       c -> 1
$P_X=(1/3,1/3,1/3)$\\
$I(X:Y)=\underbrace{H(X)}_{\log 3} - H(X|Y)$\\
$H(X|Y)=?$
\[P_{X|Y=0}(x)=\frac{P_X(x)W(0|x)}{1/2}=
\begin{cases}
2/3 & \text{if $x=a$}\\
1/3 & \text{if $x=b$}
\end{cases}
\]
$H(X|Y)=h_2(1/3)$\\
$I(X:Y)=2/3$\\
A better distribution is: $P_X=(1/2,0,1/2)$\\
$I(X:Y)=\underbrace{H(X)}_{=1}-\underbrace{H(X|Y)}_{=0}=1$

We cannot do better:\\
$I(X:Y)\leq H(Y)\leq \log |\mathcal{Y}|=1$\\
$C(W)=1$

\item $BSC_f$ $P_X(0)=1-p$ and $P_X(1)=p$
\begin{align*}
I(X:Y) &=H(Y)-H(Y|X)\\
H(Y|X) &=P_X(0)H(Y)_{P_{Y|X=0}}\\
&+P_X(1)H(Y)_{P_{Y|X=0}}\\
& = h_2(f)
\end{align*}

\[Y= \begin{cases}
0 & \text{wp } (1-p)(1-f)+pf\\
1 & \text{wp } p(1-f)+(1-p)f
\end{cases}\]

$H(Y)=h_2\left( (1-p)(1-f)+pf \right)$\\
$I(X:Y)=h_2\left( (1-p)(1-f)+pf \right) - h_2(f)$\\
This is maximized for $p=1/2$. So $C(BSC_F)=1-h_2(f)$
\end{enumerate}


\subsection{Converse bounds}
\begin{thm}
Any $M$-code for $W$ satisfies
\[\log M \leq \frac{C(W)+h_2(P_{err})}{1-P_{err}}\]
\end{thm}

\begin{proof}
We start with the case $P_{err}=0$. Take an $M$-code.
\[P_{SXY\hat{S}}\]

\begin{align*}
\log M = H(S) & = I(S:\hat{S})+\underbrace{H(S|\hat{S})}_{=0 \text{ as } P_{err}=0}\\
& = I(S:\hat{S})\\
S\to X \to Y \to \hat{S} & \text{ is a markov chain}\\
I(S:\hat{S}) & \leq I(X:Y)\\
(\text{See data processing inequality in tutorial:} & \; X\to Y \to Z \; \text{Markov chain}\; I(X:Y)\geq I(X:Z))\\
\log M & \leq I(X:Y)\leq C(W)
\end{align*}

Now general $P_{err}$:
\begin{lemma}[Fano's inequality]
If $S\in \{1,...,M\}$ and $\hat{S}$ such that $\mathbb{P}(S\neq\hat{S})\leq \epsilon$, then
\[H(S|\hat{S})\leq h_2(\epsilon)+\epsilon \log M\]
\end{lemma}
\begin{proof}
Introduce

\[E = \begin{cases}
1 & \text{if}\; S=\hat{S}\\
0 & \text{if}\; S\neq \hat{S}
\end{cases}\]

\begin{align*}
H(S|\hat{S}) & = H(E,S|\hat{S})-\underbrace{H(E|S\hat{S})}_{=0}\\
& = H(E|\hat{S})+H(S|E\hat{S})\\
H(E|\hat{S}) & \leq H(E) \leq h_2(\epsilon) \qquad \text{assuming $\epsilon\leq 1/2$}\\
H(S|E\hat{S}) &= \underbrace{P_E(0)}_{\leq \epsilon} \underbrace{H(S|\hat{S})_{P_{S\hat{S}|E=0}}}_{\leq \log M}\\
& + P_E(1)\underbrace{H(S|\hat{S})_{P_{S\hat{S}|E=0}}}_{\leq \log M}\\
& \leq \log M
\end{align*}
\end{proof}
Back to the theorem:
\begin{align*}
\log M & \leq I(X:Y) + H(S|\hat{S})\\
& \leq \underbrace{I(X:Y)}_{\leq C(W)}+h_2(P_{err})+P_{err} \log M\\
\log M & \leq \frac{C(W)+h_2(P_{err})}{1-P_{err}}
\end{align*}
\end{proof}

\paragraph{Example} 
\begin{enumerate}
\item Identity channel on $\{1,...,N\}$. For any $M$-code:
\[\log M \leq \frac{\log N+h_2(P_{err})}{1-P_{err}}\]
We have already seen that
\begin{align*}
P_{err} \geq 1 - \frac{N}{M} & \Rightarrow \frac{M}{N} \leq \frac{1}{1-P_{err}}\\
& \Rightarrow \log M \leq \underbrace{\log N + \log \left( \frac{1}{1-P_{err}} \right)}_{\substack{\text{This was better}\\\text{but very specific}}}
\end{align*}
\item Binary symmetric channel $BSC_f^{\times n}$
\[\log M \leq \frac{C(BSC_f^{\times n})+h_2(P_{err})}{1-P_{err}}\]
We know that
\begin{align*}
C(BSC_f)&=1-h_2(f)\\
C(BSC_p^{\times n}) & = \max_{P_{X^n}} I(X^n:Y^n)
\end{align*}
Easy to find a lower bound:
\begin{align*}
P_{X_1...X_n} & =P_{X_1}\times ... \times P_{X_n}\\
C(BSC_f^{\times n}) & \geq I(X^n:Y^n)\\
\text{But there $X_i,Y_i$} & \text{ are mutually independant}\\
&=\sum_{i=1}^n I(X_i:Y_i)\\
\text{If take }P_{X_i}= \text{unif,} & \text{ then } I(X_i:Y_i)=C(BSC_f)\\
&=nC(BSC_f)
\end{align*}
\end{enumerate}

\begin{thm}
Given two channels
\[W_{Y_1|X_1}^1 \qquad W_{Y_2|X_2}^2\]
Define:
\[W_{Y_1Y_2|X_1X_2}^{12}(y_1y_2|x_1x_2)=W_{Y_1|X_1}^1(y_1|x_1) . W_{Y_2|X_2}^1(y_2|x_2)\]
Then:
\[C(W^{12})=C(W^1)+C(W^2)\]
\end{thm}
\begin{proof}
\begin{itemize}
\item Easy direction: $C(W^{12})\geq C(W^1)+C(W^2)$.
Choose $P_{X_1X_2}=P_{X_1}\times P_{X_2}$\\
Then $P_{X_1X_2Y_1Y_2}=P_{X_1Y_1}\times P_{X_2Y_2}$ using the definition of $W^2$.
So 
\[I(X_1X_2:Y_1Y_2) = \underbrace{I(X_1:Y_1)}_{\substack{\text{mutual information between input}\\\text{and output of $W^1$}}} + I(X_2:Y_2)\]
By taking the sup over $P_{X_1}$ and $P_{X_2}$
\[C(W^{12})\geq C(W^1)+C(W^2)\]

\item More difficult direction: $\leq$
Take a general $P_{X_1X_2}$, $X_1$ and $X_2$ \emph{not} independent.

\begin{align*}
I(X_1X_2:Y_1Y_2) & = H(Y_1Y_2)-H(Y_1Y_2|X_1X_2) \\
& \leq H(Y_1)+H(Y_2)-\sum_{x_1x_2}P_{X_1X_2}(x_1x_2)H(Y_1Y_2)_{P_{Y_1Y_2|X_1X_2=x_1x_2}} \\
P_{X_1X_2}(y_1y_2) & = W^1(y_1|x_1)W^2(y_2|x_2) \quad \text{by definition of }W^{12} \\
H(Y_1Y_2)_{P_{Y_1Y_2|X_1X_2=x_1x_2}} & = H(Y_1)_{P_{Y_1|X_1X_2=x_1x_2}} + H(Y_2)_{P_{Y_2|X_1X_2=x_1x_2}} \\
& = H(Y_1)_{P_{Y_1|X_1=x_1}}+  H(Y_2)_{P_{Y_2|X_2=x_2}}
\end{align*}
We get:
\begin{align*}
I(X_1X_2:Y_1Y_2) & \leq H(X_1) + H(Y_1|X_1)+H(Y_2) - H(Y_2|X_2)\\
& = I(X_1:Y_1)+I(X_2:Y_2)\leq C(W^1)+C(W^2)
\end{align*}


\end{itemize}
\end{proof}

\subsection{Achievability bound}
We had define $h_X(X)=\log_2 \frac{1}{P_X(X)}$ with $\mathbb{E}h_X(X)=H(X)$.

\begin{defi}
For $X \in \mathcal{X}$ and $Y \in \mathcal{Y}$, we define the mutual information density:
\[\text{for $x,y \in \mathcal{X} \times \mathcal{Y}$: } i_{XY}(x:y)=\log_2 \frac{P_{X|Y}(X|Y)}{P_Y(Y)}=\log_2\left( \frac{P_{XY}(x,y)}{P_X(x)P_Y(y)} \right)\]

If $P_{XY}(x,y)=0$ but $P_X(x)>0, P_Y(y)>0$, let $i_{XY}(x:y)=-\infty$; if $P_X(x)=0$ or $P_Y(y)=0$, then $i_{XY}(x:y)=+\infty$
\end{defi}
Observe that 
\begin{align*}
\underset{(X,Y)\sim P_{XY}}{\mathbb{E}} (i_{XY}(X:Y)) & = \sum_{(x,y)\in \mathcal{X}\times \mathcal{Y}} P_{XY}(x,y)i_{XY}(x:y)\\
&= I(X:Y)
\end{align*}

\paragraph{Example} $X^n=X_1X_2...X_n\in \{0,1\}^n$ uniform; $Y^n=Y_1...Y_n \in \{0,1\}^n$ where\\
$Y_i= \begin{cases}
X_i & \text{wp } 1-f\\
1 \oplus X_i & \text{wp } f
\end{cases}$. We assume $f<1/2$.

\begin{align*}
i_{X^nY^n}(x^n y^n) & = \log_2 \frac{P_{Y^n|X^n}(y^n|x^n)}{P_{Y^n}{y^n}}\\
& = \log_2 \frac{\prod_{i=1}^f (1-f)^{x_i \oplus y_i \oplus 1} f^{x_i \oplus y_i}}{2^{-n}}\\
& = n + \log_2 (1-f)^{n-d_{H}(x^n,y^n}f^{d_H(x^n,y^n)}\qquad \text{with } d_H(x^n,y^n)=|\{i\in [n]:x_i\neq y_i\}|\\
&= n + (n-d_H(x^n,y^n))\log_2(1-f)+d_H(x^n,y^n)\log_2 f\\
i_{X^nY^n}(X^n:Y^n) & = n+(n-d_H(X^n:Y^n))\log_2(1-f)+d_H(X^n:Y^n)\log_2 f\\
\mathbb{E}(i_{X^nY^n}(X^n:Y^n) & = n (1-h_2(f))
\end{align*}

\begin{thm}
Let $W$ be a channel input $\mathcal{X}$ and output $\mathcal{Y}$. For any $P_X$ on $\mathcal{X}$, define $P_{XY}(xy)=P_X(x)W(y|x)$ and any $\tau > 0$, there exists an $M$-code with
\[P_{err} \leq \mathbb{P}\big(i_{XY}(X:Y)< \log M + \tau\big)+2^{-\tau}\]
\end{thm}

This means that we can send $\log M$ bits with error probability lower than $\delta$ provided $\mathbb{P}(i_{XY}(X:Y)\leq \log M) \lessapprox \delta$ 

\begin{proof}
We need to construct a $(E,D)$.
\begin{align*}
P_{err} & = 1 - \frac{1}{M}\sum_{s=1}^{M}\sum_{y\in \mathcal{Y}} W(y|E(s))\mathbb{1}_{D(y)=s}\\
& = \frac{1}{M}\sum_{y\in \mathcal{Y}} W\big(y|E(D(y))\big)
\end{align*}

If we choose $D^*(y)=\underset{s\in [M]}{\text{argmax}}\; W(y|E(s))$, then for this $D^*$,
\[P_{err} = 1 - \frac{1}{M}\sum_{y\in \mathcal{Y}} \max_{s\in[M]} W(y|E(s))\]

This is optimal but not so easy to analyse. Instead we define a threshold and let $D(y)$ be the only $s$ above the threshold.

Recall that we had a distribution $P_X$ over $\mathcal{X}$. Define $P_{XY}(x,y)=P_X(x)W(y|x)$ and $P_Y(y)=\sum_{x}P_X(x)W(y|x)$. The threshold will be, if $W(y|E(s)) \lessapprox MP_Y(y)$

\[D(y)=\begin{cases}
s & \text{if there is a unique $s$ such that $i_{XY}(E(s):y)\geq \log M + \tau$} \\ x_0 & \text{otherwise} \end{cases}\]

For this $D$, we analyse the eror probability $P_{err}=\frac{1}{M}\sum_{s=1}{M}\underbrace{P_{err,s}}_{\substack{\text{error prob}\\\text{for msg $s$}}}$

\begin{align*}
\underbrace{P_{err,s}}_{= \mathbb{P}(\hat(S)\neq S | S = s)} & = \sum_{y\in \mathcal{Y}} W(y|E(s))\mathbb{1}_{D(y)\neq s}\\
& \leq \sum_{y\in \mathcal{Y}} W(y|E(s))\mathbb{1}_{(i_{XY}(E(s):y)<\log M + \tau) \text{ or } (\exists s'\neq s : i_{XY}(E(s'):y)\geq \log M + \tau)}\\
& \leq \underbrace{ \sum_{y\in \mathcal{Y}} W(y|E(s))\mathbb{1}_{i_{XY}(E(s):y)<\log M + \tau}}_{\underset{Y\sim W(\cdot|E(s))}{\mathbb{P}} (i_{XY}(E(s):Y)<\log M + \tau | S=s)} + \sum_{s'\neq s} \underbrace{ \sum_{y\in \mathcal{Y}} W(y|E(s))\mathbb{1}_{i_{XY}(E(s):y)\geq \log M + \tau}}_{\underset{Y\sim W(\cdot|E(s))}{\mathbb{P}} (i_{XY}(E(s'):y)\geq \log M + \tau)}
\end{align*}

\paragraph{Aside on $BSC_f$} $f<1/2$\\
In this case $i_{X^nY^n}(E(s):Y^n)=n+(n-d_H(E(s),Y^n))\log(1-f)+d_H(E(s),Y^n)\log_2 f$.
The optimal decoder is 
\[D^*(y^n)=\underset{s\in [M]}{d_H(E(s),y^n}\]
And the ``threshold" decoder is 
\[D(y^n) = \begin{cases} s & \text{if there is a unique $s$ s.t. $D_H(E(s),y)\leq \_\_\_ $} \\ \ast & \text{otherwise} \end{cases}\]

Now we choose $E$. $E(1),...,E(M)$ random according to $P_X$ and independent, and we compute the expectation of the error probability.

\begin{enumerate}
\item \begin{align*}
\underset{E(s)\sim P_X}{\mathbb{E}} \Big( \sum_{y \in \mathcal{Y}} W(y|E(s)) \mathbb{1}_{i_{XY}(E(s):y)<\log M + \tau} \Big) & = \sum_{x\in \mathcal{X}} P_X(x)\sum_{y\in \mathcal{Y}}W(y|x)\mathbb{1}_{i_{XY}(x:y)<\log M + \tau}\\
&= \mathbb{P}(i_{XY}(x:y)<\log M + \tau)
\end{align*}

\item \begin{align*}
\underset{\substack{E(s)\sim P_X \\ E(s')\sim P_y}}{\mathbb{E}} \Big( \sum_{y\in \mathcal{Y}} W(y|E(s))\mathbb{1}_{i_{XY}(E(s'):y)\geq \log M + \tau} \Big) & = \sum_{x,x',y} P_X(x)P_X(x')W(y|x)\mathbb{1}\underbrace{_{i_{XY}{x':y)\geq \log M + \tau}}}_{\frac{W(y|x')}{P_Y(y)}\geq M.2^{-\tau}}\\
& = \sum_{x',y} P_{Y}(y).P_X(x')\mathbb{1}_{W(y|x')\geq P_Y(y)M.2^\tau}\\
& \leq \sum_{x,y} W(y|x').\frac{2^{-\tau}}{M}.P_{X}(x')\\
& = \frac{2^{-\tau}}{M}
\end{align*}
\end{enumerate}

So overall, we have
\[ \mathbb{E}( P_{err} ) = \frac{1}{M} \sum_{s=1}^{M}\mathbb{E}(P_{err,s})\leq \mathbb{P}(i_{XY}(X:Y)<\log M + \tau) + \underbrace{(M-1)}_{\text{sum for $s\neq s'$}}\frac{2^{-\tau}}{M}\]

This implies that there exists an $M$-code with
\[P_{err}\leq \mathbb{P}(i_{XY}<\log M + \tau) + 2^{-\tau}\]
\end{proof}

\paragraph{Important special case} Memoryless channel $W^{\times n}$\\
Look at rate: $\frac{\log_2 M}{n}$.

%scheme

\begin{thm}[Shannon's noisy coding theorem]
Let $W$ be a channel. For any $\delta \in (0,1)$
\[C(W)\leq \lim_{n\to \infty} \frac{\log M^{OPT}(W^{\times n})}{n}\leq \frac{C(W)}{1-\delta}\]
\end{thm}

\begin{proof}
\begin{itemize}
\item For upper bound: Follows directly from converse and $C(W^{\times n})=nC(W)$
\item For lower bound, let $P_X$ be a distribution on $\mathcal{X}$ achieving $\max_{P_X} I(X:Y)$ (for channel $W$). We define $X_1,...,X_n$ $n$ independent random variables with distribution $P_X$, let $Y_1,...,Y_n$ the corresponding outputs.

\begin{align*}
i_{X^n:Y^n}(X^n:Y^n) & =\sum_{i=1}^n i_{XY}(X_i:Y_i) \quad\text{i.i.d.}\\
\mathbb{P}(i_{X^nY^n}(X^n:Y^n) \leq n(I(X:Y)-\epsilon ) & \underset{n \to \infty}{\to} 0 \qquad \text{WLLN}
\end{align*}
Take $M=\lceil 2^{n(I(X:Y)-2\epsilon)}\rceil$ and $\tau=n\epsilon$, then $\mathbb{P}(i_{X^nY^n}(X^n:Y^n)\leq \log M + \tau ) + 2^{-\tau} \leq 2^{-n\epsilon} + \frac{\delta}{2}\leq \delta$ for large enough $n$. 
\end{itemize}
\end{proof}

\paragraph{Comments:\\}
\begin{itemize}
\item It turns out that for any $\delta \in \{0,1\}$,
\[\lim_{n\to \infty} \frac{\log M^{OPT}(W^{\times n},\delta)}{n}=C(W)\]
\end{itemize}
\item One can obtain good finite $n$ bounds:
\[\frac{\log_2M^{OPT}(W^{\times n},\delta)}{n}=C(W)+\frac{Q\delta}{\sqrt{n}}+O(\frac{\log n}{n})\]


\paragraph{Zero error coding}
For the binary symmetric channel:
\[M^{OPT}(BSC^{\times n},0)=1\]


For $W(1|1)=1/2, W(2|1)=1/2, W(2|2)=1/2, W(3|2)=1/2, W(3|3)=1$, with the codebook $\{1,3\}$, we can decode with zero error.

For zero error, the relevant description of $W$ is the confusability graph.

\begin{align*}
G(W) = & \bullet \text{ vertices are channel input }\mathcal{X} \\
& \bullet (u,v) \text{ is on an edge if } \exists y \in \mathcal{Y}, W(y|x)>0 \text{ and } W(y|v)>0
\end{align*}

Then, \[M^{OPT}(W,0)=|\text{MaxIndSet}(G(W))|\]

Where an independent set of $G$ is a subset of vertices with no edge between them.

We can ask the question for a memoryless channel:

\[\lim_{n\to \infty} \frac{\log M^{OPT}(W^{\times n},0)}{n}\]

\begin{align*}
G(W^{\times n}) = & \text{Vertices indexed by } \mathcal{X}^n\\
& \text{Edges: }(x_1,...,x_n)\sim(x_1',...,x_n') & \big(\text{There is an edge}\\
& & \text{between $(x_1,...,x_n)$}\\
& & \text{and $(x_1',...,x_n')$}\big)
\end{align*}
\begin{align*}
& \exists y_1...y_n W^{\times n}(y_1...y_n|x_1...x_n)>0   \text{ and } W(y_1...y_n|x_1'...x_n')>0\\
\Leftrightarrow  & \exists y_1...y_n W(y_1|x_1)...W(y_n|x_n) > 0 \text{ and } W(y_1|x1')...W(y_n|x_n')>0\\
\Leftrightarrow & x_1 \sim x_1' \text{ and } x_2\sim x_2' ... \text { and } x_n\sim x_n' \text{ in graph } G(W)
\end{align*}

The question is then: given a graph $G$, how does
\[\text{MaxIndSet}(G^{\times n})\]
grow with n?

Given an independent set $I$ for $G$, then $I^n$ is an independent set of $G^{\times n}$.

\[I^n=\{(x_1,...,x_n) \text{ such that } x_i\in I\}\]
So MaxIndSet$(G^{\times n})\geq \text{MIS}(G)^n$

\paragraph{Famous example}
Let $C_5$ be the connected graph with 5 edges and 5 vertices. $\text{MIS}(C_5)=2$ but $\text{MIS}(C_5^{\times 2})=5$.
So $MIS(C_5^{\times 2n})\geq 5^n$.

\[\lim_{n\to \infty}\frac{\log M^{OPT}(C_5^{\times n},0)}{n}\geq \frac{1}{2}\log 5\]

It is possible to show that 
\[\lim_{n\to \infty}\frac{\log M^{OPT}(C_5^{\times n},0)}{n} = \frac{1}{2}\log 5 \qquad \text{Hard}\]


\section{Information and combinatorics}
\paragraph{Simple inequality}

\begin{lemma}[Shearer's lemma]
$(X_1,...,X_n),S_1,...,S_m \subseteq [n]=\{1,...,n\}$
Suppose that for all $i\in [n]$, $i$ appears in more ($\geq$) than $k$ sets, then:
\[H(X_1,...,X_n)\leq \frac{1}{k}\sum_{j=1}^{m}H(X_{S_j})\]
Where
\[H(X_S)=H(X_{e(1)}...X_{e(|S|)}) \qquad \text{with }S=\{e_1,...,e_n\}\]
\end{lemma}

\begin{proof}
\begin{align*}
H(X_1,..,X_n) & =H(X_1)+H(X_2|X_1)+...+H(X_n|X_1,...,X_{n-1})\\
S_j & =\{e_j(1),...,e_j(|S_j|)\} \qquad \text{ with } e_j(1)\leq e_j(2)\leq ...\\
H(X_{S_j}) & =H(X_{e_j(1)})+H(X_{e_j(2)}|X_{e_j(1)})+... \geq H(X_{e_j(1)}|X_1...X_{e_j(1)-1})+H(X_{e_j(2)}|X_1...X_{e_j(2)})
\end{align*}

For each $i\in [n]$, the term $H(X_i|X_1...X_{i-1})$ appears $k$ times int h lower bound on
\[\sum_{j=1}^m H(X_{S_j})\]
So we get the bound.
\end{proof}

\paragraph{Application 1} Projection of points sets.

$S$ set of $m$ points in $\mathbb{R}^3, S=\{a(1),...,a(m)\}, a(i)=\{a_i(1),a_i(2),a_i(3)\}$.
Define \\ $\Pi_{XY}=\{(a_1(i),a_2(i)\; i\in [m]\}$; $\Pi_{XZ}=\{(a_1(i),a_3(i)\; i\in [m]\}$, $\Pi_{YZ}=\{(a_2(i),a_3(i)\; i\in [m]\}$

Suppose $|\Pi_{XY}|,|\Pi_{XZ}|,|\Pi_{YZ}|\leq n$. How large can $m$ be?

\paragraph{Claim} If $S$ has $m$ points with projections of size $\leq n$, then $m\leq n^{2/3}$

\begin{proof}
\[P_{A_1A_2A_3}(a_1a_2a_3)=
\begin{cases}
\frac{1}{m} & \text{if } (a_1,a_2,a_3)\in S\\
0 & \text{otherwise}\\
\end{cases}\]
\[H(A_1A_2A_3)=\log m\]

The condition $|\Pi_{XY}|\leq n$ says $|H(A_1A_2)\leq \log n$. Using Shannon's lemma:
\begin{align*}
\log m & = H(A_1A_2A_3)\leq \frac{1}{2}(H(A_1A_2)+H(A_1A_3)+H(A_2A_3))\\
& = \frac{3}{2}\log n\\
m\leq n^{3/2}
\end{align*}
\end{proof}

\paragraph{Application 2} Number of independent sets in a graph

Let $n$ be the number of vertices of the graph. We look at $d$-regular graphs.

\begin{thm}
If $G$ is a bipartite $d$-regular graph with $n$ vertices, then
\[|\underbrace{I(G)}_{\text{Set of indep. sets of }G}|\leq (2^{d+1}-1)^{\frac{n}{2d}}\]
This bound is achieved by taking copies of bipartite complete graph.
\end{thm}

\begin{proof}
$[n]=\{1,...,n\}$ labels of vertices, $[n]=A\cup B$ with edges only on $A$ and $B$, $|A|\geq |B|$. Let $I$ be a uniformly random independent set in $I(G)$. Let $X_i=\mathbb{1}_{i\in I}$

\begin{align*}
H(X_1...X_n) & = \log | I(G) |\\
& = H(X_A)+H(X_B|X_A)\\
H(X_B|X_A)&\leq \sum_{b\in B}H(X_b|X_A)\\
&\leq \sum_{b\in B} H(X_b|X_{N(b)})\\
& \text{with }N(b)=\{ a\in A : (a,b)\in E\}
\end{align*}

Define 
\begin{align*}
Q_b & = \begin{cases}
1 & \text{if } |I\cap N(b)|=0\\
0 & \text{otherwise}
\end{cases}\\
& \leq \sum_{b} H(X_b | Q_b)\\
\\
H(X_b|Q_b) & = P_{Q_b}(0)H(P_{X_b|Q_b=0})\\
& \; + P_{Q_b}(1)H(P_{X_b|Q_b=1})\\
&\leq P_{Q_b}(1) = q_b\\
H(X_B|X_A) & \leq \sum_b q_b\\
H(X_B|X_A) & \leq \frac{1}{d}\sum_{b\in B} H(X_{N(b)}) \text{ using Shearer's lemma and degree }d
\end{align*}

Note that $H(X_{N(b)}Q_b)=H(X_{N(b)})$.
\begin{align*}
H(X_{N(b)}Q_b) & =H(Q_b)+H(X_{N(b)}|Q_b)\\
& = h_2(q_b)+P_{Q_b(0)}H(P_{X_{N(b)}|Q_b=0})\\
& \; + \underbrace{P_{Q_b(1)}H(P_{X_{N(b)}|Q_b=1})}_{=0}\\
& \leq h_2 (q_b)+(1-q_b)\log (2^d - 1)\\
H(X_1...X_n) & \leq \frac{1}{d} \sum_{b} h_2(q_{b})+(1-q_b)\log (2^d - 1) + \sum_{b} q_b
\end{align*}
It turns out that this is at most
\[\frac{n}{2d}\log (2^{d+1}-1)\]
For any $q_b\in [0,1]$ using fact that $|B|\leq \frac{n}{2}$
\end{proof}

\section{Error correcting code}
Shannon’s theorem says that for any nontrivial channels there are $M$-codes with $M \approx 2^{nC(W)}$
codewords that can be decoded with very small error probability given the output of the channel $W$. It even said that provided we pick the codewords at random with a good distribution, then most
codes are good. Our objective now is to explicitly construct good codes. The notion of a good code depends on the channel being studied and involves both the construction of an encoder and a decoder. To simplify the study it is useful to consider a different error model than the one we considered so far and in this model the existence of a decoder is directly related to a simple property of the codebook. Recall that in the Shannon model, an encoder is good if there exists a decoder that can decode with a small error probability. In the Hamming
model, a good encoder is one for which there is a decoder that can correct any error of weight at most $t$. The models are not exactly the same but they are related and we will see that it is possible to construct good codes in the Shannon sense using good codes in the Hamming sense.

\subsection{General error-correcting codes}
\begin{defi}
A code $C$ of blocklength $n$ over an alphabet $\Sigma$ is a subset of $\Sigma^n$. We usually write $q = |\Sigma|$.
The dimension of a code is defined as $k = \log_q |C|$.
\end{defi}


\paragraph{Remark} Note that a way to specify a code is as an injective encoding function $C$: $\Sigma^k \to \Sigma^n$ and the code corresponds to the image of the encoding function $C$. Even though they are not the same objects, we will be using the word “code” for both of these.
As mentioned before, we consider the Hamming error model where our objective is to be able to correct all errors of weight at most $t$. Note that if you want to think it terms of channels, you should see $\mathcal{X} = \mathcal{Y} = \Sigma$ and then taking $n$ copies of the channel for example.

\begin{defi}
$C$ is $t$-error correcting if there exists a decoding map $D: \Sigma^n \to C$ such that for
any $c \in C$ and any error pattern $e$ with at most $t$ errors $D(c + e) = c$.
\end{defi}

Let us look at simple examples
\begin{enumerate}
\item The repetition code $C_{rep} = \{000, 111\}$. This code has $q = 2$, $n = 3$, $k = 1$. It is $1$-error
correcting. In fact, my decoding function can map to 000 inputs of weight at most 1 and map
to 111 inputs of weight $\geq 2$.
\item The binary code defined by $C_\oplus(x_1x_2) = x_1x_2(x_1 \oplus x_2)$ has $q = 2, n = 3, k = 2$. It is not 1-error correcting. In fact $C_\oplus(00) = 000$ and $C_\oplus(01) = 011$. If I apply a weight 1 error to the first codewords I can get 010, but I can also get to 010 by applying a weight 1 error to the second codeword. So I can detect that there is an error but I cannot correct for it.
\end{enumerate}
From this example, one sees that the relevant parameter that governs how many errors a code can correct is the Hamming distance between the codewords.
\begin{defi}[Minimum distance of a code]
The Hamming distance between $u,v \in \Sigma^n$ is
defined by $\Delta(u, v) = |\{ i \in [n] : u_i \neq v_i\}|$.

The minimum distance (or just distance) of a code $C$ is defined as
\[
d = \min_{c,c' \in C, c\neq c'} \Delta (c, c') 
\]
\end{defi}


Note that in the Hamming distance, we do not have a notion of distance between two symbols in $\Sigma$ they are either the same or different. For example, if we think of $\Sigma = \{0, 1\}$ and consider the bitstrings $u = 0010$ and $v = 1110$, their Hamming distance is 2. However, if we consider $\Sigma = \{0, 1\}$ and consider $u, v \in \Sigma 2,$ then their Hamming distance is 1.

Let us look at the examples we considered before
\begin{enumerate}
\item The repetition code $C_{rep}$ has a minimum distance of 3
\item The code $C_{\oplus}$ has a minimum distance of 2. In fact, take two different codewords $c = C_\oplus (x_1x_2)$ and $c' = C_\oplus (y_1y_2)$. Then if $\Delta (x_1x_2, y_1y_2) = 2$, then $\Delta (c, c') \geq 2$. Otherwise, if $\Delta (x_1x_2, y_1y_2) = 1$, then $\Delta (c, c') = 2$.
\end{enumerate}

We now see that minimum distance is directly related to the number of errors that can be
corrected. We only do here the special case of d odd, the even case will be done in the tutorial.
\begin{prop}
Assume $d \geq 3$ is odd. Then the following are equivalent.
\begin{itemize}
\item $C$ has minimum distance $d$
\item $C$ can correct $\frac{d-1}{2}$ errors
\end{itemize}
\end{prop}

\begin{proof}
Suppose $C$ has minimum distance $d$. Then define the function $D : \Sigma^n \to C$ by $D(y) = \underset{c\in C}{\text{argmin}} \;\Delta(c, y)$. Then suppose $c_1$ is transmitted and $\Delta(c_1, y) \leq t$. Then let $D(y) = c$. We have $\Delta (c_1, c) \leq \Delta (c_1, y) + \Delta (y, c) \leq t + t$. This is equal to $2d$ provided $t = \frac{d-1}{2}$ . As such
$c = c_1$.

Now suppose $C$ has distance $\leq d - 1$. Then there exists $c_1, c_2 \in C$ with $\Delta (c_1, c_2) \leq d - 1$. Consider $y$ such that $\Delta (y, c_1), \Delta (y, c_2) \leq \frac{d-1}{2}$. This $y$ could be received for either $c_1$ or $c_2$ so $C$
cannot correct $\frac{d-1}{2}$ errors.
\end{proof}

\paragraph{Notation} We use the notation $(n, k, d)_q$-code when blocklength $n$, dimension $k$, minimum distance $d$ and the alphabet $\Sigma$ has size $q$. Let us see another less trivial code that we have already encountered in the first lecture. This is the Hamming code. It is also a binary code with $q = 2$. We may define it by
\[
C_H(x_1x_2x_3x_4) = (x_1, x_2, x_3, x_4, x1 \oplus x_2 \oplus x_4, x_1 \oplus x_3 \oplus x_4, x_2 \oplus x_3 \oplus x_4)
\]
This is a $(7, 4, d)_2$ code where we still have to determine $d$. I claim that the minimum distance is 3. First $0000000 \in C_H$ and $1000110 \in C_H$ and they are at distance 3. Moreover, for two different codewords $C_H(x)$ and $C_H(y)$, we can write
\begin{align*}
\Delta(C_H(x),C_H(y)) & = |\{ i \in [7] : C_H(x)_i \neq C_H(y)_i\}|\\
&= |\{i \in [7] : C_H(x)_i + C_H(y)_i \neq 0 \}|\\
&= | C_H(x) + C_H(y)|\\
&= |C_H(x + y)|\\
\end{align*}
as the mapping $C_H$ is a \emph{linear} map. So it suffices to determine $\min_{x\neq 0} |C_H(x)|$. We do this by considering the different cases for the Hamming weight of $x$. If $|x| = 1$, then two or three of the following bits evaluate to 1: $x_1 \oplus x_2 \oplus  x_4, x_1 \oplus x_3 \oplus x_4; x_2 \oplus x_3 \oplus x_4$. If $|x| = 2$, then at least one of these bits evaluates to 1 and if $|x| = 3$, we already have $|C_H(x)| \geq 3$. We conclude that $C_H$ is a $(7, 4, 3)_2$ code.

Note that this code has a very nice property that we will be exploiting further. The encoding
function is a linear function. In fact, we can see messages as elements of $\mathbb{F}^4_2$
and codewords as elements of $\mathbb{F}^7_2$ and the transformation is given by a matrix
\[
G_H = \left(
\begin{matrix}
1& 0& 0& 0& 1& 1& 0\\
0& 1& 0& 0& 1& 0& 1\\
0& 0& 1& 0& 0& 1& 1\\
0& 0& 0& 1& 1& 1& 1\\
\end{matrix}
\right)
\]
and $C_H(x) = xG_H$ where we see $x$ as a row vector in $\mathbb{F}^4_2$. One can in general define linear codes
whenever $\Sigma$ has a field structure so that $\Sigma^n$ is a vector space over the field $\Sigma$ and $C \subseteq \Sigma^n$ is a subspace. Before getting into the detailed study of linear codes, let us determine some simple bounds on the best parameters one can achieve for codes.

\subsubsection{General bounds on the best codes}
For a fixed $n$ and $q$, we would like $k$ and $d$ to be as large as possible. For example, the Hamming code is a $(7, 4, 3)_2$ code, is it possible to improve it to a $(7, 5, 3)_2$ code for example? The answer is no by the following simple packing bound. Again, we only state here a simplified for with $q = 2$ and $d = 3$ but it is easy to generalize (see tutorial).

\begin{thm}[Hamming bound (special case)]
Every binary code with blocklength $n$, dimension
$k$ and distance $d = 3$ satisfies
\[
k \leq n - log_2(n + 1)
\]
For $n = 7$ and $d = 3$, this gives $k \leq 4$, which means the Hamming code is optimal in this sense.
\end{thm}

\begin{proof}
Let $C$ be such a code and $c_1, c_2$ be two codewords. For $u \in \{0, 1\}^n$, let $B(u, 1) = \{v \in \{0, 1\}^n : \Delta (u, v) \leq 1 \}$. We have $B(c_1, 1) \cap B(c_2, 1) = \emptyset$. In addition$|B(u, 1)| = 1 + n$. As a result,
\[
| \bigcup\limits_{c\in C} B(c, 1)| = (n + 1)2^k
\]
But clearly this number is at most the size of the whole space which is $2^n$. So
\[
k \leq n - log_2(n + 1) :
\]
\end{proof}
Note that having equality in this bound means that we have perfect packing, i.e., $\bigcup_{c\in C} B(c, 1) = \{0, 1\}^n $
Such codes are called perfect codes.


\begin{thm}
Let $q\leq 2$, $1\leq d \leq n$. There exists a $(n,k,d)_q$-code with $k\geq n - \log_q$ $Vol_q(d-1,n)$
\end{thm}

\begin{proof}
Greedily construct $C$.

\begin{algorithm}
$C=\emptyset$\\
\While{There is $x\in \Sigma^n$ with $\Delta(x,c)\geq d$ for all $c\in C$}{
	$C\leftarrow C \cup \{x\}$
}
\end{algorithm}

Clearly at any time $C$ has minimum distance $\geq d$.

When the algorithm terminates :

\begin{align*}
\forall x \in \Sigma^n, \exists c \in C : \Delta (x,c) & \leq d-1\\
\Sigma^n & \subseteq \bigcup_{c\in C} B(c, d-1)\\
q^n \leq |\bigcup_{c\in C} B(c,d-1)| \leq \sum_{c\in C} |B(c,d-1)| & = |C| Vol_q(d-1, n)\\
& = q^k Vol_q(d-1,n)
\end{align*}
So $k\geq n - \log_q Vol_q (d-1,n)$
\end{proof}

\subsection{Linear error correcting code}


\begin{thm}
The size of any finite field\footnote{fr : corps} is $q=p^s$ for some prime $p$ and integer $s\geq 1$. Moreover, there is a unique field of size $q$ denoted $\mathbb{F}_q$.
\end{thm}

\begin{itemize}
\item For $p=q$, $\mathbb{F}_q$ can be seen as integers mod $p$ with the usual addition and multiplication.
\item For $q=p^s$, elements of $\mathbb{F}_q$ are polynomials in $\mathbb{F}_p [X]$ modulo an irreducible polynomial $Q\in \mathbb{F}_p[X]$ of degree $s$.
\end{itemize}

\begin{defi}
Let $q$ be a prime power.
$C \subset\mathbb{F}_q^n$ is a linear code if it is a linear subspace of $\mathbb{F_q}^n$, i.e., if $x,y\in C$, $x+y \in C$ and $a.x\in C$ for $a\in \mathbb{F}_q$.
\paragraph{Notation} $[n,k,d]_q$, with $k$ the dimension and $d$ the distance.
\end{defi}

\paragraph{Example} Repetition code $C=\{000,111\}$ is a linear code over $\mathbb{F}_2$. This forms a $[3,1,3]_2$ code.

\begin{prop}
Let $S$ be a linear subspace of $\mathbf{F}_q^n$.

\begin{enumerate}
\item $|S|=q^k$ for $k$ integer
\item There exists a basis $v_1, v_2,...,v_k$ such that for any $x\in S$,there is unique $(a_1,..a_k)\in \mathbf{F}_q^k$ such that $x=\sum_{i=1}^k a_i \vec{v_i}$

Then the $k\times n$ matrix
\[G= \left( \begin{matrix}
\leftarrow & v_1 & \rightarrow\\
& \vdots & \\
\leftarrow & v_k & \rightarrow\\
\end{matrix}\right) \qquad x=(a_1,...,a_k)\]
is called a generator matrix. Note that rows of $G$ are linearly independent and so $G$ has full rank.
\item There exists a full rank $(n-k)\times n$ matrix called parity check matrix such that for all $x \in S$, $Hx^T=0_{n-k}$
\end{enumerate}
\end{prop}

\paragraph{Example} for repetition code

$G= \left(\begin{matrix} 1&1&1 \end{matrix}\right)$ and $H= \left(\begin{matrix} 1&1&0\\1&0&1 \end{matrix}\right)$

\[ \left(\begin{matrix} 0\\0 \end{matrix}\right) = \left(\begin{matrix} 1&1&0\\1&0&1 \end{matrix}\right)
\left(\begin{matrix} x_1&x_2&x_3 \end{matrix}\right) =
\left(\begin{matrix} x_1+x_2\\ x_1+x_3 \end{matrix}\right)\]

\begin{proof}[Sketch of proof]
To construct a basis, can do it in a greedy way. Take $v_1 \in S$ non-zero. Then, at step $t$, $v_t \notin \left\{ \sum_{i=1}^{t-1} a_i v_i : a_i 
\in \mathbb{F}_q \right\}$.

We obtain $v_1,...,v_k$ : It is clear that $v_1,...,v_k$ generates $S$. Also by induction, it is simple to show that $\left\{ \sum_{i=1}^{t-1} a_i v_i \right\}$ contains exactly $q^t$ elements.

\[N= \left\{ y \in \mathbb{F}_q^n : \sum_{i=1}^n x_iy_i=0 \; \forall x \in S \right\}\]

$N$ is a linear subspace of $\mathbb{F}_q^n$. To obtain a parity check matrix, take a basis of $N$.
\end{proof}

\subsubsection*{Minimum distance of a linear code}

\begin{prop}
The minimum distance of a linear code $C$ is given by $d= \min_{c\in C|c\neq 0} |c|$ where $|c|=|\{ i \in [n] : c_i\neq 0 \}$
\end{prop}

\begin{proof}
$0\in C$ and $\Delta(0,c)=|c|$ so the minimum distance is at most $\min_{c\in C|c\neq 0} |c|$.

For $c_1\neq c_2 \; c_1,c_2 \in C$

\[\Delta(c_1,c_2)=|c_1-c_2|\geq \min_{\substack{c\in C\\c\neq 0}}\]
\end{proof}

\begin{prop}
Let $C$ be an $[n,k,d]_q$ code with parity check matrix $H=\begin{pmatrix}
\uparrow & \uparrow & & \uparrow\\
H^1 & H^2 & \hdots & H^n\\
\downarrow & \downarrow & ... & \downarrow\\
\end{pmatrix}$.

Let $t=$ minimum number of linearly dependent columns.

Then 
\[d=t\]
\end{prop}

\begin{proof}
\begin{itemize}
\item Begin with $t\leq d$.

Let $c$ be a codeword with $|c|=d$. Then $Hc^T = 0$. But $Hc^T = \sum_{i=1}^n c_i H^i$.

The support of $c$ gives $d$ linearly dependent columns of $H$.

\item For $t\geq d$, let $H^{i_1},...,H^{i_t}$ be linearly dependent. There exits $C_{i_1},...,C_{i_t}$ such that \\$\sum_{j=1}^t c_i H^{i_j}=0$. Define $x\mathbb{F}_q^n$ with $x_{i_j}=c_{i_j}$ for all $j$ and $x_i=0$ otherwise.

Then $x\in C$ as $Hx^T=0$ and $|x|=t$ so $d\leq t$
\end{itemize}
\end{proof}

\paragraph{Example} Generalized Hamming codes.

$q=2$. For $n\geq 3$,

$H = 
\begin{pmatrix}
H_r^1 & \hdots & H_r^{2^r - 1}
\end{pmatrix}
$ where $H_r^i$ is the binary representation of $i$ of length $r$.

$H_3=
\begin{pmatrix}
0 & 0 & 0 & 1 & 1 & 1 & 1\\
0 & 1 & 1 & 0 & 0 & 1 & 1\\
1 & 0 & 1 & 0 & 1 & 0 & 1\\
\end{pmatrix}
$

$H_r$ has rank $r$ because $\begin{pmatrix}
0\\
\vdots\\
0\\
1\\
\end{pmatrix},
... ,
\begin{pmatrix}
1\\
0\\
\vdots\\
0\\
\end{pmatrix}
$ are linearly independent.

$H_r$ defined a $[2^r -1, 2^r - 1 - r, ?]_2$-code.

For $r=3$, we know the min distance is 3.

\paragraph{Claim} $H_r$ defines a $[2^r-1, 2^r-1-r,3]_2$-code.


\begin{proof}
\begin{itemize}
\item $H_r^1, H_r^2$ and $H_r^3$ satisfy $H_r^1+H_r^2+H_r^3 = 0$. So $d\leq 3$.
\item In addition, a distance of 2 would mean that there is a pair $i\neq j$ with $H_r^i + H_r^j = 0$. But this would implies that $i=j$. So $d\geq 3$.
\end{itemize}
\end{proof}

Rate of this code = $\frac{2^r-r-1}{2^r-1}$ very close to 1 but min distance 3 is poor.

\subsubsection*{Dual code of a linear code}
\begin{defi}
Let $C$ be a linear code with parity check matrix $H$. The code with generator matrix $H$ is called $C^\perp$ dual code. 
\end{defi}

If $C$ is an $[n,k]_q$-code, $C^\perp$ is an $[n,n-k]_q$-code.

Dual code of Hamming code $C_{Ham,r}$


\[\begin{pmatrix}
x_1 & x_2 & x_3
\end{pmatrix}
\begin{pmatrix}
0 & 0 & 0 & 1 & 1 & 1 & 1\\
0 & 1 & 1 & 0 & 0 & 1 & 1\\
1 & 0 & 1 & 0 & 1 & 0 & 1\\
\end{pmatrix}
\]

Let's call $C_{Sim,r}=C^\perp_{Ham,r}$. One encoding function $f \; C_{Sim,r}$ is given by $C_{Sim,r}(x)=xH_r$. We define $C_{Had,r}(x)=x
\begin{pmatrix}
0 & & & \\
\vdots & h_r^1 & \hdots & H_r^{2^r - 1}\\
0 & & &\\
\end{pmatrix}
$ Hadamard code.


\begin{prop}
The minimum distance of codes $C_{Sim,r}$ and $C_{Had,r}$ is $2^{r-1}$
\end{prop}

\begin{proof}
Sufficient to prove it for $C_Had,r$.
\paragraph{Claim} For any $c\in C_{Had, r}, c\neq 0, |c|=2^{r-1}$. For any $c\in C_{Had,r}, c\neq 0$, there exists $x\neq 0 \in \mathbb{F}_q^r$ such that $c=(xH_r^0,...,xH_r^{2^r-1})$.
We can write $c=(<x,u>)_{u\in \{0,1\}^n}$
As $x\neq 0$, $\exists i, x_i=1$. Let $e_i=(0...0 \underbrace{1}_i 0...0) \in \mathbb{F}_q^r$.\\
$v=u+e_i$\\
$<x,v>=<x,u>+<x,e_i>=<x,u>+1$\\
So components $<x,v>$ and $<x,u>$ are distinct.
\end{proof}


\subsubsection*{Encoding and decoding a linear code}
\paragraph{Encoding} To an $[n,k,d]$-code $C$, we can associate a neutral encoding function.

Take $G$ a generator matrix for $C$. Let the set of messages be $\mathbb{F}_q^k$.

The encoding function is 
\begin{align*}
C: &\mathbb{F}_q^k \to \mathbb{F}_q^n\\
&a \mapsto aG\\
\end{align*}
Which can be computed in $n.k$ operations in $\mathbb{F}_q$

\paragraph{Decoding}
\begin{itemize}
\item Error detection: with parity check matrix $H$, costs $n.(n-k)$ operation in general
\item Detection:\\
Start with $x \in C$\\
Error $e\in \mathbb{F}_q^n$\\
Receive: $y=x+e\in \mathbb{F}_q^n$\\
But
\begin{align*}
\underbrace{Hy^T}_{\text{syndrome}} = \underbrace{Hx^T}_{=0} + He^T = He^T\\
\end{align*}
\end{itemize}

\paragraph{Algorithm} Generic decoding linear code


\begin{algorithm}
\Input{$y \in \mathbb{F}_q^k$}
\Output{$x\in C$}
\For{$i=0$ to $t$}{
	\For{$e \in \mathbb{F}_q^k$ of weight $i$}{
		\If{$He^T=Hy^T$}{
			return $y-e$\\
		}
	}
}
\end{algorithm}

\[ \text{Number of steps}=\sum_{i=0}^t  \binom{n}{i} (q-1)^i\]

Polynomial for $t$ constant, but exponential in $t$.

\paragraph{Ex of Hamming code} $[2^r-1,2^r-1-r,3]_2$-code, Parity check matrix $H_r$.
\begin{itemize}
\item Start by computing syndrome: $s=H_ry^T$
\item Want to find $e$ of weight $\leq 1$ such that $s=H_re^T$
\end{itemize}
If we have an error in position $i$, $e_i=(0...0\underbrace{1}_{i}0...0)$

\begin{align*}
H_re_i^T & = H_r^i \qquad \text{$i$-th column of $H_r$}\\
&= i \text{written in binary}\\
\; \\
H_r &= 
\underbrace{
\begin{pmatrix}
\; & \; \\
\; & \; \\
\end{pmatrix}}_{2^r -1 } \Bigg\} r\\
\end{align*}

\paragraph{Decoding} Interpret $s\in \{0,1\}^r$ as a number between 1 and $2^r-1$ and flip corresponding bit.

In general, the problem of decoding is : Find $e\in \mathbb{F}_q^k$ of smallest weight s.t.
\[He^T=s\]

\subsection{Reed-Solomon codes}
Based on univariate polynomials.

\begin{align*}
f_m(X) & =\sum_{i=0}^d m_iX^i \in \mathbb{F}_q[X] \qquad m_i\in \mathbb{F}_q \\
\text{deg}\; f_m & = d \text{ if }m_d\neq 0\\
\end{align*}

\begin{defi}
We assume $1\leq k \leq n \leq q$. Let $\alpha_1, \alpha_2, ..., \alpha_n \in \mathbb{F}_q$ distinct.

The Reed-Solomon code is :
\begin{align*}
RS &: \mathbb{F}_q^k \to \mathbb{F}_q^m\\
RS(\underbrace{m}_{m=(m_0,...,m_{k-1})}) & = (f_m(\alpha_1),...,f_m(\alpha_n) )\\
\end{align*}
\end{defi}

For $n^0, n^1 \in \mathbb{F}_q^k$:
\[f_{n^0}(X) + f_{n^1}(X) = f_{n^0 + n^1}(X)\]

So 
\begin{align*}
RS(m^0) + RS(m^1) & = RS(m^0 + m^1)\\
\text{For $a\in \mathbb{F}_q$ } RS(am) &= aRS(m)\\
\end{align*}

RS is a linear code.

\begin{prop}
The minimum distance of RS is
\[n-k+1\]
\end{prop}

\begin{proof}
\begin{align*}
RS(m) & = \big( f_m(\alpha_1),...,f_m(\alpha_n) \big)\\
\text{Weight: } |RS(m)|&= | \{ i\in [n] : f_n(\alpha_i)\neq 0 \} |\\
& = n - |\{ i \in [n]: f_m(\alpha_i) = 0\}|
\end{align*}
But if $m\neq 0$ then $f_m$ is a non-zero polynomial of degree $\leq k-1$. So $|\{ i \in [n]: f_m(\alpha_i) = 0 \} | \leq k-1$. So $|\{i \in [n] : f_m(\alpha_i)=0\}|\leq k-1$

\paragraph{Important fact} A nonzero polynomial of degree $k-1$ has at most $k-1$ roots.

So $|RS(m)|\geq n-k+1$ : RS are $[n,k,n-k+1]_q$-codes.
\end{proof}

This minimum distance is optimal as it achieves the Singleton bound (see tutorial).

Ex of generator matrix for RS: Take basis: $1,X,...,X^{k-1}$
\[ G =
\begin{pmatrix}
1 & 1 & ... & 1\\
\alpha_1 & \alpha_2 & ... & \alpha_n \\
\alpha_1^2 & \alpha_2^2 & ... & \alpha_n^2 \\
\vdots & \vdots & \vdots & \vdots \\
\alpha_1^{k-1} & \alpha_2^{k-1} & ... & \alpha_n^{k-1} \\
\end{pmatrix}\]


\paragraph{Efficient decoding of RS-codes} Message to be send: $P$. Given $y$, We want to find $P$ such that:
\[\Delta \left( \big( P(\alpha_1),...,P(\alpha_n) \big) , y  \right) \leq t\]

Where $t=\left\lfloor \frac{d-1}{2} \right\rfloor $ where $d = n - k + 1 =$ minimum distance.

This is a polynomial interpolation problem with errors.

Introduce 
\[E(X)= \prod_{\substack{i=1 \\ y_i \neq P(\alpha_i)}}^n (X-\alpha_i) \tag{error locator poly}\]
\[\text{deg }E\leq t\]

\paragraph{Claim} We have for all $i\in [n]$
\[ y_i E(\alpha_i)=P(\alpha_i)E(\alpha_i) \]

Reason: If $E(\alpha_i)=0$, clearly satisfied, if $E(\alpha_i)\neq 0$, there is no error at position $i$ and so $P(\alpha i)=y_i$.

We have $n$ equations, and the number of variable is:
\begin{align*}
\text{\# variables}\; & = \;\text{at most $t$ for $E$ and at most $k$ for $P$}\\
& \leq t + k\\
& \leq \frac{(n-k+1)-1}{2}+k\\
& = \frac{n+k}{2}\\
\end{align*}

But these equations are \emph{not} linear in the variables.

\paragraph{Idea} Relax the equation to
\[ y_i E(\alpha_i) = N(\alpha_i) \;\text{with $N$ polynomial of degree $\leq k-1+t$}\]

\begin{algorithm}
\caption{Decoding of RS}
\Input{ $(y_1,...,y_n) \in \mathbb{F}_q^n$ with promise $\min_m \Delta(y,RS(m))\leq t$}
\Output{ $P$ polynomial of degree $\leq k-1$}
Solve $y_iE(\alpha_i) = N(\alpha_i) \;(\star)$ where variables are $e_0,...,e_{t-1}$ and $E(X)=e_0+e_1X+...+e_{t-1}X^{t-1}+X^t$ and $n_0,...,n_{t+k-1}$ and $N(X)=n_0+n_1X +...+n_{t+k-1}X^{t-k+1}$.\\
\If{no solution or $E$ does not divides $N$}{
	Fail\\
}
Return $P(X)=\frac{N(X)}{E(X)}$
\end{algorithm}

\paragraph{Running time} Solving the linear system can be done in $O(n^3)$

\begin{proof}{Correctness:} First, we show that $(\star)$ has a valid solution.
\begin{align*}
RS(m) & = \Big( f_m(\alpha_1),...,f_m(\alpha_n) \Big)\\
\Delta(RS(m),y) & \leq t\\
\end{align*}
Define
\begin{align*}
E^*(X) & = \prod_{i:y_i \neq f_m(\alpha_i)} (X-\alpha_i).X^{t-\Delta (9y,RS(m))} \\
\text{and } N^*(X) & = f_m (X) E^*(X)\\
\text{We have}\; y_i.E^*(\alpha_i) & = N^*(\alpha_i)\\
\end{align*}
For this solution $E^*$ divides $N^*$ and we output $f_m$.

Let's show that this solution is unique. Let $(N_1, E_1)$ and $(N_2, E_2)$ be solutions of $(\star)$.

\begin{align*}
R(X) & = N_1(X)E_2(X)-N_2(X)E_1(X)\\
\deg R & \leq (k+t-1) + t = 2t + k - 1\\
\text{recall}\; t &= \left\lfloor \frac{n-k+1-1}{2} \right\rfloor\\
2t+k-1 & = n-k+k-1 = n-1 \;\text{(if $n-k$ even)}\\
\end{align*}
In all cases
\[2t + k -1 < n\]
On the other hand
\begin{align*}
N_2(\alpha_i) & = y_i. E_2(\alpha_i)\\
N_1(\alpha_i) & = y_i. E_1(\alpha_i)\\
\end{align*}

So $E_1(\alpha_i)N_2(\alpha_i)=E_2(\alpha_i)N_1(\alpha_i)$, so $R_1(\alpha_i)=0$ for all $i$, $\Rightarrow$ $R$ has $n$  distinct roots, so $R=0$.

So $\frac{N_2(X)}{E_2(X)}=\frac{N_1(X)}{E_1(X)}$ 

\end{proof}


\paragraph{Objective}
``Good" binary codes: $k=\Omega (n)$, $d=\Omega (n)$, explicit and efficient encoding/decoding. As the Reed-Solomon codes, $[n,k,n-k+1]_q$, they are optimal (achieving the singleton bound).

\paragraph{Issue}$q\geq n$, the alphabet size should be large.


\subsection{Concatenation of codes}

Code $C$ on alphabet $[q]=\{1,...,q\}$ with blocklength $(x_1,...,x_n \in C)$. Assume $q=2^t$. We can interpret $(x_1,...,x_n)$ as $(x_{11},x_{12},...,x_{1t},x_{21},...,x_{n,1},...,x_{nt})\in \{0,1\}^{nt}$.

This procedure gives a binary code with blocklength $nt$ and dimension $kt$ ($2^{kt}$ codewords).

Consider $\left[ n,\frac{n}{2},\frac{n}{2}+1 \right]_n$ RS code, we will obtain a $\left( n \log_2 n, \frac{n}{2}\log_2 n, ? \right)_2$ code.

\bigskip
Let $k=\log n$

\begin{align*}
\Delta(x_{1,1},...,x_{1,t},...,x_{n,1},...,x_{n,t},y_{1,1},...,y_{n,t}) & = |\{(i,j)\in [n]\times[t]:x_{i,j} \neq y_{i,j} \}|\\
& \geq |\{i\in [n]: x_i \neq y_i \}|\\
& \geq \frac{n}{2}+1 \tag{Distance of our original code}\\
\end{align*}


We obtained a $\left( n \log_2 n, \frac{n}{2}\log_2 n, \frac{n}{2} + 1 \right)_2$.


\paragraph{Idea}
Instead of trivial representation: $x_i\to x_{i,1},...,x_{i,t}$ (binary representation), we will use a \emph{code}.


\begin{defi}[Concatenation Code]
Let $C_{out} : [Q]^K\to [Q]^N$ a $(N,K,D)_Q$ code and $C_{in}: [q]^k \to [q]^n$ be a $(n,k,d)_q$ code with $Q=q^k$.

Then the concatenation $C_{out} \circ C_{in}$ is a code on alphabet $[q]$ blocklength $nN$, dimension $kK$ defined by
\begin{align*}
C &: [Q]^K\to [q]^{nN}\\
C(m) & = \Big( C_{in}\big(C_{out}(m)_1\big),C_{in}\big(C_{out}(m)_2\big),...,C_{in}\big(C_{out}(m)_N\big) \Big)
\end{align*}
Where $C_{out}(m)_i$ is the $i$-th symbol of $C_{out}(m)$

\end{defi}


In the example: $C_{out}$: RS $\left[ N, \frac{N}{2}, \frac{N}{2}+1\right]_N$.\\
$C_{in}$ is $(n=\log N, k= \log N, d=1)_2$, and $C_{in}(x)=x$ the trivial code.

\paragraph{Remark}
We have identified $[Q]$ with $[q]^k$. For that, we can take any bijection between the sets.

When $C_{in}$ and $C_{out}$ are \emph{linear} codes, we can take this bijection so that $C_{out}\circ C_{in}$ is also a linear code.

In this, we use $[Q]=\mathbb{F}_{q^k}$ ($Q=q^k$) and $[q]^k=\big( \mathbb{F}_q\big) ^k$. $\mathbb{F}_{q^k}$ can be seen as a vector space over $\mathbb{F}_q$.

Let $\sigma : \mathbb{F}_{q^k}\to \big( \mathbb{F}_q \big)^k$ be an isomorphism, $G_{in}$ and $G_{out}$ generator matrices for $C_{in}$ and $C_{out}$.


\[G_{out \,\circ\, in} =\]
\[
\underset{\substack{\\\uparrow \\ (\mathbb{F}_q)^{kK}}}{}
\begin{pmatrix}
\sigma^{-1} & \\
0 & \ddots & \\
& & \sigma^{-1} \\
\end{pmatrix}
\underset{\substack{\\\uparrow \\ (\mathbb{F}_{q^k})^{K}}}{}
G_{out}
\underset{\substack{\\\uparrow \\ (\mathbb{F}_{q^k})^{N}}}{}
\begin{pmatrix}
\sigma & \\
0& \ddots &0 \\
& & \sigma \\
\end{pmatrix}
\underset{\substack{\\\uparrow \\ (\mathbb{F}_q)^{kN}}}{}
\begin{pmatrix}
G_{in} & & \\
0 & \ddots & 0 \\
&  & G_{in} \\
\end{pmatrix}
\underset{\substack{\\\uparrow \\ (\mathbb{F}_q)^{nN}}}{}
\]

\begin{prop}
If $C_{out}$ is $(N,J,D)_{q^k}$ and $C_{in}$ is $(n,k,d)_q$, then $C_{out\,\circ\, in}$ is a $(Nn,Kk,Dd)_q$ code.
\end{prop}

\begin{proof}
Let $m\neq m'\in [q^k]^K$ with $\Delta \Big(C_{out}(m),C_{out}(m') \Big) \geq D$.

If $C_{out}(m)_i\neq C_{out}(m')_i$, then $\Delta\Big( C_{in} \big( C_{out}(m)_i\big),C_{in}\big( C_{out}(m')_i \big) \Big) \geq d$, so
\begin{align*}
\Delta \big( C_{in \,\circ \, out}(m),C_{in \,\circ \, out}(m')\big) & = \sum_{\underbrace{i:C_{out}(m)\neq C_{out}(m')}_{D}} \Delta\Big( C_{in} \big( C_{out}(m)_i\big),C_{in}\big( C_{out}(m')_i \big) \Big)\\
& \geq Dd
\end{align*}
\end{proof}

To construct a good code it remains to find a good \emph{inner} code. What have we gained ? $\to$ Inner code is ``small", so we can more easily find a good one.


\paragraph{Explicit construction}
Explicit here means can construct code in time polynomial in the blocklength.

We construct $G_{in}$ and $G_{out}$:
\begin{itemize}
\item For $G_{in}$: RS code $[\left[ N, \frac{N}{2}, \frac{N}{2}+1\right]_N$ with $N=2^k$. $G_{out}$ is a Vandermonde matrix, so we can construct $G_{out}$ in $O(N^2)$ steps.
\item For $G_{in}$: Should have dimension $k$. We construct a code achieving Gilbert-Varshamov bound (See homework).
\end{itemize}

For example, cam construct a parity check matrix for a code with parameters $[n=2k,k,d=0.1n]_2$. This algorithm takes $O(2^{2k} poly(k)$ steps $=O\big(N^2 poly(\log N) \big)$
So we can get $G_{out\,\circ \, in}$ in time $poly(N)$.

$C_{out \,\circ\, in}$ is a $\left[ N \cdot 2\log N,\frac{N}{2}\log N, \left( \frac{N}{2}+1 \right) (0.2 \log N\right]_2$

\paragraph{Decoding a concatenated code}
$D_{C_{in}}$, $D_{C_{out}}$ for $(y_1,...,y_n)\in (\mathbb{F}_q)^N$

\[D_{C_{out\,\circ\, in}}(y_1,...,y_N)=D_{C_{out}}(D_{C_{in}}(y_1),...,D_{C_{in}}(y_n))\]


\paragraph{Running time}
\[N\cdot \underbrace{Cost(D_{C_{in}})}_{\substack{\text{generic decoder runs in}\\ O(2^{2k}poly(k))=O(N^2poly(\log(N)))}} + \qquad \underbrace{Cost(D_{C_{out}})}_{\text{For RS, } O(N^3)} \]

\begin{prop}
The algorithm $D_{C_{out \,\circ\, in}}$ can correct $<\frac{Dd}{4}$ errors.
\end{prop}
\begin{proof}
Let $m$ be such that 
\[\Delta( C_{out \,\circ \, in (m)},y) < \frac{dD}{4}\]
We want to show that we return $m$.

We define $B=\{i\in [N] \; D_{C_{in}}(y_i) \neq C_{out}(m)_i\}$.

\begin{itemize}
\item If $|B|<\frac{D}{2}$ then $D_{C_{out}}$ can correct the errors and returns $m$
\item Otherwise, if $|B|\geq \frac{D}{2}$, if $i\in B$, $\Delta(y_i,C_{in}(C_{out}(m)_i))\geq \frac{d}{2}$\\
So $\Delta \Big( (y_1,...,y_N),C_{in}(C_{out}(m)_1)...C_{in}(C_{out}(m)_N) \Big) \geq \frac{Dd}{4} \; \to \; $ contradiction
\end{itemize}
\end{proof}


\subsection{An application of ECC}

\paragraph{Communication complexity}
\[\underset{x\in \{0,1\}^n}{\text{Alice}} \qquad \qquad \underset{y\in \{0,1\}^n}{\text{Bob}} \]

\paragraph{Example}
\begin{itemize}
\item $PAR(x,y)=\sum_{i=1}^n (x_i+y_i) \mod 2$\\
Alice sends parity $\sum_i x_i \mod 2$ to Bob, and bob computes $\Big( \sum_i x_i \Big) + \Big( \sum_i y_i \Big)$
\item $EQ(x,y)=
\begin{cases}
0 & \text{if } x\neq y\\
1 & \text{if } x= y
\end{cases}$

Alice sends $x$ to Bob, Bob compute $EQ(x,y)$ and sends back the result to Alice.
\[Cost = n+1 \text{ bits}\]
\end{itemize}

\begin{defi}[Communication complexity of $f$]
\[D(f)=\min_{\mathcal{P}\text{ protocol compting } f} Cost( \mathcal{P}) \]
\end{defi}

$Cost(\mathcal{P})$ is the maximum over all input $x_{ij}$ of the number of bits communicated by applying $\mathcal{P}$ on inputs $(x,y)$.

We have seen $D(PAR)\leq 2$ and $D(EQ) \leq n+1$

\begin{prop}
\[D(EQ)\geq n+1\]
\end{prop}

\subsubsection*{Randomized protocol}
\paragraph{Require}
For all inputs $\mathbb{P}(\mathcal{P}(x,y)\neq f(x,y))\leq \epsilon$.

\begin{defi}
\[R_\epsilon (f) = \min_{\mathcal{P}:\mathbb{P}\{ \mathcal{P}(x,y)\neq f(x_{ij}\} \leq \epsilon } Cost(\mathcal{P})\]
\end{defi}

\begin{prop}
\[R_{1/3}(EQ)=O(\log n)\]
\end{prop}


\end{document}