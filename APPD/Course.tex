\documentclass{article}

\usepackage[utf8]{inputenc}
\usepackage[english]{babel}
 
\usepackage{caption}
%\usepackage{pgfplots}
\usepackage{listings}
\usepackage{graphicx}
\usepackage{footnote}
\usepackage{amsmath}
\usepackage{amsthm}
\usepackage{graphicx}
\usepackage{url}
\usepackage{amssymb}
\usepackage{mathrsfs}
\usepackage{multirow}
\usepackage{amsfonts}
\usepackage[boxed,linesnumbered,noend]{algorithm2e}
\usepackage{qcircuit}
\usepackage{enumerate}

\newtheorem{thm}{Theorem}
\newtheorem{prop}{Propriety}
\newtheorem{lemma}{Lemma}
\newtheorem{defi}{Definition}
\newtheorem{coro}{Corollary}



\setlength{\oddsidemargin}{0pt}
% Marge gauche sur pages impaires
\setlength{\evensidemargin}{0pt}
% Marge gauche sur pages paires
\setlength{\textwidth}{470pt}
% Largeur de la zone de texte 
\setlength{\topmargin}{0pt}
% Pas de marge en haut
\setlength{\headheight}{13pt}
% Haut de page
\setlength{\headsep}{10pt}
% Entre le haut de page et le texte
\setlength{\footskip}{40pt}
% Bas de page + séparation
\setlength{\textheight}{630pt}
% Hauteur de la zone de texte

\title{}
\author{Frederic Vivien\footnote{frederic.vivien@ens-lyon.fr}\\
\small ENS de Lyon}
\date{}

\newcommand{\note}{\medskip\noindent\underline}

\begin{document}

\maketitle
\tableofcontents
\newpage


\note{Bibliography:}
\begin{itemize}
\item \textit{Parallel Algorithms}, H. Casanova, A. Legrand, Y. Robert
\item \textit{Introduction to distributed algorithms}, G. Tel
\item \textit{Scheduling and automatic parallelization}, A. Darte, Y. Robert, F. Vivien
\end{itemize}


\note{Final grade:} 50\% Finale exam + 25\% mid-term exam + 25\% programming project

\part{Theoretical models}
\section{Sorting networks}
\textit{They aim at sorting values.}

Comparators:

\begin{align*}
    \Qcircuit @C=2em @R=0.8em { 
& a &  & \multigate{1}{} & \qw & \min (a,b) \\
& b &  & \ghost{} & \qw & \min (a,b)
}
\end{align*}
%input outputs

\note{Question:} How to arrange comparators to quickly sort a large number of values?


\subsection{Odd-even merging network}
\textit{Odd-even merge sort: Algorithm due to Batcher}\\
If $c_1,c_2,...,c_n$ is an arbitrary sequence, $SORT(c_1,...,c_n)$ denotes the sorted sequence: $c_1\leq c_2 \leq ... \leq c_n$

\note{Warning:} Non-decreasing = "croissant" and increasing = "strictement croissant" (same for positive and non-negative).

\begin{itemize}
\item If $c_1\leq c_2 \leq ... \leq c_n$\\
$SORTED(c_1,...,c_n)$
\item Merged operator:\\
If $SORTED(a_1,...,a_n)$ and $SORTED(b_1,...,b_n)$,\\
then $MERGE((a_1,...,a_n),(b_1,...,b_n))=SORT(a_1,...,a_n,b_1,...,b_n)$
\end{itemize}

Two list of size 1:
\begin{align*}
    \Qcircuit @C=2em @R=0.8em { 
& a &  & \multigate{1}{} & \qw &  \\
& b &  & \ghost{} & \qw & 
}
\end{align*}

Two lists of size 2:

%TODO

$MERGE_2$
\bigskip


Merge 2 lists of size 4:

$MERGE_3$
\bigskip

Merging networks $MERGE_m$ to merge two sorted lists of size 2.

%TODO

1st merging sub-network takes as input the odd elements of the two input lists.

\begin{prop}
Let $A=(a_1,a_2,...,a_{2n})$ and $B=(b_1,b_2,...,b_{2n})$ be two sorted sequences ($SORTED(A)$ and $SORTED(B)$).
\begin{align*}
(d_1, ..., d_{2n} ) & = MERGE((a_1, a_3, a_5, ..., a_{2n-1}), (b_1, b_3, b_5, ..., b_{2n-1})\\
(e_1, ..., e_{2n} ) & = MERGE((a_2, a_4, a_6, ..., a_{2n}), (b_2, b_4, b_6, ..., b_{2n})\\
\end{align*}

Then, we have
\begin{align*}
SORTED((d_1,\min (d_2,e_1),\max (d_2,e_1),..., \min (d_{2n}, e_{2n-1}), \max (d_{2n}, e_{2n-1}), e_{2n}))
\end{align*}
\end{prop}

\begin{proof}
Without loss of generality, we assume all values to be distinct.
\begin{itemize}
\item $d_1$ is the overall minimum, being the minimum of the minimum elements of the two lists.
\item $e_{2n}$ is the overall maximum.
\end{itemize}
\note{General case:} We look at $d_i$ and $e_{i-1}$ (for $2\leq i\leq 2n$) which are, each, either in position $2i-2$ or $2i-1$.

The result is going to be correct if $d_i$ and $e_{i-1}$ dominate $2i-3$ values and are dominated by $4n-2n+1$ values.

We have to prove for $2\leq i \leq 2n$ that 
\begin{enumerate}
\item $d_i$ dominates $2i-3$ values
\item $e_{i-1}$ dominates $2i-3$ values
\item $d_i$ is dominated by $4n+ 2i + 1$ values
\item $e_{i-1}$ is dominated by $4n - 2i + 1$ values
\end{enumerate}

\begin{enumerate}
\item $d_i$ dominates $2i-3$ values $(d_1,d_2,...,d_i)$

We assume (wlog) that $d_i$ belongs to $A$. There are $k$ elements of $A$ in $(d_1,...,d_i)$. $d_i=a_{2k-1}$. $d_i$ dominates $a_1, a_2,...,a_{2k-2}$, that is $2k-2$ elements of $A$.

$d_1,...,d_i$ contains $i-k$ elements of $B$. The largest of these element is $b_{2(2-k)-1}$. $d_i$ dominates $2(2-k)-1$ elements of $B$. Therefore $d_i$ dominates (at least) $(2k-2) + (2 (i-k)-i) = 2i - 3$ elements.
\item Same principal as above
\item Same principal as above
\item $e_{i-1}$ is dominated by $4n - 2i + 1$ elements $(e_1,e_2,...,e_{i-1})$. We assume $e_{i-1}$ belongs to $B$.

Let $k$ be the number of $B$ in $e_1,...,e_i$, $e_{i-1}=b_{2k}$.

Hence, $e_{i-1}$ is dominated by $2n-(2k+1)+1$, that is, $2n-2k$ elements of $B$.

$(e_1,...,e_{i-1})$ includes $(i-1-k)$ elements of A. The largest of these elements is $a_{2(i-1-k)}$.

Therefore $e_{i-1}$ is dominated by $a_{2(i-1-k)+2}=a_{2(i-k}$. 

$e_{i-1}$ is dominated by $a_{2(i-k}$ through $a_{2n}$ and so by $2n-2(i-k+1)$ elements.

Overall $e_{i-1}$ is dominated by $(2n-2k)+(2n-2i+2k+1)=4n-2i+1$ elements
\end{enumerate}
\end{proof}


\note{Performance evaluation:}
\begin{itemize}
\item Unit time to traverse a comparator
\item Execution time of the network: largest number of comparators on a path from an input to an output
\end{itemize}

\begin{lemma}
The processing time $t_m$, and the number of comparators, $p_m$, of $MERGE_m$, satisfy the recursion:
\begin{align*}
t_1=1 & t_m=t_{m-1}+1 & t_m=m\\
p_1=1 & p_m=2p_{m-1} +2^{m-1}-1 & p_m=2^{m-1}(m-1)+1
\end{align*}

Let n be the size of the input lists.
So, $n=2^{m-1}$, $t_n'=o(\log (n))$ $p_n' = o(n\log n)$

Work = $t_n'\times p_n'=O(n\log^2 n)$
\end{lemma}


\subsection{Sorting network}
Recursive construction:

%TODO

\begin{lemma}
The processing time $t_m''$ and the number of comparators $P_m''$ of $SORT_m$ satisfy the recursions:
\begin{align*}
t_1''=1 & t_m''=t_{m-1}+t_m & t_m''=O(m^2)\\
p_1''=1 & p_m''=2p_{m-1}''+p_m & p_m''=O(2^m m^2)
\end{align*}
For inputs of size $n$, the time is $O(\log^2 n)$ and the number of comparators in $O(n\log^2 n)$.

Work is in $O(n \log^4 n)$.

\end{lemma}

\subsection{The 0-1 principle}
\begin{prop}
A network is a sorting network if and only if it is a sorting network for 0-1 sequences.
\end{prop}

\begin{proof}
If a network sorts arbitrary sequences, it sorts 0-1 sequences.

We now show that a network that does not sort arbitrary sequences does not sort 0-1 sequences. we assume that there exists a sequence $(x_1, x_2, ..., x_n)$ that is not sorted correctly.

Let $R$ be the network. The output is $R(x)$. There exists some index $k$ such that $R(x)_k>R(x)_{k+1}$. Let is consider a non decreasing function $f$. Applying the input of the network does not change the paths followed by the different values inside the network (because what matters is the relative position of values in the overall sequence).

\begin{align*}
f(y)=\begin{cases}
0 & $if $ y<R(x)_k \\
1 & $otherwise$\\
\end{cases}
\end{align*}

I fed the sequence $f(x)$ to $R$.
\begin{align*}
R(f(x))_k = f(R(x)_k)=1 \qquad R(f(x))_{k+1}=0\\
\end{align*}
$R$ does not sort all 0-1 sequences.
\end{proof}


\section{Sorting on a one dimensional network}
\subsection{Odd-even transposition sort}

%TODO scheme comparator 
% a   b
% |   |
%|-----|
%|     |
%|-----|
% |   |
%min  max

Version to sort 8 inputs:
%TODO

In total: $n$ rows of comparators for $n$ inputs.

\newpage

\note{Performance:}
\begin{itemize}
\item 1st row: $\frac{n}{2}$ comparators
\item 2nd row: $\frac{n}{2}-1$ comparators
\end{itemize}

A pattern contains (n-1) comparators, and we have $\frac{n}{2}$ patterns.

Overall $\frac{n(n-1)}{2}$ comparators.

\note{Execution time:} $n$

\note{Cost:} $o(n^3)$ \textit{(\#{comparators} $\times$ execution time)}

\begin{proof}[Correction]
0-1 principle.

Correct output if all zeroes on the left.

Let $a_1,...,a_n$ a 0-1 sequence. Let $k$ be the number of 1's in this sequence. et $k_0$ be the initial position of the right-most of these 1's. 
\begin{itemize}
\item During the first step, this 1 moves one position to the right if and only if $k_0$ was odd. After step 1, this 1 is at least at position 2.
\item Then, for each step, starting at step 2, it moves one position to the right, for the $(n-1)$ remaining steps.
\end{itemize}
Eventually, it reaches the $n$-th position.

We now look at the second right-most 1. It moves one position to the right at each step, as soon as the right-most 1 move one position to the right at each step. Ti has one fewer move possible, but has to move to position $(n-1)$.
\end{proof}


\subsection{Odd-even sorting on a one dimensional network}

\begin{itemize}
\item A one dimensional network of \underline{processors}
\item Neighbour processors can exchange values.
\item Mimic the sorting network on the row of processors.
\item At step $2i-1$, processors $P_{2k-1}$ and $P_{2k}$ exchange their data $P_{2k_1}$ will keep the smallest of their data.
\item At step $2i$, processor $2k$ and $2k+1$ exchanges data.
\end{itemize}

%TODO scheme example

We have $n$ values, $p$ processors. $n>p$, and each processor initially holds $\frac{n}{p}$ data.

\note{Complexity:}

Step 0 (local sort): $O(\frac{n}{p}. \log(\frac{n}{p}))$

Step 1: $O(\frac{n}{p})$

There are twice as many steps as there are processors.

\note{Overall complexity:} 
\begin{align*}
O(\frac{n}{p}(\log n) +n )
\end{align*}
If $p\leq \log n$ we have $\frac{\log n}{p}\geq 1$. The complexity become $O(\frac{n}{p}\log n)$.
$\Rightarrow$ if $p\leq \log n$ the algorithm has an optimal running time.

\note{Cost:} 
\begin{align*}
O(n\log n + np)
\end{align*}

\newpage
\setcounter{section}{0}
\part{PRAMs}
Parallel Random Access Machines

\begin{itemize}
\item Theoretical model
\item No communications:\\
$\Rightarrow$ Shared memory model

1 large memory\\
$n$ processing elements/units directly connected to it
\end{itemize}

All processing units execute the same algorithm simultaneously: each time step all PU\footnote{Processing Unit} execute the \emph{same} instruction (but some PUs may be inactive, like in an if ... then ... else ...).


All PU can access memory location in a unit of time. Different processing units may access the very same memory location simultaneously. 

\note{3 variants:}
\begin{itemize}
\item CREW: Concurrent Read Exclusive Write
\begin{itemize}
\item  Any number of PU can simultaneously read any given memory location
\item At any time, at most one PU can write into any given memory location
\end{itemize}
\item EREW: Exclusive Read, Exclusive Write
\begin{itemize}
\item At any given time, at most on PU can access any given memory cell
\end{itemize}

\item CRCW: Concurrent Read, Concurrent Write
\begin{itemize}
\item Consistent mode: all PU writing in the same memory location \textbf{must} write the same value
\item Arbitrary mode: among the different values that PUs attempt to simultaneously write in a given location, one of them is arbitrarily (i.e. randomly) written.
\item Priority mode: the value of the PU of highest priority/index is written
\item Fusion mode: a commutative and associative operation is applied to the values that the PUs write in a same memory location
\end{itemize}
\end{itemize}


\section{Pointer jumping}
\subsection{List ranking}

Linked list $L$ that contains $n$ objects. For any element $i$ in the list, we want to compute
\begin{align*}
d[i]=
\begin{cases}
0 & $if next$ [i]=nil \\
1+d[$next$[i]] & $otherwise$\\
\end{cases}
\end{align*}
Sequential complexity: $O(n)$ (go backward from the tail with a doubly-linked list)

\begin{itemize}
\item Associate one processor to each list element ($P_i$ is associated to element $i$)
\item At each step, we split the list in two sub-lists: one of the odd elements and one of the even ones.\\
$\Rightarrow$ at each step the size of the list is halved\\
$\Rightarrow$ $O(\log n)$ steps
\end{itemize}

%TODO scheme example

\note{Rank computation (L):}\\
\begin{algorithm}[H]
\For{all i in parallel}{
	\eIf{next[i] = nil}{
		d[i]=0
	}{
		d[i]=1
	}
	\While{there exists a node i such that next[i] != nil}{
		\For{all i in parallel do}{
			\If{next[i] != nil}{
				d[i] $\leftarrow$ d[i] + d[next[i]]\\
				next[i] $\leftarrow$ next[next[i]]
			}
		}
	}
}
\end{algorithm}

\note{Evaluation of \textit{while}:}\\
We know the value of $n$.\\
Counter initialized to 0.\\
Each time a next field is set to nil, the counter is incremented, when we reach $n$, we are done.

\begin{algorithm}[H]
\For{all i in parallel}{
	\eIf{next[i] = nil}{
		d[i]=0
	}{
		d[i]=1
	}
	\While{counter $<$ n}{
		\For{all i in parallel do}{
			\If{next[i] != nil}{
				d[i] $\leftarrow$ d[i] + d[next[i]]\\
				next[i] $\leftarrow$ next[next[i]]
			}
			\If{next[i] = nil}{counter $\leftarrow$ counter + 1}
		}
	}
}
\end{algorithm}

Only work with CRCW PRAM in fusion mode (with addition).

\begin{algorithm}[H]
\For{all i in parallel}{
	\eIf{next[i] = nil}{
		d[i]=0
	}{
		d[i]=1
	}
	Finished $\leftarrow$ true\\
	\While{$\neg$ Finished}{
		\For{all i in parallel do}{
			\If{next[i] != nil}{
				d[i] $\leftarrow$ d[i] + d[next[i]]\\
				next[i] $\leftarrow$ next[next[i]]
			}
			\If{next[i] != nil}{Finished $\leftarrow$ false}
		}
	}
}
\end{algorithm}
Work for any variant of CRCW PRAM.

$\rightarrow$ We know that we will have at most $\lceil \log_n \rceil$ steps.
\textit{For i=1 to $\lceil \log_n \rceil$} works also on EREW PRAM.

\newpage
\begin{itemize}
\item About "race condition" in updating next[i]
\end{itemize}
\begin{algorithm}[H]
temp $\leftarrow$ next[next[i]]\\
\CommentSty{(Only works if concurrent reads)}\\
next[i] $\leftarrow$ temp\\
temp$\leftarrow$ d[i]+d[next[i]]\\
d[i]$\leftarrow$temp\\
temp1$\leftarrow$ d[i]\\
temp2$\leftarrow$ d[next[i]]\\
d[i]$\leftarrow$ temp1 + temp2\\
\end{algorithm}

\note{Execution time:} $O(\log n)$

\subsection{Prefix computation}
Sequence $x_1, x_2, ..., x_n$, we want to compute the sequence
\begin{align*}
y_1, y_2, ... ,y_n \text{ where}
\begin{cases}
y_1 = x_1 & \\
y_k=y_{k-1} \otimes x_{k-1} = x_1 \otimes x_2 \otimes ... \otimes x_n & \\
\end{cases}
\end{align*}

Where $\otimes$ binary associative operation.

We assume that $x$ is given as a linked bit.

\begin{algorithm}[H]
\For{all i in parallel}{
y[i]=x[i]\\}
\While{there exists a node i such that next[i]!=nil}{
	\For{all i in parallel}{
		\If{next[i]!=nil}{
			y[next[i]]$\leftarrow$ y[i] $\otimes$ y[next[i]]\\
			next[i] $\leftarrow$ next[next[i]]]\\
		}
	}
}
\caption{Prefix Computation (L)}
\end{algorithm}


\section{Performance evaluation of PRAMs Algorithms}
\subsection{Cost, work, speed-up, efficiency}
Let $P$ be a problem of size $n$.

$T_{seq}(m)$: execution time of the best known sequential algorithm to solve $P$.

Let $T_{par}(p,n)$: execution time of our solution using $p$ processors.

\note{Cost:} $C_p(n)=p\times T_{par}(p,n)$

\note{Work:} Sum on all processing units 
of the number of operations performed by each units (or, the time each PU is effectively used).
\begin{align*}
C_p(n)=W_p(n)+\text{some idle time}
\end{align*}

\note{Speed-up}:
\begin{align*}
S_p(n)=\frac{T_{seq}(p)}{T_{par}(p,n)}
\end{align*}

\note{Efficiency:}
\begin{align*}
e_p(n) & =\frac{S_p(n)}{p}\\
& = \frac{T_{seq}(n)}{pT_{par}(p,n)}\\
& = \frac{T_{seq}(n)}{C_p(n)}
\end{align*}

\note{Another definition of speed-up:} \textit{(We will not use it)}
\begin{align*}
S_p'(n)=\frac{T_{par}(1,n)}{T_{par}(p,n)}
\end{align*}

\note{Ideal}: $S_p(n)$ close to $p$, efficiency close to 1.

\subsection{A simple simulation result}
\begin{prop}
Let $A$ be an algorithm, whose execution time is $t$ when using $p$ PUs, then $A$ can be run on $p'\leq p$ PUs of the same type in time $O(\frac{p}{p'}t)$. The cost on $p'$ PUs is at most twice the cost on $p$ PUs.
\end{prop}

\begin{proof}
The algorithm runs in $t$ steps. At each step we allocate (at most) $\frac{p}{p'}$ operations to each PU for the PUs to process them sequenctially in time $O(\frac{p}{p'})$.
\begin{align*}
C_{p'} & = p'T_{par}(p',n)\\
& = p' t'\\
C_{p'} & \leq p' \left( \left\lceil \frac{p}{p'}\right\rceil t \right) \\
& \leq p'\left( \frac{p}{p'}+1 \right) t\\
& = pt + p't\\
& \leq 2pt=2C_p
\end{align*}
\end{proof}

\note{Remark:} This proposition give an upper bound of what s possible, but it is sometimes possible to do better.

Let us consider prefix computation: $n$ values, $n$ PUs, $O(\log_2 (n))$.

\textit{Using the proposition:}
\begin{align*}
\text{With }p\leq n \text{PUs, } O\left(\frac{n}{p}\log(n)\right)
\end{align*}

\textit{One other solution:}
Each processor has $\frac{n}{p}$ values (consecutive values)
\begin{itemize}
\item Each PU computes the prefixes for its $\frac{n}{p}$ values in time $\frac{n}{p}$.
\item Pointer-jumping propagation over the $p$ PUs of $p$ prefixes in time $O(\log p)$.
\item Each PU applies the prefixes computed above to its $\frac{n}{p}$ values in $O(\frac{n}{p})$.
\end{itemize}

The overall execution time is $O\left(\frac{n}{p} + \log p\right)$

``Coarsening of the computation.''

\note{Efficiency:} Usually, when $p$ increases the efficiency decreases
\begin{align*}
0 &\leq e_p(n) \leq 1\\
0 &\leq S_p(n) \leq p
\end{align*}

\note{Super linear speed-up:}
\begin{align*}
s_p(n) & > p
\end{align*}

It can happen in practice if, for instance, by using more processors, all data fit in memory and the program no longer swaps.


\subsection{Brent's theorem}

\begin{thm}[Brent's theorem]
Let $A$ be an algorithm that executes a total of $m$
 operations and that runs in time $t$ on a PRAM (on an unspecified number of PUs).

Then $A$ can be simulated of time $O\left( \frac{m}{p}+t\right)$ on $p$ PUs of a PRAM of the same time.
\end{thm}

\begin{proof}
$A$ runs in $t$ steps, performing $m_i$ operations at step $i$.
\begin{align*}
\sum_{i=1}^t m_i & = m
\end{align*}

We simulate step $i$ in time $\left\lceil \frac{m_i}{p}\right\rceil$ using $p$ processors.

The overall execution time is thus 
\begin{align*}
\sum_{i=1}^t \left\lceil \frac{m_i}{p}\right\rceil & \leq \sum_{i=1}^t \left(\frac{m_i}{p}+1\right)\\
& = \left( \sum_{i=1}^t \frac{m_i}{p}\right) + \left( \sum_{i=1}^t 1\right)\\
& = \frac{m}{p} + t
\end{align*}
\end{proof}

\note{Example:}
Computing the maximum of $n$ values on a EREW PRAM. We do that tournament-like with a binary tree of comparisons.

Time: $t=O(\log n)$ ($n$ PUs), $m=O(n)$

On $p$ PUs, using Brent's theorem, we know we can compute the maximum in time $O\left(\frac{n}{p} + \log n \right)$.

$p=\frac{n}{\log n} \Rightarrow \text{time } O(\log n)$

\section{Comparison of PRAM models}
\subsection{Model separation}
\textit{Is there a problem that can be solved significantly faster on a CRCW than on a CREW PRAM?}


Choosing the fusion mode and computing a sum of $n$ values.

\note{CRCW:} all PUs write to the same memory location: $O(1)$

\note{CREW:} at each time step each PU can cpmbine at most 2 values: $O(\log n)$ time steps.

\bigskip
\textit{Using CRCW PRAM in coherent mode?}

Computing the maximum of $n$ values: 


\begin{algorithm}[H]
\For{all $i$ from 1 to $n$, in parallel}{
	ismax[$i$] $\leftarrow$ true
	\CommentSty{(initialization)}
}
\For{all $i,j,1\leq i \leq n, 1\leq j \leq n, i\neq j$ in parallel}{
	\If{value[$i$]$<$value[$j$]}{
		ismax[$i$]$\leftarrow$false
	}
}
\For{all $i, 1\leq i \leq n$ in parallel}{
	\If{$ismax[i]=true$}{
		result$\leftarrow value[i]$
	}
}
\end{algorithm}
Time: $O(1)$ on $n$ processors

On a CREW PRAM, at best a $\log n$ execution time



\bigskip\textit{Separation between CREW and EREW models ?}

Does the element $e$ belongs to a set $X$ of $n$ distinct values. On a CREW PRAM, each of $n$ PUs compares its values of $X$ to $e$. If equality, a boolean is sett to true $\Rightarrow$ constant time.

On a EREW PRAM, it takes at least a time $\log n$ for al processors to have a copy of $e$ (at best we double at each step the number of copies of $e$).

\subsection{Simulation theorem}

\begin{thm}[Simulation theorem]
Any CRCW algorithm with $p$ PUs has an execution time at most $O(\log n)$ lower than the \emph{best} EREW algorithm to solve the same problem using $p$ processors.
\end{thm}

\begin{proof}
We assume the CRCW PRAM to be i coherent mode (different PUs writing simultaneously in the same memory cell must write the same value).

We show how to execute one step of concurrent write in at most $\log p$ steps on a EREW PRAM.

%TODO scheme

\begin{enumerate}
\item We take a temporary array $A$. Each processor $P_I$ writes in $A[i]$ the couple (location, value).
\item The array $A$ is sorted by non-decreasing value of the 1st element of the couples
\item Each processor $P_i$ compares the address in $A[i]$ to the address held by the previous processor. If the addresses differ (or if $i=1$), it performs a write, otherwise, it does nothing.
\end{enumerate}

\note{Complexity:} (1) and (3) are executed in constant time on $O(p)$ PUs. (2) We assume that there exists am EREW algorithm\footnote{Cf part \ref{sort_log}} sorting $p$ values on $p$ PUs in time $O(\log p)$ $\Rightarrow$ time $O(\log p)$
\end{proof}

\section{Cole's sorting machine}
\label{sort_log}
\textit{Proposed in 1986.}

CREW algorithm to sort $n$ values in time $\log n$ using $n$ PUs, based on a merge sort algorithm:
We have $\log n$ levels in the tree, and $\log n$ steps in Cole's algorithm.
We want an algorithm running in time $O(\log n)$. The algorithm should thus be able to merge two sorted lists on any size in constant time.


\subsection{Merge}
\begin{defi}[Good sampler]
A sorted sequence $J$ is said to be a good sampler (GS) of a sequence $L$ if, for any $k \leq 1$, there are at most $2k+1$ elements of $J$ \emph{between} $k+1$ (arbitrary) consecutive elements of $L \cup \{-\infty \} \cup \{+\infty\}$.

\begin{itemize}
\item If $J$ and $K$ are two sorted sequences, $J|K$ is the merge of $J$ and $K$.
\item If $a$ and $b$ are two values with $a<b$, we say that $x$ is between $a$ and $b$ if and only if $a<x\leq b$
\item $rank(x,J)=card \{j\in J \;|\; j<x \}$\\
The cross-ranked of sorted sequence $A$ in the sorted sequence $B$ is:
\begin{align*}
R[A,B]: \; & A\to \mathbb{N}\\
& x\mapsto rank(x,B)
\end{align*}
\end{itemize}
\end{defi}
In practice, there are at most 3 elements of the sequence $J$ in between two consecutive elements of one of its good sampler (extended with $-\infty$ and $+\infty$).

\note{Example:} The subset of elements of odd ranks (or even ranks) is a good sampler.


Two sorted sequences, $J$ and $K$, 
\begin{align*}
J & =[2,3,7,8,10,14,15,17,18,21]\\
K & = [1,4,6,9,11,12,13,16,19,20]\\
L & = [5,10,12,17] \text{ is a good sampler of \emph{both} $J$ and $K$}
\end{align*}
In theory we should check that $L$ is a good sampler. $k=1$, we consider the intervals $]-\infty ,5], ]5,10[, ]10,12],$ $]12,17]$ and $]17,+\infty [$ and we show that there are at most $2k+1=3$ values of $J$, and $2k+1=3$ values of $K$ in each of these intervals.
We do know for $k=2, k=3,...$

We assume we know the crossrank of $J$ and $K$ in $L$ ($R[J,L]$ and $R[K,L]$).

\begin{align*}
J(1) = (2,3) & \qquad K(1)=(1,4) \qquad\Rightarrow (1,2,3,4) \\
J(2)=(7,8,10) & \qquad K(2)=(6,9) \qquad\Rightarrow (6,7,8,9,10)\\
J(3)=() & \qquad K(3)=(11,12) \qquad\Rightarrow (11,12)\\
J(4)=(14,15,17) & \qquad K(4)=(13,16) \qquad\Rightarrow (13,14,15,16,17)\\
J(5)=(18,21) & \qquad K(5)=(19,20) \qquad\Rightarrow (18,19,20,21)\\
\end{align*}

\begin{algorithm}
Merge with Help($J,K,L$):\\
$J$ and $K$ are partitioned using $L$ in $L+1$ subsets.\\
$J(i)=\{ j\in J \;|\; l_{i-1}<j<l_i\} \quad (l_0=-\infty)$\\
\For{all i in parallel $(1\leq i \leq |L|+1)$}{
res $\leftarrow MERGE(J(i),K(i))$}
$J|K \leftarrow res_1.res_2. \;\hdots\; . res_{|L|+1}$
\end{algorithm}

\begin{lemma}
If $L$ is a good sampler of both $J$ and $K$, and if crossranked $R[L,J], R[L,K],R[J,L]$ and $R[K,L]$ are known, then Merge with Help runs in $O(1)$ time with $|J|+|K|$ PUs on a CREW PRAM.
\end{lemma}

\begin{proof}
(b) By definition of a good sampler $J(i)$ and $K(i)$ contains at most 3 values each and can be merged sequentially in $O(1)$.

%TODO make it as algortihm
(a) Each $P_m$ reads the rank $r=rank(j_m,L)$.

\begin{algorithm}[H]
\eIf{$m=1$ or ($rank(j_m,L)\neq rank(j_{m-1},L)$)}{
	Add $j_m$ to $J(r)$ \CommentSty{//$J_{m-1}$ put $r-1$ and $r-2$}
}{
	\eIf{$rank(j_m,L)\neq rank(j_{m-2},L))$}{
		Add $j_m$ to $J(r)$ \CommentSty{//$J_m$ and $J_{m-1}$ should be put $r-1$ and $r-2$}
	}{
		Add $j_m$ to $J(r)$ \CommentSty{//$J_{m-1}$ put $r-1$ and $r-2$}
	}
}
\end{algorithm}
We use $|J|+|K|$ PUs.

(c) We need to know where to write the smallest element of $res_i=J(i)|K(i)$\\
$rank(l_i, J|K)=rank(l_i,J)+rank(l_i,K)$. we write sequentially (at most using 6 steps) $res_i$ starting at position $rank(L_{i-1},J|K)$. We use $|L|+1$ PUs.
\end{proof}

\subsection{Sorting trees}
We assume $n=2^m$.\\
$\rightarrow$ binary tree used in a pipelined way\\
$\rightarrow$ the value of a node at step $t$ will be a good sampler of the value at step $t+1$.

\begin{algorithm}[H]
Cole-Merge():\\
Receive $X(t+1)$ from its left child\\
Receive $Y(t+1)$ from its right child\\
Merge: $val(t+1)\leftarrow$ Merge with Help ($X(t+1), Y(t+1), val(t)$)\\
Send: $Z(t+1)=REDUCE(val(t+1))$ to the parent\\
$REDUCE$: keeps only every forth value.
\end{algorithm}

A node of the tree is said to be complete when it has received all its inputs, i.e., a sorted sequence of size $2^k$ for a node at level $k$. As soon as the input sequence is no longer empty it doubles at each step, from $1$ to $2^{k-1}$

If a node is complete at step $t$, then the $REDUCE$ operation changes:
\begin{itemize}
\item at step $t+1$: once again sends every forth element
\item at step $t+2$: sends every other elements (even ranked elements) %CAREFULL ENGLISH !!! = 1 sur 2 !!!
\item at step $t+3$: sends every element
\item From step $t+4$ on, stops working, stop reading sequences.
\end{itemize}


\note{Example:}
\begin{itemize}
\item $t=0$: leaves are complete
\item $t=1$: every fourth element out the leaves
\item $t=2$: every other elements out of the leaves
\item $t=3$: each leaf sends its value.\\
all the nodes at level 1 complete
\item $t=4$
\item $t=5$: level 1 nodes send every other value
\item $t=6$: all level 2 nodes are complete
\item $t=7$: the root merges $[8]$ and $[4]$
\item $t=8$: the root merges $[6,8]$ and $[2,4]$ using $[4,8]$ as a good sampler.
\item $t=9$: the root merges $[5,6,7,8]$ and $[1,2,3,4]$ using $[2,4,6,8]$ as a good sampler.
\end{itemize}
The root is complete, we are done.
%TODO : scheme of the exemple

\note{Execution time:} Nodes at level $k$ are complete at time $3k$. Execution time is $O(3 \log n)= O(\log n)$

\note{Number of processors:} To merge 2 lists of size $k$, in constant time, we need $2k$ PUs. (We have $\log n$ levels, and each requires at most $n$ PUs, so we know how to do it in $O(n \log n)$).

Level $k$:\\
There are $\frac{n}{2^k}$ nodes at level $k$.
\begin{itemize}
\item They are complete at time $3k$. They each have a value $val(3k)$ of size $2^k$. Overall, they use $\frac{n}{2^k}\times 2^k=n$ PUs.
\item $3k-1$, total size of input $2^{k-1}$, need $\frac{n}{2}$
\item $3k-2$, $\frac{n}{4}$
\item ...
\end{itemize}
Level $k$ needs $\frac{n}{8}$ PUs at step $3k-3$.\\
Level $k+1$ needs $\frac{n}{8}$ PUs at step $3k$

\note{At step 3k:}
\begin{itemize}
\item level $k+2=\frac{n}{64}$ PUs
\item level $k+1=\frac{n}{8}$ PUs
\item level $k=n$ PUs
\item level $k-1=0$ PUs
\item ...
\item level $0=0$ PUs
\end{itemize}

Overall: Cole's algorithm needs less than $2n$  PUs.


\subsection{Corrections}
\begin{lemma}
Les $X,X',Y$ and $Y'$ be four sorted sequences. If $X$ is a GS of $X'$ and if $Y$ is a GS of $Y'$, then $REDUCE(X|Y)$ is a GS of $REDUCED(X'|Y')$
\end{lemma}

\note{Warning:} $X|Y$ is not \emph{always} a GS of $X'|Y'$. If we take $X=[2,7] \qquad X'=[2,5,6,7]$, and $Y=[1,8] \qquad Y'=[1,3,4,8]$; then between the consecutive elements 2 and 7 of $X|Y$ we have $\{3,4,5,6,7\}$ in $X'|Y'$, that is 5 values ($>3$).

\begin{proof}

\begin{prop}
There are at most $2r+2$ elements of $X'|Y'$ in between $r$ consecutive elements of the $X|Y$.
\end{prop}


\begin{proof}
Let $e_1, e_2, ..., e_r$ be a sequence of $r$ consecutive elements of $X|Y$. There are $h_X$ elements of $X$ in it and $h_Y$ elements of $Y$; $h_X+h_Y=r$.
Without loss of generality, we assume $e_1\in X$.

\note{Case 1:} $e_r\in X$.
$X$ is a GS of $X'$. In between $e_1$ and $e_r$ we have at most $2(h_X-1)+1=2h_X-1$ elements of $X'$.
Because $Y$ is a GS of $Y'$, there are at most $2(h_y+1)+1$ elements of $Y'$ between $h_y+2$ elements of $Y$. Overall there are at most $(2h_X-1)+(2(h_Y+1)+1)=2h_X+2h_Y+2=2r+2$ elements of $X'|Y'$ in between the $r$ consecutive elements $e_1,...,e_r$.


\note{Case 2:} $e_r\in Y$.
We add an element $e_0\in Y$ preceding $e_1$, and an element $e_{r+1}\in X$ greater than $e_r$. The elements of $X'$ that are between $e_1$ and $e_r$ are between $e_1$ and $e_{r+1}$, so between $h_X+1$ elements of $X$. Because $X$ is a GS of $X'$, there are at most $2h_x+1$ such elements. Symmetrically, there are at most $2h_y + 1$ elements of $Y'$ between $e_1,...,e_r$. Hence, overall at most $(2h_X+1)+(2h_Y+1)=2r+2$ elements.
\end{proof}


Let $Z=REDUCE(X|Y)$, $Z'=REDUCE(X'|Y')$. Let us consider $k+1$ consecutive values of $Z$: $z_h, z_{h+1},...,z_{h+k}$.

By definition of $REDUCE$, $z_h=e_{4h}, z_{h+1}=e_{4h+4},...,z_{n+k}=e_{4n+4k}$ where $X|Y=e_1,e_2,...$.
There are $4k+1$ elements of $X|Y$ between $z_h$ and $z_{h+k}$. We take $r=4k+1$. we know that there are at most $2r+2=8k+4$ values of $X'|Y'$ in between $z_h$ and $z_{h+k}$. The $REDUCE$ operator keeps every forth value therefore there are at most $\frac{8k+4}{4}=2k+1$ values of $REDUCE(X'|Y')$ in between $z_h,...,z_{h+k}$, that is, $k+1$ (arbitrary) consecutive values of $REDUCE(X|Y)$
\end{proof}


\begin{lemma}
If we have the sorted sequences $X,Y,U=X|Y,X'$ and $Y'$, with $X$ a GS of $X'$, $Y$ a GS of $Y'$, and we know the crossranks $R[X',X]$ and $R[Y',Y]$. Then, we can compute the cross $R[X',U],R[Y',U],R[U,X'],$ $R[U,Y']$ in $O(1)$ time and using $O(|X|+|Y|)$ PUs.
\end{lemma}


\newpage
\setcounter{section}{0}
\part{Algorithm for rings and grid of processors}
\section{Algorithm for rings of processors}

Distributed memory model

Each processor owns some private memory and is the only processor allowed to access it (both for reading and writing).

\paragraph{Considered topology:}
Unidirectional ring of processors. We have $p$ processors.

%scheme

One direct communication link from $P_i$ to $P_{i+1(mod p)}$ for $0\leq i \leq p-1$ (most of the time the ``mod $p$" expression will be implicit.\\
$NOM\_PROCS()$: gives the number of processors in the ring.\\
$MY\_NUM()$: identifier/rank of the calling processors.

Communication links are unidirectional: to send a message to $P_0$, $P_1$ must go through $P_2,P_3,...,P_{p-1}$.

\paragraph{Primitives:}
\begin{itemize}
\item For sending
\[SEND(\underbrace{addr}_{\substack{\text{address where}\\ \text{to read the data}\\ \text{to send}}}, \underbrace{m}_{\substack{\text{the size of} \\ \text{the message}}})\]

Point-to-point communications: we do not need to specify the destination because processor $P_i$ can only send a message to processor $P_{i+1}$.

\item For receiving message:
\[RECIEVE(\underbrace{addr}_{\substack{\text{address where to} \\ \text{store what is}\\ \text{received}}}, \underbrace{m}_{\substack{\text{the size of}\\ \text{the message}}}) \]
\end{itemize}


\begin{itemize}
\item $SEND$ and $RECIEVED$ works in pairs
\item \emph{Two kinds of primitives :}
\begin{itemize}
\item blocking primitives: when reaching a communication primitive, the algorithm stops and only resume its execution when the communication is completed.
\item asynchronous or non-bocking communications: the algorithm instantaneously returns from the communication itself will take place at some later time (we have no idea when). One may \emph{test} whether  the communication has completed. Enabled an overlap of communication and computation.
\end{itemize}
\end{itemize}

Most of the time we will use asynchronous sends and blocking receives.

\paragraph{Program Multiple Data (SPMD) model:} (no longer the synchronization we had with PRAM) Different processors can execute different instructions (of the same program) at the same instant.

\paragraph{Cost (time) of communications:} Sending a message of size $m$ from a processor to its neighbour takes a time
\[\underbrace{L}_{\text{latency}}+m.\underbrace{b}_{\substack{\text{the inverse}\\ \text{of the bandwidth}}}\]

Each processor can simultaneously:
\begin{itemize}
\item send a message
\item received a message
\item perform some computation
\end{itemize}

\subsection{Macro-communication}
\subsubsection{Broadcast}
A given processor $k$ wants to send a message of size $m$ to all other processors.
Addresses :
\begin{itemize}
\item For $P_k$: where the message is initially stored
\item For $P_i (i\neq k)$: where to store the message
\end{itemize}

\begin{algorithm}[H]
BROADCAST(k, addr, m)\\
p$\leftarrow$NUM\_PROCS()\\
q$\leftarrow$MY\_NUM()\\
\eIf{$q=k$}{
	SEND(addr, m)
	}{
	\eIf{$q=k-1\mod p$}{
	RECEIVE(addr, m)
	}{
	RECEIVE(addr, m)\\
	SEND(addr, m)
	}
}
\end{algorithm}

The algorithm is automatically correct, and has same running time with blocking receives and asynchronous sends.

\subsubsection{Scatter}

Processor $P_k$ sends a different message to every other processor. Initially, processor $P_k$ holds a message for processor $P_q$ at address $addr[q]$ (we may assume that $addr[k]$ is holding a message for $P_k$)


\begin{algorithm}[H]
SCATTER(k, msg, addr, m)\\
p$\leftarrow$NUM\_PROCS()\\
q$\leftarrow$MY\_NUM()\\
\eIf{$q=k$}{
	\For{i=1 to p-1}{
		SEND( addr$[k-1-i+1\mod p]$, m)
	}
}{
	\For{j=1 to p-(q-k)-1}{
		RECEIVE(tempR, m)\\
		SEND(tempR, m)
	}
	RECEIVE(msg, m)
}
\end{algorithm}

\note{Execution time:} $(p-1)(L+m.b)$

\begin{algorithm}[H]
RECEIVE(tempR, m)\\
\For{i=1 to p-q+k-1}{
	SEND(tempR, m)\\
	RECEIVE(tempS, m)\\
	tempR $\leftrightarrow$ tempS
}
\end{algorithm}
% ?
If messages are in order $Addr[k+1],Addr[k+2],...$\\
$[(p-2)+(p-1)](L+mb)$\\
$(2p-3)(L+mb)$

\subsubsection{All to all}
$p$ simultaneous broadcast

\begin{algorithm}[H]
ALL-TO-ALL(my\_message, addr, m)\\
p$\leftarrow$NUM\_PROSC()\\
q$\leftarrow$MY\_NUM()\\
addr$[q]\leftarrow$my\_message\\
\For{i=1 to p-1}{
	SEND(addr$[q-i+1\mod p],m$)\\
	RECEIVE(addr$[q-i\mod p],m)$
}
\end{algorithm}

\note{Execution time:}$(p-1)(L+m.b)$

\subsubsection{Pipelined broadcast}
Split the message of size $m$ into $r$ pieces (we assume that $r$ divides $m$).

\begin{algorithm}[H]
PIPELINED\_BROADCAST(k,addr,m)\\
p$\leftarrow$NUM\_PROSC()\\
q$\leftarrow$MY\_NUM()\\
\eIf{q=k}{
	\For{i=0 to r-1}{
		SEND(addr$[i]$,m/r)\\
	}
}{
	\eIf{q=k+1}{
		\For{i=0 to r-1}{
			RECEIVE(addr$[i]$,m/r)
		}
	}{
		RECEIVE(addr$[0]$,m/r)\\
		\For{i=0 to r-2}{
			SEND(addr$[i]$,m/r)\\
			RECEIVE(addr$[i+1]$,m/r)
		}
		SEND(addr$[r-1]$,m/r)
	}
}
\end{algorithm}


\note{Execution time:}
\[(p-1)(l+\frac{m}{r}.b)+(r-1)(L+\frac{m}{r}b)=(p+r-2)(L+\frac{m}{r}b)\]
The value of $r$ minimizing the execution time is: 
\[r=\sqrt{\frac{m(p-2)b}{L}}\]

The execution time becomes:
\[(\sqrt{(p-2)L}+\sqrt{mb})^2 \underset{m\to +\infty}{\sim} mb \]


\subsection{Matrix vector multiplication}

Let $A$ be a matrix of size $n\times n$, $x$ a vector of size $n$.

The aim is to calculate $y=A.x$ (all indices starting at 0)

Sequential version:

\begin{algorithm}
\For{i=0 to n-1}{
	$y_i\leftarrow 0$
	\For{j=0 to n-1}{
		$y_i \leftarrow y_i +A_{ij} \times x_j$
	}
}
\end{algorithm}

The computation of the $n$ values of vector can be computed in parallel. We assume $p<n$, and $p$ divides $n$ ; let $r=\frac{n}{p}$.

We charge each processor to compute $r$ entries of vector $y$. We assume there is not enough memory to replicate matrix $A$ on each processor. $A$ is distributed among the $p$ processors.

\paragraph{Classical solution:} each processor is given a set of $r$ consecutive rows.\\
Processor $P_0$ holds row 0 to $r-1$,\\
Processor $P_1$ holds row $r$ to $2r-1$,\\
...

This is called a \emph{block} distribution of rows.

We could assume that we have enough memory available to have one copy on each processor.

In such a case, the whole computation could be performed without communication. Processor $P_i$ would hold rows $i \times r$ through $(i+1)r-1$ of A and $x$ and thus compute $y_{ir}$ through $y_{(i+1)r-1}$. At the end of the algorithm, $y$ would be distributed. Therefore, if we wanted to apply to apply a matrix-vector multiplication to $y$, we could not reuse our algorithm. Because, most of the time, matrix operations happen not alone but in a series of operations, we always assume that the inputs and outputs of algorithms are distributed the same way.

We assume that $x$ (and later $y$) is distributed by blocks of $r$ values. Processor $P_i$ holds the component $x_{ir}$ to $x_{(i+1)r-1}$ and will compute the component $y_{ir}$ to $y_{(i+1)r-1}$.

\begin{center}
\begin{tabular}{ c c c c c c c c c c c }
 \multirow{2}{*}{$P_0$} 
& $A_{00}$ & $A_{01}$ & $A_{02}$ & $A_{03}$ & $A_{04}$ & $A_{05}$ & $A_{06}$ & $A_{07}$ & & $x_0$ \\
&  $A_{10}$ & $A_{11}$ & $A_{12}$ & ... & & & & $\vdots$ & & $x_1$\\
\hline
 \multirow{2}{*}{$P_1$} 
& $A_{20}$ & $A_{21}$ & $A_{22}$ & ... & & & & $\vdots$ & & $x_2$\\
&  $A_{30}$ & $A_{31}$ & $A_{32}$ & ... & & & & $\vdots$ & & $x_3$\\
\hline
 \multirow{2}{*}{$P_2$} 
& $A_{40}$ & $A_{41}$ & $A_{42}$ & ... & & & & $\vdots$ &  & $x_4$\\
&  $A_{50}$ & $A_{51}$ & $A_{52}$ & ... & & & & $\vdots$ & & $x_5$\\
\hline
 \multirow{2}{*}{$P_3$} 
& $A_{60}$ & $A_{61}$ & $A_{62}$ & ... & & & & $\vdots$ & & $x_6$\\
&  $A_{70}$ & $A_{71}$ & $A_{72}$ & ... & ... & ... & ... & $A_{77}$ & & $x_7$
\end{tabular}

%There misses a colorful tabular illustrationg each step
\end{center}

\begin{algorithm}
Matrix-vector Multiplication($A$, x , y)\\
$p\leftarrow NUM\_PROCS()$\\
$q\leftarrow MY\_NUM()$\\
$tempS\leftarrow x$\\
\For{step=0 to p-1}{
	$SEND(tempS,r)$\\
	$RECEIVE(tempR,r)$\\
	\For{i=0 to r-1}{
		\For{j=0 to r-1}{
			$y[i]\leftarrow y[i] + A[i,(q-step \mod p)r + j]tempS[j]$		
		}
	}
	$tempR\leftrightarrow tempS$
}
\end{algorithm}

\paragraph{Execution time:}
\begin{align*}
 & p\times \max (L+rb, L+rb, \underbrace{r^2w}_{\substack{\text{cost of a}\\ \text{multiply-add}}}) \\
 & = p \max (L+rb,r^2w)\\
 r & =\frac{n}{p}\\
 \text{Execution time} & = \max{pL+nb, \frac{n^2}{p}w}\\
 & \underset{n\to \infty}{\sim} \frac{n^2}{p}w\\
\end{align*}
Asymptotically, our algorithm is efficient.


\subsection{Matrix-Matrix Multiplication}
3 matrix $A, B, C$ square $n\times n$ matrices.
\begin{algorithm}
\For{i=0 to n-1}{\For{j=0 to n-1}{\For{k=0 to n-1}{$C_{i,j}\leftarrow C_{i,j} + A_{ik}B_{kj}$}}}
\end{algorithm}

This is nothing but
\begin{itemize}
\item $n^q$ scalar products
\item $n$ matrix-vector multiplication
\end{itemize}

We assume all matrix to be distributed the same way. Processor $P_{i}$ holds rows $i\times r$ to $(i+1)r-1$ of matrices $A,B$ and $C$.

We logically divide the $r$ rows of $A$ assigned to processor $P_i$ in $p$ blocks of size $r$, this set of row is seen as $p$ $r\times r$ matrices.

$P_i$ holds the block $\hat{A}_{i,l}, \hat{B}_{i,l}$ and $\hat{C}_{i,l}$ of element of $A, B$ and $C$.

\[
P_i \quad
\begin{array}{|c c c|}
\hline
A_{ir,0} & \dots & A_{ir,r-1}\\
A_{ir+1,0} & \vdots & \vdots\\
\vdots & \vdots & \vdots\\
A_{(i+1)r-1,0} & \hdots & \vdots\\
\hline
\end{array}
\qquad
\begin{array}{|c c c|}
\hline
. &\hdots&A_{ir,(p-1)r}\\
\vdots& & A_{ir+1,(p-1)r}\\
\vdots& & \vdots\\
\vdots&\hdots& A_{(i+1)r-1,(p-1)r}\\
\hline
\end{array}
\]

\paragraph{Step 0:}
\begin{align*}
P_q: \qquad& \hat{A}_{q,l},\hat{B}_{q,l}, \hat{C}_{q,l}\\
& \hat{C}_{q,l}\leftarrow \hat{A}_{q,q}.\hat{B}_{q,l}\\
& \text{Sending } \hat{B}_{q,l} \text{ to } P_{q+1}\\
& \text{Recieving } \hat{B}_{q-1,l} \text{ from } P_{q-1}\\
\end{align*}

\paragraph{Step 1:}
\[
P_q: \qquad \hat{C}_{q,l} \underset{\text{for } 0\leq l \leq p-1}{\leftarrow} \hat{C}_{q,l}+\hat{A}_{q,q-1}\times \hat{B}_{q-1,l}
\]


\begin{algorithm}
Matrix Matrix Multiplication (A, B, C)\\
$p\leftarrow NUM\_PROCS()$\\
$q\leftarrow MY\_NUM()$\\
$tempS\leftarrow B$\\
\For{step=0 to p-1}{
	$SEND(tempS, r \times n)$\\
	$RECEIVE(tempR, r \times n)$\\
	\For{l=0 to p-1}{
		\For{i=0 to r-1}{
			\For{j=0 to r-1}{
				\For{k=0 to r-1}{
					$C[i, lr +j]\leftarrow C[i,lr+j]+A[i,((q-step)\mod p)\times r + k] \times tempS[k,lr+j]$
				}			
			}		
		}	
	}
	$tempR\leftrightarrow tempS$
}
\end{algorithm}

\paragraph{Execution time:} ($r=\frac{n}{p}$)

\begin{align*}
p \times \max(L+rnb, pr^3 w) & = \max (pL+n^2 b,\frac{n^3 w}{p})\\
& \underset{n\to \infty}{\sim}\frac{n^3 w}{p}
\end{align*}

Asymptotically an efficient algorithm

Complexity of using $n$ times the matrix-vector product:
\[n \times \max(pL+nb,\frac{n^2w}{p}) = \max(npL+n^2b,\frac{n^3w}{p})\]
Benefit of using matrix-matrix multiplication: the number of communications, and thus the cost of latencies, is divided by $n$.

\subsection{Stencil computation}

$A$ a 2 dimensional array of data. % scheme
The data is repeatedly updated.

A cell is updated using a function that takes as input the value (past and/or new) of some of its neighbouring cell.

\begin{center}
\begin{tabular}{c c c}
NW & N & NE\\
W & C & E\\
SN & S & SE
\end{tabular}

neighbourhood
\end{center}

We take an array of size $n\times n$.

Our stencil:
\[C_{new} \leftarrow UPDATE(C_{old},W_{new},N_{new})\]

If the cell has no west neighbour, $W_{new}$ is replaced by $NIL$; if the cell has no north neighbour, $N_{new}$ is replaced by $NIL$.

\begin{algorithm}[H]
\For{i=0 to n-1}{
	\For{j=0 to n-1}{
		$a_{ij}\leftarrow UPDATE(a_{i,j},a_{i,j-1}, a_{i-1,j})$
	}
}
\end{algorithm}

\paragraph{Greedy Parallelization:} We assume that $n=p$. Each processor holds one row.

\note{Idea:} do things as soon as possible.


\begin{algorithm}[H]
$p\leftarrow NUM\_PROCS()$\\
$q\leftarrow MY\_PROC()$\\
\eIf{q=0}{
	$A[0]\leftarrow UPDATE(A[0], NIL, NIL)$\\
	$SEND(A[0],1)$\\
	}{
	$RECEIVE(v,1)$\\
	$A[0]\leftarrow UPDATE(A[0],NIL,v)$
}

\tcp{----- To correct number of SEND/RECEIVED}
\If{$0\neq p-1$}{
	$SEND(A[0],1)$
}
\If{$q\neq p-1$}{
	$SEND(A[0],1)$
}	
\tcp{-----}

\For{j=1 to n-1}{
	\eIf{q=0}{
		$A[j]\leftarrow UPDATE(A[j],A[j-1],NIL)$\\
		$SEND(A[j],1)$
	}{ \eIf{q=p-1}{
			$RECEIVE(v,1)$\\
			$A[j]\leftarrow UPDATE(A[j],A[j-1],v)$
		}{
			$SEND(A[j-1],1)$\\
			$RECEIVE(v,1)$\\
			$A[j]\leftarrow UPDATE(A[j],A[j-1],v)$
		}
	}
}

\end{algorithm}


\paragraph{General case $p<n$:} $p$ divides $n$. How to distribute rows to processors?

\paragraph{First solution:} cycle distribution of rows to processors. $p=3, n=9$ rows.

\begin{center}
\begin{tabular}{c c | c |}
\cline{3-3}
$p_0$ & 0 & $\qquad \qquad \qquad \qquad \qquad \qquad $\\
\cline{3-3}
$p_1$ & 1 & $\qquad \qquad \qquad \qquad \qquad \qquad $\\
\cline{3-3}
$p_2$ & 2 & $\qquad \qquad \qquad \qquad \qquad \qquad $\\
\cline{3-3}
$p_0$ & 3 & $\qquad \qquad \qquad \qquad \qquad \qquad $\\
\cline{3-3}
$p_1$ & 4 & $\qquad \qquad \qquad \qquad \qquad \qquad $\\
\cline{3-3}
$p_2$ & 5 & $\qquad \qquad \qquad \qquad \qquad \qquad $\\
\cline{3-3}
$p_0$ & 6 & $\qquad \qquad \qquad \qquad \qquad \qquad $\\
\cline{3-3}
$p_1$ & 7 & $\qquad \qquad \qquad \qquad \qquad \qquad $\\
\cline{3-3}
$p_2$ & 8 & $\qquad \qquad \qquad \qquad \qquad \qquad $\\
\cline{3-3}
\end{tabular}
\end{center}

Row $i$ is allocated to processor $P_{i\mod p}$.

\begin{algorithm}[H]
StencilWithCyclicDistribution(A)\\
$p\leftarrow NUM\_PROCS()$\\
$q\leftarrow MY\_PROC()$\\
\For{i=0 to $\frac{n}{p}-1$}{
	\eIf{q=0 and i=0}{
		$A[0,0]\leftarrow UPDATE (A[0,0],NIL,NIL)$\\
		$SEND(A[0,0],1)$
	}{
		$RECEIVE(v,1)$\\
		$A[i,0]\leftarrow UPDATE(A[i,0],NIL,v)$\\
		\If{$q\neq p-1$ or $i\neq \frac{n}{p}-1$}{
			$SEND(A[i,0],1)$\\
		}
	}
	\For{j=1 to n-1}{
		\eIf{q=0 and i=0}{
			$A[i,j]\leftarrow UPDATE(A[i,j],A[i-1,j],NIL)$\\
			$SEND(A[i,j],1)$
		}{
			\eIf{q=p-1 and i=$\frac{n}{p}-1$}{
				$RECEIVE(v,1)$\\
				$A[i,j]\leftarrow(A[i,j],A[i-1,j],v)$			
			}{
				$RECEIVE(v,1)$\\
				$A[i,j]\leftarrow UPDATE(A[i,j],A[i-1,j],v)$\\
				$SEND(A[i,j],1)$
			}
		}
	}
}
\end{algorithm}
Where A is  set of $\frac{n}{p}$ rows of size $n$

\note{Alternate version:} Replace line 21 by $SEND(A[i,j],1)||RECV(v,1)$ and add at the end of the program $if\; q\neq \; -1 \; then \;SEND(A[i,n],1)$


\paragraph{Performance:} For the alternate version: 
\[ T=( \underbrace{p-1}_{\substack{\#\text{ step before}\\\text{$P_{p-1}$ starts to}\\\text{work}}}+\underbrace{\frac{n^2}{p}}_{\substack{\# \text{ of elements/}\\\text{steps for $P_{q-1}$}}})\times (w+L+b)\]

\paragraph{Asymptotical efficiency:}
\begin{align*}
e = \frac{p\times T_{seq}(n)}{pT_{p}(n)} & =\frac{pn^2w}{p(p-1+\frac{n^2}{p})(w+b+L)}\\
& \underset{n\to \infty}{\sim} \frac{w}{w+b+L}
\end{align*}


$L$ can be large and the efficiency small.

\paragraph{Coarsen the communications:}
Instead of sending data one at a time, we send them by bulk of $k$ ($k$ divides $n$).

\begin{center}
\begin{tabular}{c | c c c|}
\cline{2-4}
$P_0$ & \hspace{6pt} k & \vline & k \hspace{6pt} \vline \hspace{6pt} k \hspace{6pt} \vline\hspace{6pt} k \hspace{6pt}\vline\hspace{6pt} k \hspace{6pt}\\
\cline{2-4}
$P_1$ &  & \vline &\\
\cline{2-4}
 & & &\\
\cline{2-4}
\end{tabular}
\end{center}

\begin{align*}
T_p(n,k)= & \left( p-1 + \frac{n^2}{pk}\right) (wk+L+kb)\\
e_p(n,k)= & \frac{n^2w}{p[(p-1)(wk+L+kb)+(\frac{n^2w}{p}+\frac{n^2L}{pk}+\frac{n^2b}{p})]}\\
e_p(n,k) & \underset{n\to\infty}{\sim} \frac{w}{w+\frac{L}{k}+b}
\end{align*}

\paragraph{1st solution:} Block distribution
$P_k$ holds the row from $\frac{kn}{p}$ to $\frac{n}{p}-1$.

\paragraph{2nd solution:} (generalization) Block cyclic distribution.

\note{Formally:}
Processor $P_i$ holds row $j$ such that $i=\left\lfloor \frac{j}{r} \right\rfloor$.
\emph{Now in one step} a processor:
\begin{itemize}
\item Updates: $k\times n$
\item Receiving: $k$
\item Sending: $k$
\end{itemize}


\[T_p(n,k,r)=\left( \frac{n^2}{pkr}\right) (krw+L+kb)\]

Let $t=\max (p,\frac{n}{k})(kw+L+kb)$. $P_0$ does not stop if and only if $p\leq \frac{n}{k}$. We assume that $n\geq pk$ to write $T_p(n,k,r)$.

\[e_p(n,k,r) \underset{n\to +\infty}{\sim}\frac{w}{w+\frac{L}{kr}+br}\]

Start form $T_p(n,k,r)$. Fix $n,p$. We differentiate $T_p$ with respect to $k$. $T_p$ is minimized for $k'(r)=n\sqrt{\frac{L}{p(p-1)r(rw+b)}}$.

Best solution: take the two functions $\lceil k'(r)\rceil$ and $\lfloor k'(r)\rfloor$, inject in $T_p$ and look numerically for the best value.

\section{Algorithm for grids of processors}
2D square grid of processors, $p=q^2$.
%scheme (communication between the first and the last processor of each row and column

Each processor link to its four neighbour. We only consider torus: so $P_{i,j}$ has connection to processors $P_{i-1,j},P_{i+1,j},P_{i,j-1 \mod p}, P_{i,j+1 \mod p}$
So every processor belongs to two rings (processor of sane row, processor of same column)

Links are bidirectional: full-duplex (they could have been unidirectional) (no performance degradation if two simultaneous communications (of different directions)).

Simultaneously, a processor can:
\begin{itemize}
\item do some computation
\item receive a message
\item send a message
\end{itemize}

Multiport (4-port):
\begin{itemize}
\item can send/receive 4 messages simultaneously (one per link)
\end{itemize}

1-port:
\begin{itemize}
\item at most one receive (or send) at any time
\end{itemize}

$NUM\_PROCS()$ return the number of processors in the system

$MY\_PROC\_ROW()$ and $MY\_PROC\_COLUMN()$ return the coordinates of the calling processor.

\begin{align*}
SEND&(dest, \underbrace{addr}_{\substack{ \text{address of}\\\text{data to}\\\text{send}}}, \underbrace{L}_{size})\\
RECV&(src,addr,L)\\
BROADCASTROW&( \underbrace{i,j}_{\substack{\text{coordinates of the}\\\text{processor source of}\\\text{the broadcast}}},srcaddr, destaddr)\\
BROADCASTCOLUMN&(i,j,srcaddr,destaddr,L)
\end{align*}

If broadcast is called by a processor whose row is not $i$, we assume that it does nothing and returns immediately.

\subsection{Outer matrix product}
3 square matrix $A,B,C$ of size $n$.
\[C\leftarrow A\times B\]
The three matrices are distributed in the same manner on the processors.

We use a 2D distribution of data. $q=4, p=16, n=16$.

\begin{center}
\begin{tabular}{c | c | c | c}
$\hat{A}_{00}$ & $\hat{A}_{01}$ & $\hat{A}_{02}$ & $\hat{A}_{03}$\\
\hline
$\hat{A}_{10}$ & $\hat{A}_{11}$ & $\hat{A}_{12}$ & $\hat{A}_{13}$\\
\hline
$\hat{A}_{20}$ & $\hat{A}_{21}$ & $\hat{A}_{22}$ & $\hat{A}_{23}$\\
\hline
$\hat{A}_{30}$ & $\hat{A}_{31}$ & $\hat{A}_{32}$ & $\hat{A}_{33}$\\
\end{tabular}
\end{center}

Processor $p_{i,j}$ ($0\leq i,j\leq q-1$) holds the block matrices $\hat{A}_{i,j}, \hat{B}_{i,j}$ and $\hat{C}_{i,j}$ that includes the elements $A_{k,l}, B_{k,l}$ and $C_{k,l}$ (respectively) where $i.m\leq k \leq (r-1)m-1$ where $m=\frac{n}{q}$ and $j.m\leq l \leq (j+1)m-1$.

\paragraph{Sequential product:}$\;$

\begin{algorithm}[H]
\For{k=0 to n-1}{
\For{i=0 to n-1}{
\For{j=0 to n-1}{
$C_{ij}\leftarrow C_{ij}+A_{ik}\times B_{kj}$
}
}
}
\end{algorithm}

(idem with blocks)

\paragraph{Step $k$:} processor holding $\hat{C}_{i,j}$ will completely compute it. $P_{i,j}$ needs block matrices $\hat{A}_{i,k}$ and $\hat{B}_{i,k}$ at step $k$. $\hat{A}_{i,k}$ is owned by $\hat{P}_{i,k}$. So $P_{i,k}$ must send  $\hat{A}_{i,k}$ to $P_{i,j}$ at step $k$ (for all value of $j$). At step $k$, $P_{i,k}$ must broadcast $A_{i,k}$ to its row of processors.

$\hat{B}_{k,j}$ is held by processor $P_{k,j}$. Processor $P_{k,j}$ must send  $\hat{B}_{i,k}$ to $P_{i,j}$ (whatever the value of $i$). At step $k$, $P_{k,j}$ must broadcast  $\hat{B}_{i,k}$ to its column of processors.

%scheme on propagation of values.

\begin{algorithm}[H]
OUTER MatrixMultiplication(A,B,C)\\
$q\leftarrow SQRT(NUM\_PROSC())$\\
$myrow\leftarrow MY\_PROC\_ROW()$\\
$mycol\leftarrow MY\_PROC\_COL()$\\
\For{k=0 to q-1}{
	\For{i=0 to q-1}{
		$BROADCASTROW(i,k,A,bufferA,m\times n)$
	}
	\For{i=0 to q-1}{
		$BROADCASTCOL(i,k,B,bufferB,m\times n)$
	}
	\eIf{(myrow=k)and(mycol=k)}{
		$MATRIXMULTIPLYADD(C,A,B,m)$
	}{\eIf{(myrow=k)}{
		$MATRIXMULTIPLYADD(C,bufferA,B,m)$
	}{\eIf{(mycol=k)}{
		$MATRIXMULTIPLYADD(C,A,bufferB,m)$
	}{
		$MATRIXMULTIPLYADD(C,bufferA,bufferB,m)$
	}}}
}
\end{algorithm}
\[T(m,q)=q(2T_{bcast}+m^3w)\]

Better solution:
\begin{itemize}
\item Communications for $k=0$
\item Loop for $k=0$ to $q-2$ do:
\begin{itemize}
\item Communication for step $k-1$
\item Computation for step $k$
\end{itemize}
\item Computation for step $q-1$
\end{itemize}

$\Rightarrow$ communication and computation overlap 1-port

\[T(m,q)=2T_{bcast}+(q-1)\max (2T_{bcast},m^3w)+m^3w\]
\[T_{bcast}=(\sqrt{(q-2)L}+\sqrt{m^2b})^2\]
\begin{itemize}
\item rings of size $q$
\item matrices size of size $m^2$ are sent
\end{itemize}

\begin{align*}
T_{bcast} &\underset{n\to +\infty}{\sim}\frac{n^2b}{p}\\
T(m,q) &\underset{n\to +\infty}{\sim} qm^3 \omega\\
m=\frac{n}{p} &\qquad p=q^2\\
qm^3w &= q \frac{n^3}{q^3}w\\
& = \frac{n^3}{p}w
\end{align*}

$\to$ there is a factor $\frac{\sqrt{p}}{2}$ between the two cost of communication.

\setcounter{section}{0}
\part{Introduction to distributed systems: the model}
\section{Transition system and algorithms}
\subsection{Transition systems}
Discrete algorithms, algorithm being described by events.
The processes communicate through messages (message passing), messages are asynchronous (no shared memory).

\begin{defi}[Transition system]
A transition system $S$ is a triplet $(C,\to,I)$, with 
\begin{itemize}
\item $C$: a set of configurations
\item $\to$: a binary transition relation on $C$
\item $I$: ($I\subset C$) the subset of the initial      configurations
\end{itemize}
\end{defi}

Each process has a \emph{state}. The set of all the states of the processes is called a \emph{configuration}.

\begin{defi}[Execution]
Let $S=(C,\to,I)$. An execution of $S$ is a maximal sequence $E=(\gamma_0,\gamma_1,\gamma_2,...)$ where $\gamma_0 \in I$, and for all $i\geq 0$, $\gamma_i \to \gamma_{i+1}$
\end{defi}

A terminal configuration is a configuration $\gamma$ such that there does not exist $\delta \in C$ 
such that $\gamma \to \delta$.

\begin{defi}
A configuration $\gamma$ is reachable if there exists $\delta_0 \in I$, and an execution $(\gamma_0,\gamma_1,\gamma_2,...,\gamma_k)$ such that $\gamma_k = \gamma$.
\end{defi}

\subsection{Systems with asynchronous message passing}

A distributed system: A set of processes and a communication subsystem.

Each of the processes is a transition system by itself.

We use the words : \emph{transition} and \emph{configuration} for the whole system, and \emph{event} and \emph{state} for a single process.

Three types of events:
\begin{itemize}
\item Internal event
\item Send event
\item Receive event
\end{itemize}

\begin{defi}
The local algorithm of a process is a quintuple $(Z,I,\vdash^i,\vdash^s,\vdash^r)$ where:
\begin{itemize}
\item $Z$ is the set of states
\item $I\subset Z$ is the set of initial states\
\item $\vdash^i$ is a relation on $Z\times Z$
\item $\vdash^s$ and $\vdash^r$ are relations on $Z\times M \times Z$ where $M$ is the set of possible messages.
\end{itemize}
The binary relation $\vdash$ on $Z$ is defined 

\[c \vdash d \Leftrightarrow c \vdash^i d \vee \exists m \in M \; ((c,m,d)\in \;\vdash^s \cup \; \vdash^r) \]
\end{defi}

A configuration of the system is defined by the state of each of the processes and the message present in the system.

\begin{defi}
The transition system induced, under asynchronous communications, by a distributed algorithm for processes $p_1, ...,p_n$ where the local algorithm for process $p_i$ is $(Z_{p_i},I_{p_i},\vdash^i_{p_i},\vdash^s_{p_i},\vdash^r_{p_i})$ is $S=(C,\to,I)$ where
\begin{enumerate}
\item $C=\{(c_{p_1},...,c_{p_n},M)\}$ with $\forall p \in \underbrace{\{p_1,...,p_n\}}_{=\mathbb{P}}, c_p\in Z_p$, and $ M \in \underbrace{\mathbb{M}}_{\text{multiset}}(\underbrace{\mathcal{M}}_{\text{set of messages}})$
\item $I=\{ c_{p_1},...,c_{p_n},M), \forall p \in \{ p_1,...,p_n\},  c_p \in I_p \text{ and } M=\emptyset \}$
\item $\to \; = \; \bigcup_{p\in \mathbb{P}} \to_p$\\
$\to_{p_i}$ the set of pairs $(c_{p_1},...,c_{p_i},...,c_{p_n},M_1),(c_{p_1},...,c_{p_{i-1}},c_{p_i}'c_{P_{i+1}},...,c_{p_n},M_2)$ for which one of these conditions hold:
\begin{itemize}
\item $(c_{p_i},c_{p_i}')\in \; \vdash^i_{p_i}$ and $M_1=M_2$
\item for some $m\in \mathcal{M}, (c_{p_i},m,c_{p_i}') \in \;\vdash_{p_i}^s$ and $M_2=M_1 \cup \{m\}$
\item for some $m\in \mathcal{M}, (c_{p_i},m,c_{p_i}') \in \;\vdash_{p_i}^r$ and $M_1=M_2 \cup \{m\}$
\end{itemize}
\end{enumerate}
\end{defi}

\section{Proving properties}
\subsection{Safety property}

A safety property $P$ is a property that is true in each configuration of each execution of the algorithm ($P$ is always true). A safety property is called an invariant.

If $\gamma$ is a configuration and $P$ is a propriety, $P(\gamma)$ is the boolean value of $P$ on the configuration $\gamma$.

We write $\{P\}\to\{Q\}$ to denote that for each transition $\gamma \to \delta$, if $P(\gamma)$ then $Q(\delta)$. Hence, $\{P\}\to\{Q\}$ means that if $P$ holds before the transition, then $Q$ holds afterwards.

\begin{defi}
An assertion $P$ is an invariant if
\begin{enumerate}
\item $\forall \gamma \in I, P(\gamma)$
\item $\{P\}\to\{P\}$
\end{enumerate}
\end{defi}

\begin{thm}
If $P$ is an invariant of $S$, then $P$ holds for each configuration of each execution.
\end{thm}

\subsection{Liveness property}
Let $P$ be an assertion true in some configuration of each execution of the algorithm (property $P$ will eventually be true).

Let $S$ be a system, and $P$ a property. Let $term$ be a predicate true in all terminal configuration and false otherwise.

\begin{defi}
The system $S$ terminates properly (or is deadlock-free) if the predicate $(term \Rightarrow P)$ is always true.
\end{defi}

\begin{defi}
A partial order $(W,<)$ is well founded if there is no infinite decreasing sequence.
\end{defi}

\begin{defi}
Let $S$ be a transition system, and $P$ an assertion. A function $f$ from $C$ to a well-founded set $W$ is called a norm function (with respect to $P$) if for each transition $\gamma \to \delta$, then $f(\gamma) > f(\delta)$ or $P(\gamma)$
\end{defi}

\begin{thm}
Let $S$ be a transition system, and $P$ be an assertion. If S terminates properly and a norm function $f$ (with respect to $P$) exists, then $P$ is true in some configuration of each execution of $S$.
\end{thm}


\section{Causal order of events and logical clocks}
The view of execution as sequences of transitions naturally induces a notion of time. Events of an execution (of a distributed system) can \emph{sometimes} be interchanged.

\subsection{Independence and dependence of events}
Two consecutive events that influence disjoint parts of the system are independent and can occur in reversal order (you can interchange them).

\begin{thm}
Let $\gamma$ be a configuration, and let $e_p$ be events of different processes $p$ and $q$, both applicable in $\gamma$. Then $e_p$ is applicable in $e_q(\gamma)$ and $e_q$ is applicable in $e_p(\gamma)$ and $e_p(e_q(\gamma))=e_q(e_p(\gamma))$.
\end{thm}

The premise of this theorem applies to any pair of events $e_p$ and $e_q$ except if \textit{(i)} $p=q$ and except if \textit{(ii)} $e_p$ is a send event and $e_q$ is the corresponding receive event.

\begin{defi}
Let $E$ be an execution. The relation $\prec$, called the casual order, on the events of the execution is the smallest relation that satisfies:
\begin{enumerate}
\item if $e$ and $f$ are events of the same process and $e$ occurs before $f$, then $e\prec f$
\item if $e$ is a send and $f$ is the corresponding receive, then $e \prec f$
\item $\prec$ is transitive
\end{enumerate}

$a\preceq b$ if $a\prec b$ or $a=b$. $\preceq$ is a partial order. There may be event $a$ and $b$ such that neither $a \preceq b$ nor $b\preceq a$. Such events are called \emph{concurrent}.
\end{defi}


\subsection{Equivalence of executions}
Let $f=(f_0,f_1,f_2,...)$ be a sequence of events. It is related to an execution $F=(\delta_0,\delta_1,\delta_2,...)$ if for each $i$, $f_i$ is applicable to $\delta_i$ and $f(\delta_i)=\delta_{i+1}$. $F$ is called the implicit execution of $f$. Let us consider a permutation $\sigma$ of the events. The permutation $(f_{\sigma(0)},f_{\sigma(1)},f_{\sigma(2)},...)$ of the events is consistent with the casual order if 
\[f_{\sigma (i)} \prec f_{\sigma(j)} \Rightarrow i \leq j\]

\begin{thm}
Let $f'=(f_{\sigma(0)},f_{\sigma(1)},f_{\sigma(2)},...)$ be a permutation of the events of $f$ that is consistent with the casual order. Then $f$'defines a unique execution $F'$ starting in the execution of $F$. $F'$ has as many events as $F$, and if $F$ is finite, then $F'$ has the same last configuration.
\end{thm}

Executions $F$ and $F'$ have the same collection of events, and the causal order of these events are the same. We say that execution $F$ and $F'$ are equivalent, and we denote $F\sim F'$.

\begin{defi}
A computation if a distributed algorithm is an equivalence class under $\sim$ of executions of the algorithms.
\end{defi} 


\subsection{Logical clocks}
\begin{defi}
A clock is a function $\Theta$ from the events to an order set such that
\[a \prec b \Rightarrow \Theta(a)<\Theta(b)\]
\end{defi}

\subsubsection{Order in sequence}
Execution $E$ defined by a sequence of events $(e_0,e_1,e_2,...)$. Set $\Theta_g (e_i)=i$. This \emph{cannot} be computed within the system, this cannot be computed by a distributed algorithm.

\subsubsection{Lamport's logical clock}
Event $a$.
Let $k$ be the length of the longest sequence such that $e_1 \prec e_e\prec ... \prec e_k=a$ then $\Theta_L(a)=k$. Obviously, $a\prec b \Rightarrow \Theta_L(a) < \Theta_L(b)$
\begin{enumerate}[a.]
\item if $a$ is an internal event or a send event and $a'$ is the previous event of the same process then $\Theta_L (a)=\Theta_L(a')+1$
\item If $a$ is a receive event, $a'$ the previous event in the same process and $b$ the corresponding send event, $\Theta_L(a) = 1 + \max (\Theta_L (a'), \Theta_L (b) )$. In order to compute $\Theta_L$, we add to each message the clock of the sending event.
\end{enumerate}

\newpage
\setcounter{section}{0}
\part{Wave and traversal algorithms}

Wave algorithms cover broadcasting, synchronisation, compute some global function (e.g., maximum of values stored among the processes).

\section{Definition and use of the wave algorithm}
\begin{itemize}
\item Topology is fixed (no communication link will appear during the execution of the algorithm)
\item Communication links/channels are undirected: for two processes $p$ and $q$, if $p$ can send a message to $q$, then $q$ can send a message to $p$.
\item the set of processes is connected: $\forall p,q \in \mathbb{P}$, there is a path from $p$ to $q$, where $\mathbb{P}$ is the set of processes and $E$ a set of communication channels/links
\item Asynchronous messages
\item No global link
\end{itemize}

\subsection{Definition of wave algorithms}
\begin{itemize}
\item A special type of internal event: \emph{decide}.
\item If $C$ is a computation, $|C|$ is the number of event in $C$.
\end{itemize}

\begin{defi}
A wave algorithm is a distributed algorithm that satisfies 3 properties:
\begin{enumerate}
\item Termination: each computation $C$ is finite, i.e. $|C|<+\infty$
\item Decision: each computation contains at least one decide event, i.e. $\forall C, \exists e \in C$, $e$ is a decide event
\item Dependence: in each computation, \emph{each} decide event is causally preceded by an event of \emph{each} process, i.e. $\forall C, \forall e\in C : \text{$e$ is a decide event } \Rightarrow \forall p \in \mathbb{P}, \; \exists f \in C_p,\; f\preceq e$ where $C_p$ is the set of events of computation $C$ happening on process $p$)
\end{enumerate} 
\end{defi}

We call \emph{wave} one computation of a wave algorithm; \emph{indicators (or starters)} the processes that spontaneously start the algorithm and \emph{non-initiate (or followers)} the processes that start their algorithm on reception of a message.

\subsubsection*{Differences among wave algorithms}
\begin{enumerate}
\item \emph{Centralized}: algorithm that always has exactly one initiator (the opposite is \emph{decentralised}). Centralized = single-source
\item \emph{Topology}: algorithm may be design for a special topology for any topology
\item \emph{Initialisation}: 
\begin{enumerate}
\item Process identity = each process knows its unique name
\item Neighbour's identity
\end{enumerate}
\item \emph{Number of decisions}: In all cases we are going to consider: at most one decision per process. However, one processor or all processors can take a decision (or in one case two decisions)
\item \emph{Complexity}: 
\begin{itemize}
\item The number of messages exchanged
\item The number of bit exchanged
\item (The time of a computation)
\end{itemize}
\end{enumerate}
Often messages will be empty (signals)

\subsection{Elementary results on wave algorithms}
\subsubsection{Structural properties}
\begin{lemma}
$\forall e \in C$, there exists an initiator $p$ and an event $f\in C_p$ such that $f \preceq e$
\end{lemma}

\begin{proof}
Let $f$ be any minimal predecessor of $e$ $f \preceq e$ and $\neg (\exists g, g\prec f)$. Let $p$ be the process where $f$ happens. If $p$ is not an initiator, the first event $g$ on $p$ was a received event. Let $h$ be the corresponding send event
\[\underbrace{h}_{\text{send}} \prec \underbrace{g}_{\text{receive}}\preceq f \]
which contradicts the definition of $f$.
\end{proof}

\begin{lemma}
Let $C$ be a wave with one initiator $p$, then for each non-initiator $q$, let $father_{q}$ be the neighbour of $q$ which $q$ received a message in its first event. Then the graph defined by $(\mathbb{P}, E_T)$ where $E_T = \{(q,father_q) \;| \; q\in \mathbb{P}\smallsetminus \{ p \} \}$ is a spanning tree.
\end{lemma}

\begin{proof}
wave $\Rightarrow$ at least one decide event $\Rightarrow$ at least one event in each process $\Rightarrow$ all processes (except maybe $p$) have received at least one message $\Rightarrow$  $father_q$ was defined for all $q\in \mathbb{P} \smallsetminus \{p\}$ we have exactly $N-1$ edges ($N=|P|-1$). We must show that there is no cycle in the graph. By definition, if $(q,father_q)\in E_T$, then $father_q \prec q$ $\Rightarrow$ no cycle possible.
\end{proof}

\begin{lemma}
Let $C$ be a wave and $d_p \in C$, $d_p$ being a decide event. Then $\forall q \neq p$: $\exists f \in C_q $ ($f\preceq d_p \land f$ is a send event)
\end{lemma}

\begin{proof}
Let $q \neq p$ be any process. By definition of wave algorithms, there exists an event $g \in C_q$ such that $q\preceq d_p$. We consider on (arbitrary chosen) causality path(?) between $g$ and $d_p$.
$g=g_0 \prec g_1 \prec g_2 \prec ...\prec g_n = d_p \quad g_0\in C_q$. We consider the last event in $g_0,...,g_{n-1}$ that is an event of $q$ (say $g_i$). $g_i\in C_q, g_{i+1} \notin C_q \Rightarrow g_i$ is a send event.
\end{proof}

\subsubsection{Lower bounds on the complexity of waves}
(Lower bound on the number of messages)

Corollary of previous lemma: at least $N-1$ messages (sends) in any execution.

\begin{thm}
Let $C$ be a wave with only one initiator $p$, such that a decide event occurs in $p$. Then at least $N$ messages are exchanged in $C$.
\end{thm}

\begin{thm}
Let $A$ be a wave algorithm, for arbitrary networks without any initial knowledge of the neighbour identities. Then $A$ exchanges at least $|E|$ messages.
\end{thm}

\begin{proof}
By contradiction, there is a wave algorithm $A$ and a graph $G=(\mathbb{P},E)$ such that there exists a computation $C$ for which $A$ take a decision using at most $|E| - 1$ messages. $\Rightarrow$ There is at least one edge $e=(p,q)$ such that no message is exchanged through that link.

Let $G'=G$ except I remove edge $(p,q)$ and add a process $z$ and two edges $(p,z)$ and $(z,q)$. I execute on $G'$ all the events on $C$. This is possible because nothing on $C$ dealt with the edge $(p,q)$ I removed. As in $G$, $A$ reaches a decision on $G'$. However, no events had place on process $z$, which contradicts the \emph{dependence} property of wave algorithms.
\end{proof}

\section{Some wave algorithm}
\subsection{Ring algorithm}
The topology is a ring. Each process $p$ has a dedicated neighbour $next_p$. We note $<tok>$ a token.

\paragraph{Algorithm}
\begin{itemize}
\item For the initiator:\\
send $<tok>$ to $next_p$\\
receive $<tok>$\\
decide
\item For the non initiators:\\
receive $<tok>$\\
send $<tok>$ to $next_p$
\end{itemize}

\begin{thm}
The ring algorithm is a wave algorithm.
\end{thm}

\begin{proof}
\begin{itemize}
\item Each execution has a finite number of events ($|C|\leq 2N+1$)
\item We must show that there is at least one decide event and that the decide event is preceded (causally) by an event in each other process. We look at a terminal configuration $\gamma$. Let $p_0$ be the initiator.
\begin{itemize}
\item In $\gamma$, $p_0$ has sent a token (otherwise it can do it and $\gamma$ is not terminal)
\item In $\gamma$ there is no $<tok>$ message in the system (because it could be received and $\gamma$ would not be terminal).
\item Each non initiators that received the token has sent it (otherwise $\gamma$ non terminal) $\Rightarrow$ All non initiator processes has received the token, then $p_0$ has received the token, and $p_0$ has taken a decision, and this decision depends on a send in each other process.
\end{itemize}
\end{itemize}
\end{proof}

\subsection{The tree algorithm}
\begin{itemize}
\item The topology of the network is a spanning tree (can be applied on arbitrary network with a spanning tree).
\item All the leaves of the tree are initiators.
\end{itemize}

\begin{algorithm}
$rec_p[$number of neighbour of $p]$ initialised to false\\
\While{$|\{q\; | \; rec_p[q] = \text{false} \}|>1$}{
	Receive $<tok>$ from any $q$\\
	$rec_p[q]=$ true\\
}
send $<tok>$ to $q_0$ such that $rec_p[q_0]$=false\\
receive $<tok>$ from $q_0$\\
$rec_p[q_0]=$ true\\
decide\\
(\For{all $q \in Neighbour(p) \neq q_0$}{
	send $<tok>$ to $q$
}) \tcp{Propagation on decision}
\end{algorithm}


\subsection{The echo algorithm}
Centralized wave algorithm for any arbitrary network. The algorithm floods the network with $<tok>$ messages.

\begin{algorithm}
$rec_p\leftarrow 0$ \tcp{\# of received message}
$father_p \leftarrow$ undefined\\
\For{The initiator}{
	\For{all $q \in Neighbourhood(p)$}{
		send $<tok>$ to $q$\\
	}
	\While{ $rec_p < \# Neighbourhood(p)$}{
		receive $<tok>$\\
		$rec_p \leftarrow rec_p + 1$\\
	}
	decide\\
}
\For{non initiators}{
	receive $<tok>$ from some neighbour $q$\\
	$father_p \leftarrow q$\\
	$rec_p \leftarrow rec_p +1$\\
	\For{all $q\in Neighbourhood(p) \smallsetminus \{father_p\}$}{
		send $<tok>$ to $q$\\
	}
	\While{$rec_p < \# Neighbourhood(p)$}{
		receive $<tok>$\\
		$rec_p\leftarrow rec_p + 1$\\
	}
send $<tok>$ to $father_p$
}
\end{algorithm}


The set of values of $faher_p$ define a spanning tree. One can prove that the decide event is causally preceded be an event in each process: We do that by induction on the spanning tree, starting by the leaves.

Each process holds a value (integer). How can I compute the minimum ?

\subsection{Usage of wave algorithms}
\subsubsection{Computation of infinimum functions}

If $(X,\leq)$ is a partial order. $c$ is the infinimum of $a$ and $b$ if $c\leq a$ and $c\leq b$ and $\forall a (d\leq a \text{ and } d\leq b, \text{ then } d \leq c)$ infinimum of $a$ and $b$ $a\land b$

\paragraph{Infinimum computation:} Each process $q$ holds an input $j_q$, in a partially ordered set. Some processes compute the value of infinimum of $\{j_1\}_q$ and the processes knows when the computation has completed. They write the outcome of the computation as a variable out and are not allowed to changed it afterwards.

\begin{thm}
Every Infinimum computation algorithm is a wave algorithm.
\end{thm}


\begin{thm}
Every wave algorithm can be used to compute an infinimum.
\end{thm}

\begin{thm}
If $\star$ is a binary operator on a set $X$ such that
\begin{enumerate}[1.]
\item $\star$ is commutative
\item $\star$ is associative
\item $\star$ is idempotent ($a \star a = a$)
\end{enumerate}
Then there is a partial order $\leq$ on $X$ such that $\star$ is an infinimum function.
\end{thm}

\begin{coro}
We can compute $\land, \lor, gcd, lcm, \min, \max, \cup, \cap$ of local values using wave algorithms
\end{coro}

\subsubsection{Synchronisation}
Synchronization algorithm: In each process $q$ an event $a_q$ must be executed and, in some process, an event $b_p$ must be executed such that the execution of all $a_q$ events must take place temporarily before any event $b_p$.

\begin{thm}
Every synchronization algorithm is a wave algorithm.
\end{thm}

\begin{thm}
Every wave algorithm can be employed as a synchronization algorithm.
\end{thm}

\subsubsection{Propagation of information with feedback (PIF)}
A subset of processes have a same message $M$ which must be broadcast (all processes must receive it and acknowledge it). Certain processes must be notified of termination of the broadcast (they must execute a notify statement), but only after al processes have received the message. There is a finite number of messages exchanged.

\begin{thm}
Any PIF algorithm is a wave algorithm.
\end{thm}

\begin{thm}
Every wave algorithm can be used as a PIF algorithm.
\end{thm}

\section{Traversal algorithm}
\begin{defi}
A traversal algorithm is a wave algorithm with the following time proprieties
\begin{enumerate}[i.]
\item In each computation, there is a single initiator. The first time the initiator sends some messages, it sends a signle message and does not send any more message before receiving a message.
\item A process upon reception of a message, either sends out one message or decides.
\item The algorithm terminates in the initiator and when this happens each process has sent a message at least once
\end{enumerate}
\end{defi}

At any time there is at most one message in the system: there is a single token that is passed around processes. Traversal algorithms are wave algorithms in which all events are totally ordered by the causality relation.


\subsection{Traversing cliques}
Clique = complete graph


\begin{algorithm}
$rec_p \leftarrow 0$\\
\For{the initiator}{
	\While{$rec_p < \#Neighbourhood(p)$}{
		send $<tok>$ to $q_{rec_p}$\\
		receive $<tok>$\\
		$rec_p\leftarrow rec_p +1$\\
	}
	decide\\
}
\For{the non-initiator}{
	receive $<tok>$ form some process $q$\\
	send $<tok>$ to $q$\\
	(the sequential polling algorithm)\\
}
\end{algorithm}


\subsection{Traversing connected network}

\begin{algorithm}[H]
$used_p[q]\leftarrow false\; (\forall q \in Neighbourhood(p))$\\
$father_p \leftarrow$ undefined\\
\For(execute one){the initiator}{
	$father_p \leftarrow p$\\
	choose $q\in Neighbourhood(p)$\\
	$used_p[q]\leftarrow$ true\\
	send$<tok>$ to $q$\\
}
\For{each process, upon reception of $<tok>$ from $q_0$}{
	\If{$father_q$ is undefined}{
		$father_p \leftarrow q_0$\\
	}
	\eIf{$\forall q \in Neighbourhood(p) \text{such that} q\neq father_p \text{ and } used_p[q] is false$}{
		choose $q\in Neighbourhood(p) \smallsetminus \{father_p\}$ such that $used_p[q]$ is false\\
		$used_p[q]\leftarrow$true\\
		send $<tok>$ to $q$
	}{
		$used_p[father_p]=$ true\\
		send $<tok>$ to $father_p$\\
	}
}
\end{algorithm}

\section{Election algorithms}


\paragraph{Problem:} Start form a configuration where all processes are in the same state, and arrive in a configuration where exactly one process is in the state leader and all the other are in state lost.

\begin{defi}
An election algorithm is an algorithm that satisfies the following proprieties:

\begin{enumerate}[i.]
\item Each process has the same local algorithm
\item The algorithm is decentralized (an arbitrary number of initiators)
\item The algorithm reaches a terminal configuration in each computation, and in each computation, and each reachable terminal configuration, there is exactly one process in the state leader and all the other ones in the state lost.
\end{enumerate}
\end{defi}

\paragraph{Assumptions:}
\begin{enumerate}
\item System is fully asynchronous.
\item Each process is identified by a unique name which it initially knows.
\end{enumerate}

\begin{itemize}
\item We design election algorithms such that the process with the smallest name (identifier) is elected.
\item We can use a wave algorithm to compute the minimum of the names.
\item Our previous solution was the echo algorithm, a centralized algorithm. We can add to it a preliminary wake-up phase.
\end{itemize}


\paragraph{Election algorithm based on the tree algorithm.}
\begin{itemize}
\item Boolean $ws_p$: whether process $p$ has woken up.
\item Integer $wr_p \leftarrow 0$: number of wake up message received.
\item boolean $rec_p[q]$: whether process $p$ receive a message (after the wake-up phase) from process $q$.
\item $v_p$: temporary variable to store the id of the current minimum
\item $state_p$: (sleep, leader, lost)
\end{itemize}


\begin{algorithm}[H]
\If{$p$ is the initiator}{
	$ws_p \leftarrow$ true\\
	\For{all $q\in Neighbourhood(p)$}{
		send $<wake\; up>$ to $q$\\
	}
}
\While{$wr_p < \# Neighbouhood(p)$}{
	receive $<wake\; up>$\\
	$wr_p\leftarrow wr_p+1$\\
	\If{not $ws_p$}{
		\For{all $q\in Neighbourhood(p)$}{
			send $<wake\; up>$ to $q$\\
		}
	}
}
\tcp{Start of the tree algorithm}
\While{$\# \{q\;|\; \neg rec_p[q]\}>1$}{ 
	receive $<tok, r>$ from $q$\\
	$rec_p[q]\leftarrow$ true\\
	$v_p\leftarrow \min(v_p,r)$\\
}
send $<tok, v_p>$ to $q_0$ such that $rec_p[q_0]=$ false\\
receive $<tok, r>$ from $q_0$\\
$v_p\leftarrow \min(v_p,i)$ \tcp{decide}
\eIf{$v_p=p$}{
	$state_p\leftarrow$leader\\
}{
	$state_p\leftarrow$lost\\
}
\For{all $q\in Neighbourhood(p) \smallsetminus \{q_0\}$}{
	send $<tok, v_p>$ to $q$\\
}
\end{algorithm}


\section{Extinction principle and the echo algorithm}
\begin{itemize}
\item We extend a wave algorithm
\item All initiators are starting a wave
\item We tag each wave by its initiator
\item We distinguish messages and only forward those corresponding to the smallest initiator
\end{itemize}


$caw:$ currently active wave.

\begin{algorithm}[H]
\tcp{Variables:}
$caw_p\leftarrow$ undefined (minimum of the identifiers of encountered waves)\\
$rec_p\leftarrow 0$ \# of messages received for the wave $caw_p$\\
$father_p\leftarrow$undefined (father in $caw_p$)\\
$lrec_p\leftarrow 0$ \# of leader messages received\\
$win_p\leftarrow$undefined (identity of the leader)\\
\If{$p$ is initiator}{
	$caw_p\leftarrow p$\\
	\For{all $q \in Neighbourhood(p)$ }{
		send $<tok, caw_p>$ to $q$\\
	}
}
\While{$lrec_p < \# Neighbourhood(p)$}{
	receive $msg$ from $q$\\
	\eIf{$msg = <leader, r>$}{
		\If{$lrec_p=0$}{
			\For{all $q'\in Neigbhourhood(p)$}{
				send $<leader, r>$ to $q$\\
			}
		}
		$lrec_p\leftarrow lrec_p+1$\\
		$win_p\leftarrow r$\\
	}{
		\tcp{$msg=<tok,r>$}
		\If{$r<caw_p$}{
			$caw_p\leftarrow r$\\
			$rec_p\leftarrow 1$\\
			$father_p\leftarrow q$\\
			\For{all $s\in Neighbourhood(p)\neq q$}{
				send $<tok, r>$ to $s$\\
			}
		}\ElseIf{$r=caw_p$}{
			$rec_p\leftarrow rec_p+1$\\
			\If{$rec_p=\# Neighbourhood(p)$}{
				\If{$caw_p=p$}{
					\For{all $s\in Neighbourhood(p)$}{
						send $<leader, p>$ to $s$\\
					}
				}\Else{
					send $<tok, caw_p>$ to $father_p$\\
				}
			}
		}
	}
}
\If{$win_p=p$}{
	$state_p=$ leader\\
}\Else{
	$state_p=$ lost\\
}
\end{algorithm}

\setcounter{section}{0}
\part{Task-graph scheduling}
\section{Introduction}
\subsection*{Where do task graphs come from?}
Linear system $Ax=B$. $A$ is a lower triangular matrix ($b$ and $x$ are two vectors), $b$ is known. $A$ is a $n \times n$ matrix.

\begin{algorithm}[H]
\For{$i=1$ to $n$}{
	$T_{ii}:x_i\leftarrow \frac{b_i}{a_{ii}}$\\
	\For{$j=i+1$ to $n$}{
		$T_{ij}: b_j\leftarrow b_j-a_{j,i}.x_i$\\
	}
}
\end{algorithm}


Original algorithm is sequential.


Total ordering of the tasks:
\[T_{1,1} <_{seq} T_{1_2} <_{seq} T_{1,3} <_{seq} ... <_{seq} T_{1,n} <_{seq} T_{2,2} <_{seq} 2,3 <_{seq} ... <_{seq} T_{n,n}\]

Some parallelism: $T_{1,2}$ and $T_{1,3}$ are independent.

Condition for two tasks $T$ and $T'$ to be a dependence relation:
\begin{itemize}
\item both access the same variable and at least one access is a \emph{write} access.
\item $In(T)$: the set of variables read by task $T$
\item $Out(T)$: the set of variable written by task $T$
\end{itemize}

$T$ and $T'$ are independent ($T \perp T'$)

\[T \perp T' \Leftrightarrow
\begin{cases}
In(T)\cap Out(T')\neq \emptyset & \text{or}\\ 
Out(T)\cap In(T') \neq \emptyset &\text{or}\\
Out(T)\cap Out{T'} \neq 0
\end{cases}
\]

We define a partial order $\prec$ by:\\
if $T\perp T'$ and $T<_{seq} T'$ then $T\prec T'$

$\prec$ is $(<_{seq}\cap \perp)^+$ (transitive closure)

Representation of dependencies with a graph:
\[G=(\underbrace{V}_{\text{set of tasks}},\underbrace{E}_{oriented edges})\]
With
\begin{align*}
e=(T,T')\in E & \Leftrightarrow T \prec T'\\
\big(\text{and } \nexists V \text{ s.t. } T\prec T' & \text{ and } V \prec T' \big)\\
\end{align*}

(we do not include transitive edges for the sake of readability)

\section{Scheduling task graph}
\begin{defi}
A task graph is a directed weighted graph $G=(V,E,\omega )$

Where:
\begin{itemize}
\item The set $V$ represent the tasks
\item The set $E$ of edges represents the dependences: $e=(u,v) \in E $ if and only if $u\prec v$.
\item A weight function $\omega : V \to \mathbb{N} $ gives the weight or execution time of a test.
\end{itemize}
\end{defi}

\begin{defi}
A schedule of a task graph $G(V,E,\omega)$ is a function
$\sigma : V\to \mathbb{N}$ such that if $e=(u,v) \in E$ then $\sigma (u) + \omega (v) \leq \sigma (v)$. $\sigma$:starting time of the tasks.
\end{defi}

If we have a limited number $p$ of processor (if $p<|v|$), $alloc$: $v\to \{1,...,p\}$ states on which processor a task is executed.

\[alloc(u) = alloc (v) \Leftrightarrow 
\begin{cases}
\sigma(u) + \omega{v} \leq \sigma(v) & \text{or}\\
\sigma(v) + \omega(v) \leq \sigma(u)
\end{cases}\]

\begin{thm}
Let $G$ be a task graph. There exists a schedule for $G$ if and only if $G$ does not contain any cycle.
\end{thm}

\begin{proof}
\begin{itemize}
\item There is a cycle $T_1 \to T_2 \to ... \to T_k \to T_1$, so  $T_1 \prec T_2 \prec ... \prec T_k \prec T_1$, and $T_1\prec T_1$, $\sigma(T_1) + \omega(T_1) \leq \sigma(T_1)$ impossible.
\item There is no cycle
\item Topological order
\end{itemize}
\begin{algorithm}
pick a task $T$ without predecessor\\
$\sigma(T) \leftarrow t$\\
$t\leftarrow t + \omega(T)$\\
remove $T$ from $G$\\
repeat\\
\end{algorithm}
\end{proof}

\begin{defi}[Makespan]
Task graph $G=(V,E,\omega )$
\begin{enumerate}
\item Let $\sigma$ be a schedule for $G$ using $p$ processors. The makespan of $G$ is the total execution time.
\[MS(\sigma, p) = \max_{v\in V} \{\sigma (v) + \omega (v) \} - \min_{v\in V}\{\sigma ( v)\}\]
Usually we assume that $\min_{v\in V}\{\sigma ( v)\}=0$

\item $Pb(p)$: Problem of finding a best scheduling (i.e. a schedule of minimum makespan) using $p$ processors. Let $MS_{opt}(p)$ be the makespan of an optimal scheduling using $p$ processors.

$\omega : V \to \mathbb{N}^*$. We extend it on paths $\Phi : T_1,T_2,...,T_k$ where $T_1 \prec T_2 \prec ... \prec T_k$ by $\omega ( \Phi) =\sum_{T_i \in \Phi} \omega (T_I)$
\end{enumerate}
\end{defi}

\begin{prop}
Let $G=(V,E,\omega)$ be a DAG (directed acyclic graph) and $\sigma$ a schedule for $G$ with $p$ processors.

Then $MS(\sigma, p)\geq \omega (\Phi) \geq \omega (\Phi)$ for all paths $\Phi$ in $G$.
\end{prop}

\begin{proof}
Let $\Phi$ be any path. 
\[\Phi=T_1 \prec T_2 \prec ... \prec T_k\]
\[T_i \prec T_{i+1} \Rightarrow \sigma (T_i) + \omega ( T_{i+1} \leq \sigma (T_{i+1})\]

Sum for $i$ to $d-1$:
\[\sigma (T_1) + \sum_{i=1}^{k-1} \omega(T_i) \leq \sigma(T_k) \]
\[MS(\sigma, p) \geq \sigma (T_k) + \omega (T_1) \geq \omega (\Phi)\]
\end{proof}


\begin{defi}
$G=(V,E,\omega)$, $\sigma$ on $p$ procc.
\begin{enumerate}
\item Speedup ratio:
\[s(\sigma, p)=\frac{seq}{MS(\sigma,p)} = \frac{\sum_{v\in V} \omega (v)}{MS(\sigma,p)}\]
\item Efficiency:
\[e(\sigma, p)=\frac{s(\sigma, p)}{p} = \frac{\sum_{v\in V} \omega (v)}{p.MS(\sigma,p)}\]
\end{enumerate}
\end{defi}


\begin{thm}
Let $G=(V,E,\omega)$ for any schedule $G$ on $p$ process $0\leq e(\sigma, p) \leq 1$
\end{thm}

\begin{proof}
The area of a rectangle on the scheduling timetable is $p.MS(\sigma, p)=Idle+work=Idle+seq$ and $l=Idle + \frac{Seq}{p.MS(\sigma,p)}=Idle+e(\sigma,p)$
\end{proof}

\begin{thm}
Let $G=(V,E,\omega)$ a task graph.
\[seq=MS_{opt}(1)\geq ... \geq MS_{opt}(n)\]
\end{thm}

Let $MS'(p)$ be the minimum makespan of all schedule that use exactly $p$ processors. Then for all $1\leq p \leq |V| \; MS'(p)=MS_{opt}(p)$, because there is no communication (implicit hypothesis).



\section{Solving problems ($\infty$)}
We have an unbounded number of resources ($p\geq |V|$). Let $G=(V,E,\omega)$ be a task system.

\begin{defi}
\begin{enumerate}
\item 
$\forall v \in V$:
\begin{itemize}
\item $Pred (v)=$ set of immediate predecessors of a task $v$.
\item $Succ(v)=$ set of immediate successors.
\end{itemize}

\item 

\begin{itemize}
\item $v$ is an entry task if $Pred(v)=\emptyset$
\item $v$ is an exit task if $Succ(v)=\emptyset$
\end{itemize}

\item The top level $tl(v)$ is the maximum weight of a path from an entry task to the task $v$, excluding the weight of $v$.\\
$tl(v)$: lower bound on the time elapsed in any schedule between the start of the execution of $G$ and the start of $v$.

\item The bottom level $bt(v)$ is the maximum weight of a path from the task $v$ (included) to an exit task (included).\\
$bt(v):$ lower bound on the execution time on any schedule once the execution of $v$ has started.
\end{enumerate}

\[tl(v) = \max \{ tl(v) + \omega (v) \; | \; v \in Pred(v) \}\]
computed through a traversal of $G$ ($O(|V|+|E|)$)
\end{defi}

\begin{thm}
Let $G=(V,E,\omega)$ be a task system. We define $\sigma_{free}$ as follow:
\begin{itemize}
\item $\sigma_{free}=tl(v) \quad \forall v \in V$
\item $\sigma_{free}$ is an optimal schedule for $G$
\end{itemize}
\end{thm}
\begin{proof}
This is a schedule (dependences are satisfied by definition).

We show by induction (through a topological sort) that each task starts as soon, as possible.

ASAP schedule.
\end{proof}

\paragraph{Another optimal schedule:} as late as possible schedule: $\sigma_{ALAP}(v)=MS_{OPT}(\infty) - bt(v)$

\section{Solving Pb(p)}
\subsection{NP-completeness of Pb(p)}

\paragraph{(2-)partition:} given a set of $n$ (strictly) positive integers $a_1,...,a_n$, is there a subset $I$ of $\{1,...,n\}$ such that $\sum_{i\in I} a_i=\sum_{i\notin I}a_i$

NP-hard (weak sense, there is a pseudo polynomial algorithm to solve it)


\paragraph{3-partition:} we have a set of $3n$ (strictly) positive integers $b_1,...,b_n$. Let $B$ be such that $\sum_{i=1}^{3n}b_i=nB$. We assume that for each $i\leftarrow \{ 1,...,3n\}$, $\frac{b}{4}\leq b_i \leq \frac{B}{2}$. Can we partition the $3n$ $b_i$'s in $n$ subsets $I_1, ..., I_n$ such that $\sum_{i\in I} b_i = B$?

\begin{defi}
Let $Dec(p)$ the problem; let $G=(V,E,\omega)$ be a tsk system, let $p$ be the number of processors, and let $k\in \mathbb{N}^*$, is there a schedule for $G$ using at most $p$ processors, whose makespan is no greater than $k$?\\
If $E=\emptyset$ (no dependencies) we denote the problem by $Indep(p)$.
\end{defi}  

\begin{thm}
\begin{itemize}
\item $Indep(2)$ is NP-complete but can be solved in pseudo-polynomial time
\item $Indep(p)$ is NP-complete in the strong sense
\item $Dec(2)$ is NP-complete in the strong sense
\end{itemize}
\end{thm}

\begin{proof}
\begin{itemize}
\item $Indep(2)$:
\begin{itemize}
\item Problem belongs to NP, for each proc, check that the sum of the weights of the tasks associated to it does not exceed $k$.
\item Reduction from 2-partition:\\
$k=\frac{1}{2} \sum_{i=1}^{n}a_i$, we have an identical problem.
\end{itemize}
\item $Indep(p)$:\\
Reduction from 3-partition: $b_1,...,b_{3n}$ of 3-partition, we take $p=n$ and $k=B$.

\item $Dec(2)$:\\
Reduction from 3-partition
\begin{itemize}
\item $I_1$ is an instance of 3-partition $b_1,...,b_{3n}$ with $B=\frac{1}{n}\sum_{i=1}^n b_i$
\item $I_2$ an instance of $Dec(2)$\\
We have $3n$ tasks: $T_1,...,T_{3n}$, independent with $\omega (T_i)=b_i$\\
We take $3n$ tasks $X_1,...,X_n$, $Y_1,...,Y_n$ and $Z_1,...,Z_n$, with $\forall i \in \{1,...,n\} \; \omega (X_1) = \omega (Y_i) = \omega (Z_i) = B_i$.\\
$X_i$ depends on $Y_{i-1}$ and $Z_{i-1}$, both depending from $X_{i-1}$.
\item $|I_2|$ is polynomial in $|I_1|$
\item Let us assume that $I_1$ has a solution $I_1,...,I_n$ such that $\sum_{j\in I_j} b_j=B$. On processor $P_1$, I execute $X_1,Y_1,...,X_n,Y_n$; on processor $P_2$, I execute $Z_i$ when $P_1$ executes $X_i$.\\
It works because $\omega (X_i)=\omega (Y_i) =\omega (Z_i) = B$ and because $\sum_{T\in I_i} \omega(T) = B$
\item Assume we have a solution to the scheduling problem $\Phi_1=X_1 \to Y_1 \to X_2 \to Y_2 \to ... \to X_n \to Y_n$. Then $\omega(\Phi_1)=2nB$, so there is no freedom on the execution time of these tasks. So $\sigma (X_i)=2(i-1)B$, and $\sigma (Y_i)=(2i-1)B$.
Besides, $\Phi_2=X_1 \to Z_1 \to X_2 \to Z_2 \to ... \to X_n \to Z_n$, with $\omega(\Phi_2)=2nB$ and $\sigma(z_i)=(2i-1)B$. Without loss of generality, we can assume that $P_1$ executes all the $X_i$'s and $Y_i$'s and $P_2$ executes all the $T_i$'s and the $Z_i$'s.

We have exactly $n$ disjoint intervals of size $B$ to execute the $T_i$'s on $P_2$.

Let $I_j=$ the subset of the $T_i$'s that are executed by $P_2$ while $P_1$ executes $X_j$.

This schedule is satisfying the bound $k=2nB \Rightarrow \bigcup_{i=1}^n I_i = \{ 1,...,3n\}, \; \sum_{j\in I_i} \omega(T_j)\leq 3$, which gives a solution to the partition.
\end{itemize}
\end{itemize}
\end{proof}

\subsection{List scheduling heuristics}
\paragraph{Historically:} set priorities to tasks, put the tasks in a list ordered by these priorities and greedily schedule the tasks.

\paragraph{Principle:} do not voluntarily let a process idle.

\begin{defi}
Let $G=(V,E,\omega)$ be a task system, let $\sigma$ be a schedule and let $v$ be a task ($v\in V$). The task $v$ is free at time $t$ if and only if the processing of $v$ has started but all its predecessors have completed.
\end{defi}

\begin{thm}
Let $G=(V,E,\omega)$ be a task system, and let $\sigma$ be any list schedule for $G$, then 
\[
\tag{Graham's bound}
MS(\sigma , p) \leq \left(2 - \frac{1}{p}\right) MS_{OPT}(p)
\]
\end{thm}

\begin{proof}

\begin{lemma}
There exists a dependence path $\Phi$ in $G$ such that $Idle \leq (p-1)\omega (\Phi )$ with $Idle=\sum $ Idle times.
\end{lemma}
\begin{proof}
Let $T_1$ be a task that completes last: $\sigma (T_1) + \omega (T_1) = MS(\sigma , p)$. Let $t_1$ be the last time before $\sigma (T_1)$ such that at least one processor was idle. $T_1$ could not be started at time $t_1$ because at least one task $T_1$ depends upon was still proceed at that time.

Let $T_2$ be one such task. Let $t_2$ be the last time before the start $\sigma ( T_2)$ of $T_2$ at which one processor was idle.

I end up with a path $\Phi : \; T_k \to T_{k-1} \to ... \to T_2 \to T_1$. I can only have idle times while a task of $\Phi$ is executed: $Idle \leq \underbrace{(p-1)}_{\substack{\text{execution of the}\\\text{tasks of $\Phi$}}}\omega (\Phi)$.
\end{proof}

\begin{align*}
pM(\sigma,p)& = Idle + \sum_{i=1}^n \omega (T_i) \leq (p-1) \omega ( \Phi ) + \sum_{i=1}^n \omega ( T_i )\\
MS(\sigma,p) & \leq \left(1-\frac{1}{p}\right) \omega (\Phi) + \sum_{i=1}^n (\omega (T_i))\frac{1}{p} \leq \left(1-\frac{1}{p}\right) MS_{OPT}(p) + MS_{OPT}(p) = \left(2-\frac{1}{p} \right) MS_{OPT}(p)
\end{align*}
\end{proof}

\begin{prop}
Let $MS_{list}(p)$ be the shortest possible makespan produced by a list scheduling algorithm.

Then the following bound is tight:
\[MS_{list} (p) \leq \left( 2 - \frac{1}{p}\right) MS_{OPT}(p)\]
\end{prop}

\begin{proof}
Let $2p+1$ task such that $T_1,...,T_{p}$ are independent, and $T_{p+1},...,T_{2p}$ depending from $T_p$; and $T_{2p+1}$ depending from the latter.

Consider a list schedule $\Rightarrow$ all entry task start at time 0. Let us say that task $T_i$ for $1\leq i\leq p-1$ is executed by $P_i$ during $[0,k(p-1)]$. Then $T_p$ is executed by $P_p$ in $[0,1]$. Wlog, $P_p$ executes $T_{p+1}$ during $[1, k+1]$, and wlog, $P_p$ executes $T_{p+i}$ during $[1+(i-1)k;k+i+1]$ for $i\geq 1$. (Sanity check: what would be the starting time of $T_{2p}$: $1+(p-1)k$)

Hence: processor $P_p$ executes $(p-1)$ of the tasks $T_{p+1},...,T_{2p}$ in the time interval $[1,1+(p-1)k]$. At time $(p-1)k$, all the processors $P_1,...,P_{p-1}$ where available. One of them executes the list of the tasks $T_{p+1},...,T_{2p}$ during $[(p-1)K,pK]$.


Then one of the $p$ processors starts task $T_{2p+1}$ at time $pk$ and completes at time $k(2p-1)$, so $MS_{list}(p)=k(2p-1)$. 
\end{proof}

\paragraph{Optimal}
\begin{itemize}
\item At time 0, $P_1$ executes $T_p$
\item From 1 to $1+k$, each processor executes a task among $T_{p+1},...,T_{2p}$
\item From $1+k$ through $1+k+k(p-1)=1+kp$ each processor executes a task among $T_1,...,T_{p-1}$ and $T_{2p-1}$
\end{itemize}


\paragraph{Performance ratio of list schedule:}
\[\frac{k(2p-1)}{1+kp}\underset{k \to + \infty}{\longrightarrow} \frac{2p-1}{p}=2-\frac{1}{p}\]


\subsection{Critical path scheduling}
\begin{defi}[Critical path]
The critical path of a task is its bottom-level, i.e. the lower bound on the remaining execution time of the graph from the state of the execution of the considered tasks.
\end{defi}

\paragraph{Example:}
$T_1$ has no dependencies, $T_2$ is required for $T_4, T_5, T_6$ and $T_7$, $T_8$ depends on $T_3$.

\begin{center}
\begin{tabular}{l | c | c |c|c|c|c|c|c}
task & $T_1$ & $T_2$ & $T_3$ & $T_4$ & $T_5$ & $T_6$ & $T_7$ & $T_8$ \\
\hline
weight & 3 & 2 & 1& 3 & 4 & 4 & 3 & 6\\
\hline
critical path & 3 & 6 & 7 & 3 & 4 & 4 & 3 & 6\\
\end{tabular}
\end{center}

With the heuristic of doing first the task with the maximum critical path, we end up with a schedule of 10. This is not optimal, as the optimal schedule weights only 9.

\paragraph{Lower bound on the makespan:}
\[ \left\lceil \frac{\sum_{i}\omega (T_i)}{p}\right\rceil = \left\lceil \frac{26}{3} \right\rceil = 9\]


\section{Taking communication into account}
Macro-dataflow model: let two tasks $T$ and $T'$ with $T \to T'$ ($\to$ meaning that some data produced by $T$ is used by $T'$):
\begin{itemize}
\item If $alloc(T)=alloc(T')$, then there is no communication cost
\item If $alloc(T)\neq alloc(T')$, then there is some communication cost $= c(T,T')$: does not depend on the choice of $alloc(T)$ and $alloc(T')$.
\end{itemize}

\paragraph{Assumptions (implicit ones}
\begin{itemize}
\item Complete graph
\item There are no contention between communications, i.e. the time of the communication is the same if there is one or $n$ simultaneous input communications for a unique processor ($\simeq$ infinite input bandwidth)
\end{itemize}

\paragraph{Scheduling and communications:} A communication DAG in a four-uplet $G=(V,E,\omega,c)$ with:
\begin{itemize}
\item $V$: tasks
\item $E$: set of dependencies between tasks
\item $\omega$: execution times of the tasks $\omega: V \to \mathbb{N}^*$
\item $c:$ communication costs: $c:E\to \mathbb{N}$
\end{itemize}


A schedule $\sigma$ must respect the dependences:
\[
\forall e \in E \quad e=(T,T') 
\begin{cases}
\sigma(T)+\omega(T)\leq \sigma(T') & $If $alloc(T)=alloc(T')\\
\sigma(T) + \omega(T)+c(T,T')\leq \sigma(T') & $otherwise$
\end{cases}\]

\section{Pb($\infty$) with communications}
%exemple here
Problem:
\begin{itemize}
\item Sequential execution: we pay the cost of each task
\item Each task on its own processor: we pay all the communications
\end{itemize}

\subsection{NP-completeness of Pb($\infty$)}
\paragraph{Decision problem}
Given a communication DAG $G=(V,E,\omega,c)$, and an execution bound $k$, is it possible to execute/schedule $G$ in a time no greater than $k$?

\begin{proof}[Reduction from 2-Partition]
$n$ positive integers $a_1,a_2,...,a_n$ with $\sum_{i=1}^n a_i = \alpha$. Is there a subset $I$ of $\{1,...,n\}$ such that $\sum_{i\in I} a_i = \sum_{i \notin I} a_i$?

We build an instance $I_2$ from the scheduling problem: tasks $T_1,...,T_n$ (weights $2a_i$) depends on $T_0$ (weight $A$); and task $T_{n+1}$ depends on $T_1,...,T_n$.

The $2n$ edges has the same communication cost: $C$, any integer in the interval $]\alpha - \min_{1\leq i ]leq n} 2a_i, \alpha[$. $\forall i \; a_i>0 \Rightarrow a_i \geq 1$ length of the open interval $\geq 2$, so $C$ exists. Let $K=2A+C+\alpha$. The size of $I_2$ is polynomial in the size of $I_1$.

If $I_1$ has a solution $I$, then $I_2$ has a solution of makespan $\leq k$ by taking on $P_0$ the task of $I$ and $T_0$, and on $P_1$ the tasks not in $I$ and $T_{n+1}$.

\begin{lemma}
$T_0$ and $T_{n+1}$ are not executed on the same processor.
\end{lemma}

\begin{proof}
By contradiction: $T_0$ and $T_{n+1}$ are executed on the same processor, say $P_0$. Can all tasks be executed on $P_0$? $M=\omega(T_0)+\sum_{i=1}^n \omega(T_i) + \omega(T_{n+1}=2A+2\alpha$.

By definition $\alpha > C$. $M=2A+2\alpha> 2A+\alpha+C=k \; \Rightarrow$ at least one task $T$ is not executed on $P_0$. As $T_0 \to T \to T_{n+1}$, we have to pay both communication costs.


\begin{align*}
M & \geq A + C + \omega(T) + C + A\\
& \geq 2A + 2C + \min_{1\leq i\leq n}2a_i\\
M & > 2A + 2C+(\alpha-C)=2A+C+\alpha=K
\end{align*}
Contradiction.
\end{proof}

Let $P_0$ execute $T_0$ and $P_1$ execute $T_{n+1}$.

\begin{lemma}
Each task is executed either by $P_0$ or $P_1$.
\end{lemma}
\begin{proof}
By contradiction. There is a task $T$ executed neither by $P_0$ nor $P_1$. But $T_0 \to T \to T_{n+1}$, the two communication take place, and $M\geq 2A+2C+\omega(T)>K$.
\end{proof}

\begin{itemize}
\item Each task is executed either on $P_0$ (like $T_0$) or $P_1$ (like $T_{n+1}$). Let $I$ be the set of indices of task among $T_0,...,T_n$ executed on $P_0$, let $I$ be the set of indices of task among $T_0,...,T_n$ executed on $P_1$.
$I\cup J=\{1,...,n\}$
\item Consider $I$:
\begin{align*}
K\geq M & \geq \omega(T_0) + \omega(I)+C+\omega(T_{n+1})\\
& = 2A + \omega(I)+C\\
\end{align*}

\item Consider $J$:
\begin{align*}
K & \geq \omega(T_0) + C + \omega(J) + \omega(T_{n+1})\\
K & \geq 2A + C + \omega (J)\\
K= &\; 2A+\alpha+C\\
\alpha & \geq \omega(I)\\
\alpha & \geq \omega(J)\\
\omega(I)+\omega(J)=2\alpha
\end{align*}
Hence, $\omega(I)=\alpha$ and $I$ defines a solution to $I_1$.
\end{itemize}
\end{proof}

In fact, it is NP-complete in the strong sense, even if all executions times are equal to 1 and all communication costs times are equal to 1 (UET-UCT).

\section{List heuristics for Pb($p$) with communications}
\paragraph{Question}
How to extend the notion of critical path?
\paragraph{Solution}
Compute critical paths assuming that all communication take place.

\subsection{Naive critical path}
List schedule with bottom-levels \emph{including} communications and processes are always considered on the same order.


\subsection{Modified critical path}
\paragraph{Principle}
Schedule the tasks of the processor that will enable to start it (= complete it) the earliest.


\subsection{Two-step clustering heuristics}
\paragraph{Clustering}
partitioning of the tasks

\paragraph{Given a clustering}
we compute bottom-levels and top level levels including a communication cost between two tasks if and only if they belong to 2 different clusters.

The task in a same cluster will be executed on the same processor.

Let $\mathcal{C}$ be a clustering.\\
$EPT(\mathcal{C})=$ estimated parallel time of $\mathcal{C}=\max tl(v) + bl(v)$ for $v\in V$.

I behave as if $I$ had an infinite number of processors.

\subsubsection*{Kim and Browne linear clustering}
\begin{itemize}
\item Take one longest dependence path in the graph; define a cluster from it; obtain a new cluster $\mathcal{C}'$; keep $\mathcal{C}'$ if and only if $EPT(\mathcal{C})\leq EPT(\mathcal{C})$
\item Iterate with the remainder of the graph
\end{itemize}


\subsubsection*{Sarkan's greedy clustering}
\begin{itemize}
\item Sort edges by non-increasing communication costs
\item For each edge on that order, merge the clusters containing the two extremities of the edges if this does not increase the $EPT$ (initialise $\mathcal{C}_0=\{\{T_1\},...,\{T_n\}\}$).
\end{itemize}


In the case of $\{T_3,T_4,T_5\}$ with $T_3 \to T_4$ and $T_3 \to T_5$, to compute the $EPT$, we need to sequentialize $T_4$ and $T_5$ (add virtual edge) because in the end they are going to be executed ont he same processor.

\subsubsection*{Dominant Sequence clustering} 
\begin{itemize}
\item Initially: all the edges are marked non-examined
\item While there remain non-examined tasks:
\begin{itemize}
\item Pick a dominate sequence (DS)
\item Zero an edge in the DS
\begin{itemize}
\item The edge that decrease the $EPT$ the most (expensive)
\item an edge of maximum weight
\item the first edge
\end{itemize}
We keep the clustering if $EPT$ does not increase
\end{itemize}
\end{itemize}

Once we have clusters, we need to decide on which processors to map the clusters (and still some scheduling problem to solve if 2 clusters are mapped on the same processor).


\end{document}