\section{Introduction}
\subsection*{Where do task graphs come from?}
Linear system $Ax=B$. $A$ is a lower triangular matrix ($b$ and $x$ are two vectors), $b$ is known. $A$ is a $n \times n$ matrix.

\begin{algorithm}[H]
\For{$i=1$ to $n$}{
	$T_{ii}:x_i\leftarrow \frac{b_i}{a_{ii}}$\\
	\For{$j=i+1$ to $n$}{
		$T_{ij}: b_j\leftarrow b_j-a_{j,i}.x_i$\\
	}
}
\end{algorithm}


Original algorithm is sequential.


Total ordering of the tasks:
\[T_{1,1} <_{seq} T_{1,2} <_{seq} T_{1,3} <_{seq} ... <_{seq} T_{1,n} <_{seq} T_{2,2} <_{seq} T_{2,3} <_{seq} ... <_{seq} T_{n,n}\]

Some parallelism: $T_{1,2}$ and $T_{1,3}$ are independent.

Condition for two tasks $T$ and $T'$ to be a dependence relation:
\begin{itemize}
\item both access the same variable and at least one access is a \emph{write} access.
\item $In(T)$: the set of variables read by task $T$
\item $Out(T)$: the set of variable written by task $T$
\end{itemize}

$T$ and $T'$ are not independent ($T \perp T'$)

\[T \perp T' \Leftrightarrow
\begin{cases}
In(T)\cap Out(T')\neq \emptyset & \text{or}\\ 
Out(T)\cap In(T') \neq \emptyset &\text{or}\\
Out(T)\cap Out(T') \neq \emptyset
\end{cases}
\]

We define a partial order $\prec$ by:\\
if $T\perp T'$ and $T<_{seq} T'$ then $T\prec T'$
\bigskip


$\prec$ is $(<_{seq}\cap \perp)^+$ (transitive closure)
\bigskip


Representation of dependencies with a graph:
\[G=(\underbrace{V}_{\text{set of tasks}},\underbrace{E}_{\text{oriented edges}})\]
With
\begin{align*}
e=(T,T')\in E & \Leftrightarrow T \prec T'\\
\big(\text{and } \nexists V \text{ s.t. } T\prec T' & \text{ and } V \prec T' \big)\\
\end{align*}

(we do not include transitive edges for the sake of readability)

\section{Scheduling task graph}
\begin{defi}
A task graph is a directed weighted graph $G=(V,E,\omega )$

Where:
\begin{itemize}
\item The set $V$ represent the tasks
\item The set $E$ of edges represents the dependences: $e=(u,v) \in E $ if and only if $u\prec v$.
\item A weight function $\omega : V \to \mathbb{N} $ gives the weight or execution time of a test.
\end{itemize}
\end{defi}

\begin{defi}
A schedule of a task graph $G(V,E,\omega)$ is a function
$\sigma : V\to \mathbb{N}$ such that if $e=(u,v) \in E$ then $\sigma (u) + \omega (v) \leq \sigma (v)$. $\sigma$: starting time of the tasks.
\end{defi}

If we have a limited number $p$ of processor (if $p<|v|$), $alloc$: $v\to \{1,...,p\}$ states on which processor a task is executed.

\[alloc(u) = alloc (v) \Leftrightarrow 
\begin{cases}
\sigma(u) + \omega(v) \leq \sigma(v) & \text{or}\\
\sigma(v) + \omega(v) \leq \sigma(u)
\end{cases}\]

\begin{thm}
Let $G$ be a task graph. There exists a schedule for $G$ if and only if $G$ does not contain any cycle.
\end{thm}

\begin{proof}
\begin{itemize}
\item There is a cycle $T_1 \to T_2 \to ... \to T_k \to T_1$, so  $T_1 \prec T_2 \prec ... \prec T_k \prec T_1$, and $T_1\prec T_1$, $\sigma(T_1) + \omega(T_1) \leq \sigma(T_1)$ impossible.
\item There is no cycle
\item Topological order
\end{itemize}
\begin{algorithm}
pick a task $T$ without predecessor\\
$\sigma(T) \leftarrow t$\\
$t\leftarrow t + \omega(T)$\\
remove $T$ from $G$\\
repeat\\
\end{algorithm}
\end{proof}

\begin{defi}[Makespan]
Task graph $G=(V,E,\omega )$
\begin{enumerate}
\item Let $\sigma$ be a schedule for $G$ using $p$ processors. The makespan of $G$ is the total execution time.
\[MS(\sigma, p) = \max_{v\in V} \{\sigma (v) + \omega (v) \} - \min_{v\in V}\{\sigma ( v)\}\]
Usually we assume that $\min_{v\in V}\{\sigma ( v)\}=0$

\item $Pb(p)$: Problem of finding a best scheduling (i.e. a schedule of minimum makespan) using $p$ processors. Let $MS_{opt}(p)$ be the makespan of an optimal scheduling using $p$ processors.

$\omega : V \to \mathbb{N}^*$. We extend it on paths $\Phi : T_1,T_2,...,T_k$ where $T_1 \prec T_2 \prec ... \prec T_k$ by $\omega ( \Phi) =\sum_{T_i \in \Phi} \omega (T_I)$
\end{enumerate}
\end{defi}

\begin{prop}
Let $G=(V,E,\omega)$ be a DAG (directed acyclic graph) and $\sigma$ a schedule for $G$ with $p$ processors.

Then $MS(\sigma, p)\geq \omega (\Phi)$ for all paths $\Phi$ in $G$.
\end{prop}

\begin{proof}
Let $\Phi$ be any path. 
\[\Phi=T_1 \prec T_2 \prec ... \prec T_k\]
\[T_i \prec T_{i+1} \Rightarrow \sigma (T_i) + \omega ( T_{i+1} \leq \sigma (T_{i+1})\]

Sum for $i$ to $d-1$:
\[\sigma (T_1) + \sum_{i=1}^{k-1} \omega(T_i) \leq \sigma(T_k) \]
\[MS(\sigma, p) \geq \sigma (T_k) + \omega (T_1) \geq \omega (\Phi)\]
\end{proof}


\begin{defi}
$G=(V,E,\omega)$, $\sigma$ on $p$ procc.
\begin{enumerate}
\item Speedup ratio:
\[s(\sigma, p)=\frac{T_{seq}}{MS(\sigma,p)} = \frac{\sum_{v\in V} \omega (v)}{MS(\sigma,p)}\]
\item Efficiency:
\[e(\sigma, p)=\frac{s(\sigma, p)}{p} = \frac{\sum_{v\in V} \omega (v)}{p.MS(\sigma,p)}\]
\end{enumerate}
\end{defi}


\begin{thm}
Let $G=(V,E,\omega)$ be a task graph. Then for any schedule $G$ on $p$ processors, $0\leq e(\sigma, p) \leq 1$
\end{thm}

\begin{proof}
The area of a rectangle on the scheduling timetable is $p.MS(\sigma, p)=Idle+work=Idle+seq$ and $l=Idle + \frac{Seq}{p.MS(\sigma,p)}=Idle+e(\sigma,p)$
\end{proof}

\begin{thm}
Let $G=(V,E,\omega)$ a task graph.
\[seq=MS_{opt}(1)\geq ... \geq MS_{opt}(n)\]
\end{thm}

Let $MS'(p)$ be the minimum makespan of all schedule that use exactly $p$ processors. Then for all $1\leq p \leq |V|, \; MS'(p)=MS_{opt}(p)$, because there is no communication (implicit hypothesis).



\section{Solving problems ($\infty$)}
We have an unbounded number of resources ($p\geq |V|$). Let $G=(V,E,\omega)$ be a task system.

\begin{defi}
\begin{enumerate}
\item 
$\forall v \in V$:
\begin{itemize}
\item $Pred (v)=$ set of immediate predecessors of a task $v$.
\item $Succ(v)=$ set of immediate successors.
\end{itemize}

\item 

\begin{itemize}
\item $v$ is an entry task if $Pred(v)=\emptyset$
\item $v$ is an exit task if $Succ(v)=\emptyset$
\end{itemize}

\item The top level $tl(v)$ is the maximum weight of a path from an entry task to the task $v$, excluding the weight of $v$.\\
$tl(v)$: lower bound on the time elapsed in any schedule between the start of the execution of $G$ and the start of $v$.
\[tl(v) = \max_{ v' \in Pred(v)} \{ tl(v') \}\]

\item The bottom level $bt(v)$ is the maximum weight of a path from the task $v$ (included) to an exit task (included).\\
$bt(v):$ lower bound on the execution time on any schedule once the execution of $v$ has started.
\[bl(v) = \max_{v' \in Succ(v)} \{ bl(v') \} + \omega(v')\]

\end{enumerate}

Both can be computed through a traversal of $G$ ($O(|V|+|E|)$)
\end{defi}

\begin{thm}
Let $G=(V,E,\omega)$ be a task system. We define $\sigma_{free}$ as follow:
\begin{itemize}
\item $\sigma_{free}=tl(v) \quad \forall v \in V$
\item $\sigma_{free}$ is an optimal schedule for $G$
\end{itemize}
\end{thm}
\begin{proof}
This is a schedule (dependences are satisfied by definition).

We show by induction (through a topological sort) that each task starts as soon, as possible.

ASAP schedule.
\end{proof}

\paragraph{Another optimal schedule:} as late as possible schedule: $\sigma_{ALAP}(v)=MS_{OPT}(\infty) - bt(v)$

\section{Solving Pb(p)}
\subsection{NP-completeness of Pb(p)}

\paragraph{(2-)partition:} given a set of $n$ (strictly) positive integers $a_1,...,a_n$, is there a subset $I$ of $\{1,...,n\}$ such that $\sum_{i\in I} a_i=\sum_{i\notin I}a_i$

NP-hard (weak sense, there is a pseudo polynomial algorithm to solve it)


\paragraph{3-partition:} we have a set of $3n$ (strictly) positive integers $b_1,...,b_n$. Let $B$ be such that $\sum_{i=1}^{3n}b_i=nB$. We assume that for each $i\leftarrow \{ 1,...,3n\}$, $\frac{b}{4}\leq b_i \leq \frac{B}{2}$. Can we partition the $3n$ $b_i$'s in $n$ subsets $I_1, ..., I_n$ such that $\sum_{i\in I} b_i = B$?

\begin{defi}
Let $Dec(p)$ the problem; let $G=(V,E,\omega)$ be a tsk system, let $p$ be the number of processors, and let $k\in \mathbb{N}^*$, is there a schedule for $G$ using at most $p$ processors, whose makespan is no greater than $k$?\\
If $E=\emptyset$ (no dependencies) we denote the problem by $Indep(p)$.
\end{defi}  

\begin{thm}
\begin{itemize}
\item $Indep(2)$ is NP-complete but can be solved in pseudo-polynomial time
\item $Indep(p)$ is NP-complete in the strong sense
\item $Dec(2)$ is NP-complete in the strong sense
\end{itemize}
\end{thm}

\begin{proof}
\begin{itemize}
\item $Indep(2)$:
\begin{itemize}
\item Problem belongs to NP, for each proc, check that the sum of the weights of the tasks associated to it does not exceed $k$.
\item Reduction from 2-partition:\\
$k=\frac{1}{2} \sum_{i=1}^{n}a_i$, we have an identical problem.
\end{itemize}
\item $Indep(p)$:\\
Reduction from 3-partition: $b_1,...,b_{3n}$ of 3-partition, we take $p=n$ and $k=B$.

\item $Dec(2)$:\\
Reduction from 3-partition
\begin{itemize}
\item $I_1$ is an instance of 3-partition $b_1,...,b_{3n}$ with $B=\frac{1}{n}\sum_{i=1}^n b_i$
\item $I_2$ an instance of $Dec(2)$\\
We have $3n$ tasks: $T_1,...,T_{3n}$, independent with $\omega (T_i)=b_i$\\
We take $3n$ tasks $X_1,...,X_n$, $Y_1,...,Y_n$ and $Z_1,...,Z_n$, with $\forall i \in \{1,...,n\} \; \omega (X_1) = \omega (Y_i) = \omega (Z_i) = B_i$.\\
$X_i$ depends on $Y_{i-1}$ and $Z_{i-1}$, both depending from $X_{i-1}$.
\item $|I_2|$ is polynomial in $|I_1|$
\item Let us assume that $I_1$ has a solution $I_1,...,I_n$ such that $\sum_{j\in I_j} b_j=B$. On processor $P_1$, I execute $X_1,Y_1,...,X_n,Y_n$; on processor $P_2$, I execute $Z_i$ when $P_1$ executes $X_i$.\\
It works because $\omega (X_i)=\omega (Y_i) =\omega (Z_i) = B$ and because $\sum_{T\in I_i} \omega(T) = B$
\item Assume we have a solution to the scheduling problem $\Phi_1=X_1 \to Y_1 \to X_2 \to Y_2 \to ... \to X_n \to Y_n$. Then $\omega(\Phi_1)=2nB$, so there is no freedom on the execution time of these tasks. So $\sigma (X_i)=2(i-1)B$, and $\sigma (Y_i)=(2i-1)B$.
Besides, $\Phi_2=X_1 \to Z_1 \to X_2 \to Z_2 \to ... \to X_n \to Z_n$, with $\omega(\Phi_2)=2nB$ and $\sigma(z_i)=(2i-1)B$. Without loss of generality, we can assume that $P_1$ executes all the $X_i$'s and $Y_i$'s and $P_2$ executes all the $T_i$'s and the $Z_i$'s.

We have exactly $n$ disjoint intervals of size $B$ to execute the $T_i$'s on $P_2$.

Let $I_j=$ the subset of the $T_i$'s that are executed by $P_2$ while $P_1$ executes $X_j$.

This schedule is satisfying the bound $k=2nB \Rightarrow \bigcup_{i=1}^n I_i = \{ 1,...,3n\}, \; \sum_{j\in I_i} \omega(T_j)\leq 3$, which gives a solution to the partition.
\end{itemize}
\end{itemize}
\end{proof}

\subsection{List scheduling heuristics}
\paragraph{Historically:} set priorities to tasks, put the tasks in a list ordered by these priorities and greedily schedule the tasks.

\paragraph{Principle:} do not voluntarily let a process idle.

\begin{defi}
Let $G=(V,E,\omega)$ be a task system, let $\sigma$ be a schedule and let $v$ be a task ($v\in V$). The task $v$ is free at time $t$ if and only if the processing of $v$ has started but all its predecessors have completed.
\end{defi}

\begin{thm}
Let $G=(V,E,\omega)$ be a task system, and let $\sigma$ be any list schedule for $G$, then 
\[
\tag{Graham's bound}
MS(\sigma , p) \leq \left(2 - \frac{1}{p}\right) MS_{OPT}(p)
\]
\end{thm}

\begin{proof}

\begin{lemma}
There exists a dependence path $\Phi$ in $G$ such that $Idle \leq (p-1)\omega (\Phi )$ with $Idle=\sum $ Idle times.
\end{lemma}
\begin{proof}
Let $T_1$ be a task that completes last: $\sigma (T_1) + \omega (T_1) = MS(\sigma , p)$. Let $t_1$ be the last time before $\sigma (T_1)$ such that at least one processor was idle. $T_1$ could not be started at time $t_1$ because at least one task $T_1$ depends upon was still proceed at that time.

Let $T_2$ be one such task. Let $t_2$ be the last time before the start $\sigma ( T_2)$ of $T_2$ at which one processor was idle.

I end up with a path $\Phi : \; T_k \to T_{k-1} \to ... \to T_2 \to T_1$. I can only have idle times while a task of $\Phi$ is executed: $Idle \leq \underbrace{(p-1)}_{\substack{\text{execution of the}\\\text{tasks of $\Phi$}}}\omega (\Phi)$.
\end{proof}

\begin{align*}
pM(\sigma,p)& = Idle + \sum_{i=1}^n \omega (T_i) \leq (p-1) \omega ( \Phi ) + \sum_{i=1}^n \omega ( T_i )\\
MS(\sigma,p) & \leq \left(1-\frac{1}{p}\right) \omega (\Phi) + \sum_{i=1}^n (\omega (T_i))\frac{1}{p} \leq \left(1-\frac{1}{p}\right) MS_{OPT}(p) + MS_{OPT}(p) = \left(2-\frac{1}{p} \right) MS_{OPT}(p)
\end{align*}
\end{proof}

\begin{prop}
Let $MS_{list}(p)$ be the shortest possible makespan produced by a list scheduling algorithm.

Then the following bound is tight:
\[MS_{list} (p) \leq \left( 2 - \frac{1}{p}\right) MS_{OPT}(p)\]
\end{prop}

\begin{proof}
Let $2p+1$ task such that $T_1,...,T_{p}$ are independent, and $T_{p+1},...,T_{2p}$ depending from $T_p$; and $T_{2p+1}$ depending from the latter.

Consider a list schedule $\Rightarrow$ all entry task start at time 0. Let us say that task $T_i$ for $1\leq i\leq p-1$ is executed by $P_i$ during $[0,k(p-1)]$. Then $T_p$ is executed by $P_p$ in $[0,1]$. Wlog, $P_p$ executes $T_{p+1}$ during $[1, k+1]$, and wlog, $P_p$ executes $T_{p+i}$ during $[1+(i-1)k;k+i+1]$ for $i\geq 1$. (Sanity check: what would be the starting time of $T_{2p}$: $1+(p-1)k$)

Hence: processor $P_p$ executes $(p-1)$ of the tasks $T_{p+1},...,T_{2p}$ in the time interval $[1,1+(p-1)k]$. At time $(p-1)k$, all the processors $P_1,...,P_{p-1}$ where available. One of them executes the list of the tasks $T_{p+1},...,T_{2p}$ during $[(p-1)K,pK]$.


Then one of the $p$ processors starts task $T_{2p+1}$ at time $pk$ and completes at time $k(2p-1)$, so $MS_{list}(p)=k(2p-1)$. 
\end{proof}

\paragraph{Optimal}
\begin{itemize}
\item At time 0, $P_1$ executes $T_p$
\item From 1 to $1+k$, each processor executes a task among $T_{p+1},...,T_{2p}$
\item From $1+k$ through $1+k+k(p-1)=1+kp$ each processor executes a task among $T_1,...,T_{p-1}$ and $T_{2p-1}$
\end{itemize}


\paragraph{Performance ratio of list schedule:}
\[\frac{k(2p-1)}{1+kp}\underset{k \to + \infty}{\longrightarrow} \frac{2p-1}{p}=2-\frac{1}{p}\]


\subsection{Critical path scheduling}
\begin{defi}[Critical path]
The critical path of a task is its bottom-level, i.e. the lower bound on the remaining execution time of the graph from the state of the execution of the considered tasks.
\end{defi}

\paragraph{Example:}
$T_1$ has no dependencies, $T_2$ is required for $T_4, T_5, T_6$ and $T_7$, $T_8$ depends on $T_3$.

\begin{center}
\begin{tabular}{l | c | c |c|c|c|c|c|c}
task & $T_1$ & $T_2$ & $T_3$ & $T_4$ & $T_5$ & $T_6$ & $T_7$ & $T_8$ \\
\hline
weight & 3 & 2 & 1& 3 & 4 & 4 & 3 & 6\\
\hline
critical path & 3 & 6 & 7 & 3 & 4 & 4 & 3 & 6\\
\end{tabular}
\end{center}

With the heuristic of doing first the task with the maximum critical path, we end up with a schedule of 10. This is not optimal, as the optimal schedule weights only 9.

\paragraph{Lower bound on the makespan:}
\[ \left\lceil \frac{\sum_{i}\omega (T_i)}{p}\right\rceil = \left\lceil \frac{26}{3} \right\rceil = 9\]


\section{Taking communication into account}
Macro-dataflow model: let two tasks $T$ and $T'$ with $T \to T'$ ($\to$ meaning that some data produced by $T$ is used by $T'$):
\begin{itemize}
\item If $alloc(T)=alloc(T')$, then there is no communication cost
\item If $alloc(T)\neq alloc(T')$, then there is some communication cost $= c(T,T')$: does not depend on the choice of $alloc(T)$ and $alloc(T')$.
\end{itemize}

\paragraph{Assumptions (implicit ones)}
\begin{itemize}
\item Complete graph
\item There are no contention between communications, i.e. the time of the communication is the same if there is 1 or $n$ simultaneous input communications for a unique processor ($\simeq$ infinite input bandwidth)
\end{itemize}

\paragraph{Scheduling and communications:} A communication DAG in a four-uplet $G=(V,E,\omega,c)$ with:
\begin{itemize}
\item $V$: tasks
\item $E$: set of dependencies between tasks
\item $\omega$: execution times of the tasks $\omega: V \to \mathbb{N}^*$
\item $c:$ communication costs: $c:E\to \mathbb{N}$
\end{itemize}


A schedule $\sigma$ must respect the dependences:
\[
\forall e \in E \quad e=(T,T') 
\begin{cases}
\sigma(T)+\omega(T)\leq \sigma(T') & $If $alloc(T)=alloc(T')\\
\sigma(T) + \omega(T)+c(T,T')\leq \sigma(T') & $otherwise$
\end{cases}\]

\section{Pb($\infty$) with communications}
%exemple here
Problem:
\begin{itemize}
\item Sequential execution: we pay the cost of each task
\item Each task on its own processor: we pay all the communications
\end{itemize}

\subsection{NP-completeness of Pb($\infty$)}
\paragraph{Decision problem}
Given a communication DAG $G=(V,E,\omega,c)$, and an execution bound $k$, is it possible to execute/schedule $G$ in a time no greater than $k$?

\begin{proof}[Reduction from 2-Partition]
$n$ positive integers $a_1,a_2,...,a_n$ with $\sum_{i=1}^n a_i = \alpha$. Is there a subset $I$ of $\{1,...,n\}$ such that $\sum_{i\in I} a_i = \sum_{i \notin I} a_i$?

We build an instance $I_2$ from the scheduling problem: tasks $T_1,...,T_n$ (weights $2a_i$) depends on $T_0$ (weight $A$); and task $T_{n+1}$ depends on $T_1,...,T_n$.

The $2n$ edges has the same communication cost: $C$, any integer in the interval $]\alpha - \min_{1\leq i ]leq n} 2a_i, \alpha[$. $\forall i \; a_i>0 \Rightarrow a_i \geq 1$ length of the open interval $\geq 2$, so $C$ exists. Let $K=2A+C+\alpha$. The size of $I_2$ is polynomial in the size of $I_1$.

If $I_1$ has a solution $I$, then $I_2$ has a solution of makespan $\leq k$ by taking on $P_0$ the task of $I$ and $T_0$, and on $P_1$ the tasks not in $I$ and $T_{n+1}$.

\begin{lemma}
$T_0$ and $T_{n+1}$ are not executed on the same processor.
\end{lemma}

\begin{proof}
By contradiction: $T_0$ and $T_{n+1}$ are executed on the same processor, say $P_0$. Can all tasks be executed on $P_0$? $M=\omega(T_0)+\sum_{i=1}^n \omega(T_i) + \omega(T_{n+1}=2A+2\alpha$.

By definition $\alpha > C$. $M=2A+2\alpha> 2A+\alpha+C=k \; \Rightarrow$ at least one task $T$ is not executed on $P_0$. As $T_0 \to T \to T_{n+1}$, we have to pay both communication costs.


\begin{align*}
M & \geq A + C + \omega(T) + C + A\\
& \geq 2A + 2C + \min_{1\leq i\leq n}2a_i\\
M & > 2A + 2C+(\alpha-C)=2A+C+\alpha=K
\end{align*}
Contradiction.
\end{proof}

Let $P_0$ execute $T_0$ and $P_1$ execute $T_{n+1}$.

\begin{lemma}
Each task is executed either by $P_0$ or $P_1$.
\end{lemma}
\begin{proof}
By contradiction. There is a task $T$ executed neither by $P_0$ nor $P_1$. But $T_0 \to T \to T_{n+1}$, the two communication take place, and $M\geq 2A+2C+\omega(T)>K$.
\end{proof}

\begin{itemize}
\item Each task is executed either on $P_0$ (like $T_0$) or $P_1$ (like $T_{n+1}$). Let $I$ be the set of indices of task among $T_0,...,T_n$ executed on $P_0$, let $I$ be the set of indices of task among $T_0,...,T_n$ executed on $P_1$.
$I\cup J=\{1,...,n\}$
\item Consider $I$:
\begin{align*}
K\geq M & \geq \omega(T_0) + \omega(I)+C+\omega(T_{n+1})\\
& = 2A + \omega(I)+C\\
\end{align*}

\item Consider $J$:
\begin{align*}
K & \geq \omega(T_0) + C + \omega(J) + \omega(T_{n+1})\\
K & \geq 2A + C + \omega (J)\\
K= &\; 2A+\alpha+C\\
\alpha & \geq \omega(I)\\
\alpha & \geq \omega(J)\\
\omega(I)+\omega(J)&=2\alpha
\end{align*}
Hence, $\omega(I)=\alpha$ and $I$ defines a solution to $I_1$.
\end{itemize}
\end{proof}

In fact, it is NP-complete in the strong sense, even if all executions times are equal to 1 and all communication costs times are equal to 1 (UET-UCT).

\section{List heuristics for Pb($p$) with communications}
\paragraph{Question}
How to extend the notion of critical path?
\paragraph{Solution}
Compute critical paths assuming that all communication take place.

\subsection{Naive critical path}
List schedule with bottom-levels \emph{including} communications and processes are always considered on the same order.


\subsection{Modified critical path}
\paragraph{Principle}
Schedule the tasks of the processor that will enable to start it (= complete it) the earliest.


\subsection{Two-step clustering heuristics}
\paragraph{Clustering}
partitioning of the tasks

\paragraph{Given a clustering}
we compute bottom-levels and top level levels including a communication cost between two tasks if and only if they belong to 2 different clusters.

The task in a same cluster will be executed on the same processor.

Let $\mathcal{C}$ be a clustering.\\
$EPT(\mathcal{C})=$ estimated parallel time of $\mathcal{C}=\max_{v\in V} (tl(v) + bl(v))$.

\paragraph{Recall} (case of a clustering)
\begin{itemize}
\item $tl(u)=\max (max_{in},max_{out})$ with
\begin{align*}
max_{in}&=\max_{\substack{v\in Pred(v)\\\mathcal{C}(u)=\mathcal{C}(v)}} \{ tl(v)+\omega (v) \}\\
max_{out}&=\max_{\substack{v\in Pred(v)\\\mathcal{C}(u)\neq\mathcal{C}(v)}} \{ tl(v)+\omega (v) + C(u,v) \}
\end{align*}

\item $bl(u)=\max (max_{in},max_{out}) + \omega(u)$ with
\begin{align*}
max_{in}&=\max_{\substack{v\in Succ(v)\\\mathcal{C}(u)=\mathcal{C}(v)}} \{ bl(v) \}\\
max_{out}&=\max_{\substack{v\in Succ(v)\\\mathcal{C}(u)\neq\mathcal{C}(v)}} \{ bl(v) + C(u,v) \}
\end{align*}
\end{itemize}


\paragraph{WARNING (!!!!!):} In the case of $\{T_3,T_4,T_5\}$ with $T_3 \to T_4$ and $T_3 \to T_5$, to compute the $EPT$, we need to sequentialize $T_4$ and $T_5$ (add virtual edge) because in the end they are going to be executed on the same processor.

We behave as we have an infinite number of processors.

\subsubsection*{Kim and Browne linear clustering}
\begin{itemize}
\item Take one longest dependence path in the graph; define a cluster from it; obtain a new cluster $\mathcal{C}'$; keep $\mathcal{C}'$ if and only if $EPT(\mathcal{C'})\leq EPT(\mathcal{C})$
\item Iterate with the remainder of the graph
\end{itemize}


\subsubsection*{Sarkar's greedy clustering}
\begin{itemize}
\item Sort edges by non-increasing communication costs
\item For each edge on that order, merge the clusters containing the two extremities of the edges if this does not increase the $EPT$ (initialise $\mathcal{C}_0=\{\{T_1\},...,\{T_n\}\}$).
\end{itemize}

\subsubsection*{Dominant Sequence clustering} 
\begin{itemize}
\item Initially: all the edges are marked non-examined
\item While there remain non-examined tasks:
\begin{itemize}
\item Pick a dominate sequence (DS)
\item Zero an edge in the DS
\begin{itemize}
\item The edge that decrease the $EPT$ the most (expensive)
\item an edge of maximum weight
\item the first edge
\end{itemize}
We keep the clustering if $EPT$ does not increase
\end{itemize}
\end{itemize}

Once we have clusters, we need to decide on which processors to map the clusters (and still some scheduling problem to solve if 2 clusters are mapped on the same processor).