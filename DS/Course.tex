\documentclass{article}

\usepackage[utf8]{inputenc}
\usepackage[english]{babel}
 
\usepackage{caption}
%\usepackage{pgfplots}
\usepackage{listings}
\usepackage{graphicx}
\usepackage{footnote}
\usepackage{amsmath}
\usepackage{amsthm}
\usepackage{graphicx}
\usepackage{url}
\usepackage{amssymb}
\usepackage{mathrsfs}
\usepackage{multirow}
\usepackage{amsfonts}
\usepackage[boxed,linesnumbered]{algorithm2e}
\usepackage{qcircuit}
\usepackage{enumerate}
\usepackage{enumitem}

\newtheorem{thm}{Theorem}
\newtheorem{prop}{Propriety}
\newtheorem{lemma}{Lemma}
\newtheorem{defi}{Definition}
\newtheorem{coro}{Corollary}



\setlength{\oddsidemargin}{0pt}
% Marge gauche sur pages impaires
\setlength{\evensidemargin}{0pt}
% Marge gauche sur pages paires
\setlength{\textwidth}{470pt}
% Largeur de la zone de texte 
\setlength{\topmargin}{0pt}
% Pas de marge en haut
\setlength{\headheight}{13pt}
% Haut de page
\setlength{\headsep}{10pt}
% Entre le haut de page et le texte
\setlength{\footskip}{40pt}
% Bas de page + séparation
\setlength{\textheight}{630pt}
% Hauteur de la zone de texte
\setlength{\itemsep}{0pt}

\title{Distributed Systems}
\author{Eddy Caron\footnote{hadriencroubois.com}\\
\small ENS de Lyon}
\date{}

\newcommand{\note}{\medskip\noindent\underline}
\newcommand\tok[1]{$\langle$#1$\rangle$}
\SetKw{Var}{Var:\\}
\SetKwBlock{Label}{label}{end}
\SetKwProg{Fn}{Function}{}{}

\begin{document}

\maketitle
\tableofcontents
\newpage


\paragraph{Final grade} 2/3 Final exam + 1/3 mid-term exam + 1/3 (project + partial)


\bigskip
We will use the programming language \emph{Erlang} (1987 $\rightsquigarrow$ 1998)\\
\textbullet \; Concurrent, real time, distributed\\
$\Rightarrow$ Use BEAM
\bigskip

Variables can be integers, float, PID, functions, tuples, maps, ... and atoms (specific to Erlang).

There are built-in functions for message passing

Lists are not strongly typed, tuples are like python's

In Erlang, there is no overwriting of the variables (cannot affect them a new value).

\section{Algorithm for Distributed System}
\subsection{Modelization}
\begin{defi}[Transition relation]
Let $S=(\mathcal{C},\to,\mathcal{I})$ be a transition system. An execution of $S$ is a \emph{maximal sequence} $E=(\gamma_0,\gamma_1,...,\gamma_n)$
%TODO
\end{defi}


\begin{defi}[Terminal configuration]
A \emph{terminal configuration} is a configuration $\gamma$ for which there is no $\delta$ such that $\gamma \to \delta$. Note that a sequence $E=(\gamma_0,\gamma_1,\gamma_2,...)$
%TODO
\end{defi}

$\to$ this notion is very tricky in parallel programming, as it is difficult to know whether all the executions are finished or not. 

\begin{defi}
A configuration $\delta$ is \emph{reachable} from $\delta$ (notation $\delta \rightsquigarrow \gamma$), if there exist a sequence $\gamma=\gamma_0,\gamma_1,\gamma_2,...,\gamma_k$ with $\gamma_i \to \gamma_{i+1}$ for all $0\leq i <k$. Configuration $\delta$ is reachable if it is reachable from an initial configuration.
\end{defi}


\begin{defi}[System with Asynchronous Message Passing]
Cf APPD course
\end{defi}


\subsection{Fairness}
\begin{defi}
An execution is \emph{weakly fair} if no event is applicable in infinitely many consecutive configurations without occurring in the execution
\end{defi}


\begin{defi}
An execution is \emph{strongly fair} if no event is applicable in infinitely many configuration without occurring in the execution.
\end{defi}


\subsection{Network topology}
Many topology exists, including :
\begin{itemize}[noitemsep]
\item Ring
\item Tree
\item Hypercube
\item Star
\item Clique
\end{itemize}

You can "change the topology" of your network to adapt and algorithm, for example taking a star sub-graph of a clique.


\subsection{How to write a distributed algorithm?}
\begin{itemize}[noitemsep]
\item A specific code for a specific node (or family of node)
\begin{itemize}[noitemsep]
\item The sender code and the receiver code
\item The initial code and the non-initial code
\end{itemize}
\item One code for all
\begin{itemize}[noitemsep]
\item The same code is executed on each node
\item A requirement can switch to the right code for a node
\end{itemize}
\end{itemize}

Use label to separated the code between initiators, non-initiator or reception of a specific message (for example stopping the execution).

\paragraph{Warning}
The execution flow is not clear: there is no assumption about the label selected for the execution (it is randomly chosen), so the algorithm must run for all the chosen labels for al the processors in the current state.

It is useful to try to find an incorrect workflow to test whether the algorithm is correct.

No assumptions must be made on the execution time (especially linking synchronization with the size of the code)!


\section{Communication protocols}
\subsection{Sliding window communication protocol}
\paragraph{Example} Write an algorithm to exchange informations between both process

\begin{algorithm}[h]
\Var
data$_{in}$=N\;
data$_{recv}=-1$\;
is\_sent = -1\;
\Label ($1$\{is\_sent==1\}){
send $\langle$msg,data$_{in}\rangle$
is\_sent=0\;
}
\Label ($2$\{receive $\langle$msg,i$\rangle$\}){
data$_{recv}$=i\;
}
\end{algorithm}

\begin{algorithm}
\tcp{Now, we want to send an array of data:}
\Var
data$_{in}$=N\;
data$_{recv}=-1$\;
j=0\;
i=0\;
\Label ($1$\{is\_sent$[i]$==1\}){
send $\langle$msg,data$_{in}[i]\rangle$\;
is\_sent[i++]=0\;
}
\Label (${2}$\{receive $\langle$msg,v$\rangle$\}){
data$_{recv}[j++]$=v\;
}
\end{algorithm}

\begin{algorithm}
\tcp{Now, we had a procedure \url{lost} that erase messages:}
\Label (Lost$_n$=\{$\langle$msg$\rangle\}$){
kill(msg)\\
}
\end{algorithm}

\begin{algorithm}
\tcp{We can do it with using 0 acknowledgements, by using the data you need to send as an acknowledgement:
}
\Var
$s_p$ is the number of words received by $p$ from $q$ (without missing ones)\;
$l_p, l_q$ is the number of ahead values $\to$ size of the sliding window\;
\Label($S_p$:\{$a_p\leq i < s_p+l_p$\}){
	send \tok{pack,$in_p[i],i$} to $q$\;
}
\Label($R_p$:\{\tok{pack,$w,i$}$\in Q_p$\}){
	receive \tok{pack, $w,i$}\;
	\eIf{$out_p[i]==-1$}{
		$out_p[i]=w$\;
		$a_p = \max \{a_p, i-l_q + 1\}$\;
		$s_p = \min_{j} \{ out_p [j] = -1\}$\;
	}{
		ignore\;
	}
}
\Label($L_p$:\{\tok{$pack,w,i$}$\in Q_p$){
	$Q_p=Q_p\setminus$\tok{$pack,w,i$}\;
}
\end{algorithm}

The number of exchanged message can be a good measure of the performance of the algorithm.
\clearpage

\subsection{Timer-based protocol}

Create channel, and see whether it is opened or not (See slides for details).


\section{Wave algorithm}
\subsection{Wave algorithm}
We need to solve broadcast, global synchronisation, trigger events, parallel computing, data-parallelism.

For that, we use message passing.

Requirements:
\begin{itemize}[noitemsep]
\item Termination
\item Decision
\item Dependence
\end{itemize}
\bigskip

List of aspects:
\begin{itemize}[noitemsep]
\item Centralized: one initiator
\item Decentralized: $n$ initiators
\item Network: topology, undirected links, connected, \emph{Asynchronous} on \emph{Synchronous}
\end{itemize}
\bigskip

Initial knowledge:
\begin{itemize}[noitemsep]
\item Name
\item Name of its neighbour
\end{itemize}

Most of these algorithm send ``empty messages" (simple token).
\bigskip

The ring algorithm is a wave algorithm.
\newpage

\subsection{The tree algorithm}
On a tree:
\begin{itemize}[noitemsep]
\item All leaves initiate the algorithm
\item Each process sends exactly one message
\end{itemize}

\begin{algorithm}[h]
$rec_p\leftarrow 0$ \tcp{\# of received message}
$father_p \leftarrow$ undefined\;
\Begin(Initiator){
	\For{all $q \in Neighbourhood(p)$}{
		send \tok{tok} to $q$\;
	}
	\While{ $rec_p < \# Neighbourhood(p)$}{
		receive \tok{tok}\;
		$rec_p \leftarrow rec_p + 1$\;
	}
	decide\;
}
\Begin(Non-initiators){
	receive \tok{tok} from some neighbour $q$\;
	$father_p \leftarrow q$\;
	$rec_p \leftarrow rec_p +1$\;
	\For{all $q\in Neighbourhood(p) \smallsetminus \{father_p\}$}{
		send \tok{tok} to $q$\;
	}
	\While{$rec_p < \# Neighbourhood(p)$}{
		receive \tok{tok}\;
		$rec_p\leftarrow rec_p + 1$\;
	}
send \tok{tok} to $father_p$
}
\end{algorithm}

\newpage
\subsection{The echo algorithm}

\begin{algorithm}
$rec_p\leftarrow 0$ \tcp{\# of received message}
$father_p \leftarrow$ undefined\;
\Begin(Initiator){
	\For{all $q \in Neighbourhood(p)$}{
		send \tok{tok} to $q$\;
	}
	\While{ $rec_p < \# Neighbourhood(p)$}{
		receive\tok{tok}\;
		$rec_p \leftarrow rec_p + 1$\;
	}
	decide\;
}
\Begin(Non-initiators){
	receive \tok{tok} from some neighbour $q$\;
	$father_p \leftarrow q$\;
	$rec_p \leftarrow rec_p +1$\;
	\For{all $q\in Neighbourhood(p) \smallsetminus \{father_p\}$}{
		send \tok{tok} to $q$\;
	}
	\While{$rec_p < \# Neighbourhood(p)$}{
		receive \tok{tok}\;
		$rec_p\leftarrow rec_p + 1$\;
	}
send \tok{tok} to $father_p$\;
}
\end{algorithm}

\newpage
\subsection{The Polling Algorithm}
Cf slides


\subsection{The Traversal algorithm}
They are a subset of wave algorithms. All events are totally ordered by the causality relation, and the last event occurs in the same process as the first event.

\begin{defi}[f-traversal algorithm]
An algorithm is a f-traversal algorithm for a cass of network if:
\begin{itemize}[noitemsep]
\item It is a traversal algorithm
\item In each computation, 
%TODO
\end{itemize}
\end{defi}

\paragraph{Example}
The ring algorithm is a traversal algorithm.

\subsection{Tarry's Graph Traversal Algorithm}
We start from a connected graph, and apply two rules:
\begin{itemize}[noitemsep]
\item A process never use the same communication channel twice 
\item A process will send to its father a message only if there is no other choice in respect to the first rule
\end{itemize}

\begin{algorithm}
%TODO
\tcp{TODO}
\end{algorithm}

\section{Fault-tolerant system}
There are different types of faults:
\begin{itemize}[noitemsep]
\item Fail-stop: some processes stop and terminated unexpectingly
\item Crash: Freeze, node unavailable
\item Oversight: Some messages are lost, some information are missing
\item Byzantine: No idea of what can occur: sometimes a fault, a wrong message. Unpredictible fault in your system. 
\end{itemize}

They have different durations, degree of detection and degree of correction.
\medskip

One can implement an algorithm tolerant to a given fault class, where a specific fault id not a ``bad'' action.
\bigskip

Two parameters are important for our algorithms: \emph{safety} and \emph{liveness}.


\begin{description}
\item[Safety] $E$ sends a message if and only if the previous sent message was received by $R$ and if each message send by $R$ has been received.
\item[Liveness] Every message sent by $E$ will be received (one day).
\end{description}
\paragraph{Example}
In synchronous mode:


\begin{algorithm}
\Label(Sender){
	\Label(init){
		get(m)\;
		send(m)\;
	}
	\Label(rcv(ack)){
		get(m)\;
		send(m)\;
	}
}
\Label(Receiver){
	\Label(rcv(m)){		
		dlv(m)\;
		send(ack)\;
	}
}
\end{algorithm}


\begin{algorithm}
\tcp{With correction}
\Label(Sender){
	\Label(init){
		get(m)\;
		send(seq,m)\;
	}
	\Label({recv(ack) and (ack=seq)}){
		seq=seq $\oplus 1$\;		
		get(m)\;
		send(seq,m)\;
	}
	\Label({rcv(ack) and (ack$\neq$seq)}){
		``nothing''\;
	}
}
\Label(Receiver){
	\Label({rcv(seq,m) and (ack$\neq$seq)}){		
		dlv(m)\;
		ack:=seq\;
		send(seq)\;
	}
	\Label({rcv(seq,m) and (ack=seq)}){
		send(seq)\;
	}
}
\end{algorithm}
\clearpage

\paragraph{Two Generals' problem}
Two army are fighting, the blue and the red. The red are on both side, and they need to coordinates to attack at the same time, but the messengers must pass inside the blue army.\\
$\Rightarrow$ there is no solution!

\begin{proof}
If we consider the most efficient protocol, and we kill the last messenger. Then it should still work, so we get a better protocol: contradiction!
\end{proof}

\paragraph{Fundamental limits}
\begin{itemize}[noitemsep]
\item algorithm requires an infinity of messages
\item we never attack
\end{itemize}

We will thus consider than at least one copy of each message arrives.

\begin{thm}[Fisher, Lynch and Paterson][1985]
	Consensus is impossible with an asynchronous network if a process can stop.
\end{thm}



\paragraph{Self-stabilization}
\begin{itemize}[noitemsep]
\item Wrong messages in channels
\item Wrong values in variables
\end{itemize}



\section{An application: A peer-to-peer prefix tree}
Work from the Ph.D of Cédric Tedeschi.

\begin{itemize}
\item Based on a prefix tree for the indexing system.
\item Using DHT and Clustering, build a ring between peers.
\item The load balancing is handled by a heuristic: \emph{Max Local Throughput}
\item Fault tolerance via the \emph{snap-stabilizing protocol}
\end{itemize}

\section{More Ph.D}
See slides (self-stabilisation of DIET and termination detection).

\section{Bully algorithm}

\begin{algorithm}
\Begin(Init:){
	\For{all $i$}{
		ask\_elect$_i$=false\;
		coordinator$_i$=max\;
	}
}
\Begin(R1:){
	\tcp{I want to be a leader}
	ask\_elect$_i$=true\;
	\For{all $j>i$}{
		send($p_j$, ELECT)\;
		start\_timeout(T)\;
	}
}
\Begin(R2:){
	receive($p_j$, ELECT)\;
	\If{not(ask\_elect$_i$)}{
		\For{$k>i$}{
			send($p_k$, ELECT)\;
		}
	}
	send($p_j$, ACK)\;
}
\Begin(R3:){
	receive ACK\;
	wait for receive ELECTED from $p$\;
}
\Begin(R4:){
	receive($p_j$, ELECTED)\;
	ask\_elect$_i$=false\;
	coordinator$_i$=$j$\;
}
\Fn{start\_timeout(T):}{
	receive($p_j$, ACK)\;
	send(all, ELECTED)\;
}
\end{algorithm}

\paragraph{Efficiency}
$O(n^2)$ in worst case $\to$ costly

\paragraph{Problem} It is complicated to find the best time-out

\clearpage



\section{Chang and Roberts Algorithm}
To reduce the number of messages, we can use a topology: a virtual \emph{ring}. Multiple elections can be in progress simultaneously.


\begin{algorithm}
\Begin(Candidat){
	candidat$_{i}$=true\;
	send(succ[i],ELECT,i)\;
}
\Begin(R1: receive \{p$_i$, ELECT, j\}){
	\Switch{$i,j$}{
		\Label($j>i$){
			send(succ[i], ELECT, i)\;
		}
		\Label($j<i$){
			\If{not(candidat$_{i}$}{
				candidat$_{i}$=true
				send(succ[i], ELECT, i)
			}
		}
		\Label($j=i$){
			send(all, ELECTED, i)
		}
	}
}
\end{algorithm}

\paragraph{Complexity}
$O(n\log n)$ in average, $O(n^2)$ in worst case


\section{I.Suzuki and T.Kasami's algorithm}


\begin{algorithm}
\Begin(Init){
	RN[i] = number of the last request received from p$_{i}$ to get the token\;
	LN[i] = number of the last critical section (CS) for p$_i$\;
	queue = request queue for unsatisfied request for the token
}
\Begin(Request for CS){
	\tcp{on p$_i$}
	\If{not token\_is\_here}{
		RN[i]++\;
		broadcast(REQ, i RN[i])\;
		wait(token\_is\_here)\;
	cs\_inprogress = true\;
	enter in CS\;
	}
}
\Begin(Exit of CS){
	cs\_inprogress = false\;
	LN[i] = RN[i]
	\For{all $i$ s.t. $p_j \notin$ queue}{
		\If{RN[j]==LN[j]+1}{
			enter\_queue(j)\;
		}
		\If{not(empty\_queue}{
			k = exit\_queue()\;
			send (k, TOKEN)\;
		}
	}
}
\Begin(receive \{ REQ, j, n\} ){
	RN[j] = max ( RN[j],n)\;
	\If{token\_is\_here and not cs\_inprogress}{
		\If{RN[j]==LN[j]+1}{
			send(j, TOKEN)\;
			token\_is\_here = false\;
		}
	}
}
\Begin(receive TOKEN){
	token\_is\_here = true\;
	cs\_inprogress = true\;
	enter in CS\;
}
\end{algorithm}



\end{document}