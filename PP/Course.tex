\documentclass{article}

\usepackage[utf8]{inputenc}
\usepackage[english]{babel}
 
\usepackage{caption}
%\usepackage{pgfplots}
\usepackage{listings}
\usepackage{graphicx}
\usepackage{footnote}
\usepackage{amsmath}
\usepackage{amsthm}
\usepackage{graphicx}
\usepackage{url}
\usepackage{amssymb}
\usepackage{mathrsfs}
\usepackage{multirow}
\usepackage{amsfonts}
\usepackage[boxed,linesnumbered,noend]{algorithm2e}
\usepackage{qcircuit}
\usepackage{bussproofs}
\usepackage{enumitem}
\usepackage{multicol}

\newtheorem{thm}{Théorème}
\newtheorem{prop}{Propriété}
\newtheorem{lemma}{Lemme}
\newtheorem{defi}{Definition}
\newtheorem{coro}{Corollaire}



\setlength{\oddsidemargin}{0pt}
% Marge gauche sur pages impaires
\setlength{\evensidemargin}{0pt}
% Marge gauche sur pages paires
\setlength{\textwidth}{470pt}
% Largeur de la zone de texte 
\setlength{\topmargin}{0pt}
% Pas de marge en haut
\setlength{\headheight}{13pt}
% Haut de page
\setlength{\headsep}{10pt}
% Entre le haut de page et le texte
\setlength{\footskip}{40pt}
% Bas de page + séparation
\setlength{\textheight}{630pt}
% Hauteur de la zone de texte

\title{Proofs and Programs}
\author{Phillipe Audebaud \footnote{https://perso.ens-lyon.fr/philippe.audebaud/PnP/}\\
\small ENS de Lyon}
\date{}

\newcommand\subst[2]{\langle #1\big/#2\rangle}
\newcommand\lterm[2]{\lambda #1. #2}
\newcommand\Lterm[2]{\Lambda #1. #2}
\newcommand\tlambda[0]{$\lambda$}
\newcommand\seq[0]{\text{ seq}}

\begin{document}

\maketitle
\tableofcontents
\newpage

\subsection*{Basis}
\begin{itemize}
\item Lecture: Tue 8h-10h (Philippe Audebaud)
\item Tutorial: We 8h-10h (Aurore)
\end{itemize}
10 Weeks of courses (3x3), which is really low.

\[Final\;mark = 50\% \cdot CC + 50\% \cdot Exam\]
No mid-time exam, but weekly homework.

\paragraph{Warning} Presence at the courses and tutorial will have an impact on the marks.

\paragraph{Prerequisites}
\begin{itemize}
\item L2.2 $\to$ Logical (Natacha P., Chapter 1 \& 2): 
\begin{itemize}
\item Proof theory
\item Formal system for logic inference.
\end{itemize}
\item $\lambda$-calculus
\item Category theory
\end{itemize}

\setcounter{section}{0}
\part{(Pure) $\lambda$-Calculus}
\subsection{Computing with functions ?}
How do we do mathematics ?
\begin{enumerate}[label=\Alph*.]
\item Having \emph{structures:} numbers, spaces (points, vectors, functions) $\to$ Eilenberg-Mac Lane ($\sim$ 1942) Category theory
\item Build, explore, transform structures $\to$ Church ($\sim$ 1930) \emph{$\lambda$-Calculus}
\item Compare "stuff": \emph{equality} $\to$ Voevoski ($\sim$ 2006) Algebraic topology $\to$ search HoTT (Hight order Type Theory)
\item Provide a framework (\emph{rules}) to reasoning on all that! $\to$ 1st point
\end{enumerate}


\subsection{Church $\lambda$-calculus (informally)}
\begin{align*}
f: &A \to B\\
& x \mapsto e
\end{align*}
Given $a\in A$, $f(a)$ is the "replacement of the occurrence of $x$ in $e$ by a"

\begin{align*}
f & \overset{\text{def}}{=} \lambda x.e \tag{$\lambda$-abstraction}\\
f\;a & = (\lambda a.e) \;a \tag{Application}\\
\end{align*}

\paragraph{Notation}
\[e\subst{a}{x}\] is the replacement in $e$ of all the occurrences of $a$ by $x$.

\paragraph{Example}
\begin{enumerate}
\item \begin{align*}
&\lambda x. x\\
&x\mapsto x
\end{align*}
is the identity function
\item \begin{align*}
&\lambda x.y\\
&x\mapsto y
\end{align*}
Here $x$ and $y$ are variables, $x\neq y$. $(\lambda x.y)\; a$ leads to $y\subst{a}{x}\equiv y$

\[ (\lambda x.a) \; b \to_\beta a\subst{b}{x}\]
$\to_\beta$ is a binary relation on lambda-terms $\Rightarrow$ idea of computation on terms.

\end{enumerate}

\paragraph{Notion of $\alpha$-equivalence}
\[\lambda x.a \overset{?}{=}_\alpha \lambda y.b\]
Pick a \emph{fresh} variable, let say $z$,
\[a\subst{z}{x} =_\alpha b\subst{z}{y}\]

\emph{All the results and proofs will be done up to  $\alpha$-equivalence (no difference made between $\lambda x.x$ and $\lambda y.y$).}


\section{A toolbox on $\lambda$-calculus}
\emph{$\lambda$-calculus: Syntax and Semantics}, Herk Barendregt (1977)
\bigskip


Let $\mathcal{X}$ be a measurable set of variables, ranged over by $x,y,z,...$

\begin{defi}
A $\lambda$-term $e$ is generated by the grammar:
\[ a,b,e ... ::= x\in \mathcal{X} \;|\; \lambda x.e \; | \; a \; b\]

The set of $\lambda$-terms is denoted $\Lambda$.
\end{defi}


\begin{defi}[Free variable]
The set of free variables in $e$, denoted $FV(e)$ is defined inductively:
\begin{itemize}
\item if $e\equiv x\in \mathcal{X},\; FV(x)\equiv\{x\}$
\item if $e\equiv \lambda x . a_0, \; FV(\lambda x.a_0) \equiv FV(a_0)\setminus \{x\}$
\item if $e\equiv a_1 \; a_2, \; FV(a_1\;a_2)\equiv FV(a_1) \cup FV(a_2)$
\end{itemize}
A term $e$ is \emph{closed} if $FV(e)=\emptyset$
\end{defi}

\begin{defi}[Substitution]
Given $x \in \mathcal{X}$, $a\in \Lambda$, the substitution of (all the) occurrences of $a$ in $e\in \Lambda$, denoted $e\subst{a}{x}$ is:
\begin{itemize}
\item if $y\in \mathcal{X} \setminus \{x\}, \; y\subst{a}{x}\equiv y $ and $x\subst{a}{x}\equiv a$
\item $(\lambda y.e)\subst{a}{x} = \lambda y.e \subst{a}{x}$
\item $(e\; f)\subst{a}{x} = (e\subst{a}{x})\; f\subst{a}{x} $
\end{itemize}
\end{defi}

\begin{defi}[$\to_\beta$ reduction]
\begin{align*}
\to_\beta & \subseteq \Lambda \times \Lambda\\
& \Big\{ \big( \underbrace{(\lambda x.a)\; b}_{\text{redex}},\underbrace{a\subst{b}{x}}_{\text{contraction}}\big)\big| x\in \mathcal{X},\; a,v \in \Lambda \Big\}
\end{align*}
\end{defi}

\paragraph{Example}
\begin{enumerate}
\item \begin{align*}
(\underbrace{\lambda x. (\lambda y.y)\; a}_{\nrightarrow_\beta (\lambda y.y)\; b})\; b) \to_\beta ((\lambda y.y)\; a)\;\subst{b}{x} & \equiv ((\lambda y.y)\;\subst{b}{x})a\subst{b}{x}\\
& \equiv (\lambda y.y)\; a\subst{b}{x}
\end{align*}

\item \begin{align*}
(\lambda x.y)\; a \to_\beta y\\
\end{align*}

\item \begin{align*}
(\lambda x.x\; x)(\lambda x.x \;x) \to_\beta & (x\; x)\subst{\lambda x.x \; x}{x} \text{ or } (x\; x)\subst{\lambda y.y \; y}{x}\\
& (\lambda x. x\;x)(\lambda x.x\; x)
\end{align*}
Russell paradox: we get an infinite $\beta$-reduction!
\end{enumerate}

\[\to_\beta \; \subseteq \; \beta_0 \; \subseteq \; \underbrace{\beta}_{\beta-\text{reduction}} = \beta_0^* \]
\[\to_\beta^* \text{ is the $\beta$-reduction, noted } \twoheadrightarrow_\beta \]

\begin{defi}[$\beta_0$-contraction]
Let $a,b\in \Lambda$. $a\; \beta_0 \; b$ is defined by cases:
\begin{itemize}
\item $x\; \beta_0 \; x$
\item $(\lambda x.u)v\; \beta_0 \; u\subst{v}{x}$
\item $(\lambda x.u)\; \beta_0 \; (\lambda x.v)$ if $u\; \beta_0 \; v$
\item $(u\; v)\; \beta_0 \; (u'\; v)$ if $u\;\beta_0\;u'$
\item $(u\; v)\; \beta_0 \; (u\;v')$ if $v\; \beta_0 \; v'$
\end{itemize}
\end{defi}

\noindent\textit{Maintenant en français !}

\paragraph{Remarque:} $\beta_0$ est réflexive.

\begin{defi}
La \emph{$\beta$-réduction} est la clôture transitive de $\beta_0$:
\[\beta = \beta_0^* \]
\end{defi}


\paragraph{Remarque}
Si $a,b \in \Lambda$, alors $a \; \beta \; b$ si il existe $n\geq 0$ et $(e_k)_{0\leq k \leq n}$ $\lambda$-termes tels que :
\begin{itemize}
\item $a=e_0$ et $b=e_n$
\item pour tout $k<n$, $e_k \; \beta_0 \; e_{k+1}$
\end{itemize}

\begin{defi}
Soit $\mathcal{R}$ une relation binaire sur $\Lambda$. On dit que $\mathcal{R}$ est \emph{$\lambda$-compatible} si elle satisfait les propriétés suivantes :
\begin{itemize}
\item $x\;\mathcal{R}\; x$
\item si $a \; \mathcal{R} \; b$ et $c \;\mathcal{R}\; d$ alors $a\; c\; \mathcal{R} \; b \; d$
\item si $a\; \mathcal{R}\; b$ alors $\lambda x.a\; \mathcal{R}\; \lambda x.b$
\end{itemize}
\end{defi}

\begin{prop}
La $\beta$-réduction est la plus petite relation $\lambda$-compatible et transitive contenant $\to_{\beta}$
\end{prop}

\begin{proof}
$\star$ On vérifie d'abord :
\[\to_\beta \; \subseteq \beta_0 \subseteq \beta_0^* = \beta \]

D'autre part, $\beta_0$ est $\lambda$-compatible :
\begin{itemize}
\item par réflexivité, $x \; \beta	\; x$
\item soit $a\; \beta \; b$ ; par définition,il existe $n\geq 0$, $(e_k)_{0\leq k \leq n}$ tel que $a=e_0$, $b=e_n$ et pour tout $k<n$, $e_k \; \beta_0 \; e_{k+1}$. Du coup, par définition de $\beta_0$, pour tout $k<n$, $\lterm{x}{e_k} \; \beta_0 \; \lterm{x}{e_{k+1}}$.

Ainsi, $\lterm{x}{a} \; \beta \; \lterm{x}{b}$.
\end{itemize}

$\star$ Soit $\mathcal{R}$ une autre relation $\lambda$-compatible et transitive contenant $\to_\beta$. Montrons que $\beta\subseteq \mathcal{R}$. Il "suffit" de vérifier que $\beta_0 \subseteq \mathcal{R}$ (laissé en exercice).
\end{proof}

\subsection*{Propriétés essentielles de la $\beta$-réduction}
\paragraph{Remarque}
$(\Lambda, \beta_0)$ est un \emph{système de réduction abstrait\footnote{Cf ThPr}}.

\begin{defi}[Forme normale, Relation normalisante]
Soit $\mathcal{R}$ une relation binaire sur $\Lambda$,
\begin{itemize}
\item On dit que $a$ \emph{est} une forme normale (relativement à $\mathcal{R}$) s'il n'existe pas $b\in \Lambda$ tel que $a\; \mathcal{R} \; b$.
\item On dit que $a$ \emph{a} une forme normale (relativement à $\mathcal{R}$) s'il existe $b\in \Lambda$ tel que $b$ \emph{est} une forme normale et $a \; \mathcal{R}^*\;b$
\item On dit que $\mathcal{R}$ est normalisante si tout $a \in \Lambda$ a une forme normale
\end{itemize}
\end{defi}


\paragraph{Exemple}
\begin{itemize}
\item $\lterm{x}{x}$ est une forme normale relativement à $\beta_0$
\item $\beta_0$ n'est pas normalisante !
\begin{align*}
\Omega & \equiv (\lterm{x}{x\; x})\; (\lterm{x}{x\; x})\\
\Omega & \to_\beta \Omega
\end{align*}
\end{itemize}

\begin{defi}[Confluence]
Soit $\mathcal{R}$ une relation binaire sur $\Lambda$. On dit que $\mathcal{R}$ est confluente si pour tout $(a,b,c) \in \Lambda^3$ tel que 
\[a \; \mathcal{R}^* \; b \text{ et } a\; \mathcal{R}^*\; c\]
alors il existe $d\in \Lambda$, tel que
\[b \; \mathcal{R}^* \; d \text{ et } c \; \mathcal{R}^* \; d\]
\end{defi}

\begin{thm}
La $\beta_0$-réduction est confluente.
\end{thm}
\begin{proof}
En semaine 3 ou 4.
\end{proof}

\begin{coro}
Tout $\lambda$-terme admet au plus une forme normale, relativement à $\beta_0$
\end{coro}

\subsection*{Notion d'égalité sur les \tlambda-termes}

\begin{defi}
La $\beta$-équivalence sur $\Lambda$ est la relation binaire notée $=_\beta$, définie comme la clôture réflexive symétrique transitive de $\beta_0$ :\\
$a=_\beta b$ s'il existe $n\geq 0$ et $(e_k)_{0\leq k \leq n}$ tel que $a=e_0$ et $b=e_n$ et $\forall k<n$, soit $e_k \; \beta_0 \; e_{k+1}$ soit $e_{k+1} \; \beta_0 \; e_{k}$
\end{defi}

\begin{defi}[\tlambda-congruence]
Une relation binaire $\mathcal{R}$ (sur $\Lambda$) est une \tlambda-congruence si c'est une relation d'équivalence et qu'elle est \tlambda-compatible.
\end{defi}

\begin{thm}[Church-Rosser]
Pour tout $(a,b)\in \Lambda^2$, $a=_\beta b$ si et seulement si il existe $c\in \Lambda$ tel que $a \; \beta \; b$ et $b \; \beta \; c$
\end{thm}

\begin{proof}
La condition est suffisante
\bigskip

Réciproquement, pour la condition nécessaire, on introduit $R\subseteq \Lambda \times A$ défini par :\\
$a \; \mathcal{R} \; b$ s'il existe $c$ tel que $a\; \beta \; c$ et $b \; \beta \; c$.

On remarque, par définition de $\mathcal{R}$,
\begin{itemize}
\item $\mathcal{R}$ est réflexive et symétrique
\item $\mathcal{R}$ est transitive
\end{itemize}
De plus, $\mathcal{R}$ contient $\beta$ (ou $\beta_0$). Donc, si $a=_\beta b$, alors $a\; R \; b$.
\end{proof}

\begin{thm}
La relation d'équivalence $=_\beta$ est la plus petite \tlambda-congruence contenant $\to_\beta$
\end{thm}
\begin{proof}
En exo.
\end{proof}



\paragraph{Notation} On note $\equiv$ pour une définition ($\overset{\text{def}}{=}$), mais aussi pour l'$\alpha$-équivalence ($=_\alpha$). On peut utiliser la notation de Bruijn (cf références).


\section{Calcul propositionnel et correspondance de Curry-Howard}
\subsection{Éléments de langage (informels)}
\begin{itemize}
\item Théorie de la \emph{démonstration} (prouvabilité)
\item Thème des modèles
\end{itemize}

Quelques "ingrédients" :
\begin{itemize}
\item \emph{énoncés} (logiques) : ici les familles du calcul propositionnel:
\[A ::= x \;|\: \top \:|\; \bot \;|\; A\Rightarrow B \;|\; A\land B \;|\; A \lor B \;|\; \neg A\footnote{$\neg A$ signifie en fait $A \Rightarrow \bot$} \label{eq:star}\tag{$\star$}\]
La notation "$A$ propriété" signifie que $A$ est engendrée par la grammaire (\ref{eq:star})
\item On parle de \emph{jugements} sur ces énoncés : "$A$ true" 
\item On introduit aussi des jugements hypothétiques : 
$A_1$ true, $A_2$ true, ... ,$A_n$ true $\vdash$ $B$ true
\end{itemize}


Commentaires sur les différentes règles de (NJ):
\begin{itemize}
\item Le vrai
\item L'implication (/!\textbackslash: $A \Rightarrow B \neq \neg A \lor B$ dans (NJ))
\item Le faux
\item La négation
\item La disjonction
\end{itemize}


\subsection{Fragments $\lambda_\to$}
On peut associer des règles au typages de \tlambda-termes en raisonnant sur $\lterm{x}{t} : T$


\begin{thm}[Curry-Howard]
Le fragment $NJ_\to$ et $\lambda_\to$ sont en correspondance via:
\begin{enumerate}
\item Si $\Delta \vdash t : T$ dans $\lambda_\to$
%TODO
\end{enumerate}
\end{thm}

\begin{figure}[h]
\begin{center}
\begin{tabular}{| c | c |}
\hline
$\lambda_\to$ & $NJ_\to$\\
\hline
\emph{type} & \emph{proposition}\\
\hline
variable de type & proposition atomique\\
type flèche & implication\\
\hline
\emph{terme} & \emph{dèrivation}\\
\hline
variable de terme & hypothèse\\
\tlambda-abstraction & règle d'introduction\\
application & règle d'élimination\\
$\beta$-redex & coupure\\
$\beta$-réduction & transformation sur les dérivations\\
\hline
\emph{forme normale} & \emph{dérivation sans coupure}\\
\hline
\end{tabular}
\caption{Correspondance de Curry-Howard}
\end{center}
\end{figure}


\subsection{Interprétation BHK}
L'interprétation de Brouwer-Heyting-Kolmogorov consiste à construire un témoin (une preuve) d'une proposition selon le protocole suivant :
\begin{itemize}
\item Un témoin pour $A \land B$ est une paire formée par un témoin pour $A$ et un témoin pour $B$
\item Il y a un témoin unique pour $\top$
\item Un témoin pour $A\lor B$ est soit un témoin pour $A$, soit un témoin pour $B$
\item Il n'y a pas de témoin pour $\bot$
\item Un témoin pour $A\Rightarrow B$ est une application de témoins pour $A$ vers des témoins pour $B$
\item Un témoin pour $\neg A$ est un témoin de $A \Rightarrow \bot$
\end{itemize}
Avec $A, B$ engendrés par la grammaire
\[A ::= X \;|\: A \Rightarrow A \;|\; A \lor A \;|\; A \land A \;|\; \top \;|\; \bot\]

\begin{defi}[Produit (paire)]
Soit A,B. Le produit de $A$ par $B$ est le damier de $A\times B$, et de la propriété universelle suivante :\\
Pour tout $f : D \to A$ et $g : D\to B$, il existe $h : D\to A\times B$ tel que $\pi_1 \circ h = f$ et $\pi_2 \circ h = g$\footnote{Ces égalité correspondent à des $\beta$-réduction dans le \tlambda-calcul}.

De plus, $h$ est \emph{unique} et ne dépend que de $f$ et de $g$,
\[h = \langle f, g \rangle\]
\end{defi}

Par ailleurs, si $e : D \to A \times B$, alors 
\[
\begin{cases}
\pi_1 \circ e : D \to A\\
\pi_2 \circ e : D \to B\\
\end{cases}
\]
Pour ce couple, il existe $\langle \pi_1 \circ e, \pi_2 \circ e\rangle : D \to A \times B$

Du coup, par unicité, on a nécessairement
\[\langle \pi_1 \circ e, \pi_2 \circ e\rangle = e \]

Cette observation donne lieu à :
\begin{itemize}
\item une transformation sur les dérivations
\item une autre forme de réduction sur les \tlambda-termes
\end{itemize}
On parle alors d'$\eta$-réduction.

\bigskip

On rajoute alors les règles de typage du produit ($\times_i$) et ($\times_{E,k}$) pour $k\in \{1,2\}$.


\begin{defi}[Somme (coproduit)]
Soit $A,B$. C'est la donné de $A + B$ avec la propriété universelle suivante :\\
Si $f : A \to C$, et $g : B \to C$, il existe $k : A + B \to C$ unique, ne dépendant que de $f$ et de $g$ noté $\{ f, g \}$, tel que 
\[
\begin{cases}
k \circ in_l = f\\
k \circ in_r = g
\end{cases}
\]

Par ailleurs, si on se donne
\[e : A + B \to C\]
Alors
\[
\begin{cases}
e \circ in_l : A \to C\\
e \circ in_r : B \to C
\end{cases}
\]
Donc 
\[\{ e \circ \in_l, e \circ in_r\} = e\]
\end{defi}

On rajoute alors trois règles : (+Ig), (+Id) et (+E)




\section{$\lambda$-calcul simplement typé}
\subsection{Quelques Lemmes}

\begin{lemma} 
Si $\Delta \vdash t : T$ clos, $FV(t) \subseteq FV(\Delta)$, où $FV(\emptyset)=\emptyset$, et $FV(\Delta,x : S) = FV(\Delta)\cup \{x\}$.
\medskip

\emph{Attention} : Un contexte de typage $\Delta \equiv x_1 : S, ..., x_p : S_p$ où $p\geq 0$ est valide si les variables $x_1, ... ,x_p$ sont distinctes deux à deux.
\end{lemma}

On peut rajouter des règles sur la validité de $\Delta$ en tant que contexte.

\begin{prooftree}
\AxiomC{}
\AxiomC{}
%\RightLabel{\scriptsize Rule Name}
\BinaryInfC{$\emptyset$ contexte valide}
\end{prooftree}

\begin{prooftree}
\AxiomC{$\Delta$ contexte valide}
\AxiomC{$T$ type}
\AxiomC{$x\notin FV(\Delta)$}
%\RightLabel{\scriptsize Rule Name}
\TrinaryInfC{$\Delta, x : T$ contexte valide}
\end{prooftree}

Et on augmente (Hyp).

\begin{prooftree}
\AxiomC{$\Delta$ contexte valide}
\AxiomC{$x : T \in \Delta$}
\RightLabel{\scriptsize (Hyp)}
\BinaryInfC{$\Delta, x : T$ contexte valide}
\end{prooftree}


\begin{lemma}[Affaiblissement]
Si $\Delta \vdash t : T$ et si $\Delta \subseteq \Delta'$, avec $\Delta'$ contexte valide, alors $\Delta' \vdash t : T$.
\end{lemma}

\begin{proof}
Par induction sur la dérivation principale, c'est-à-dire
$\Delta \vdash t : T$.
Le seul cas ``délicat'' est lorsque
\begin{prooftree}
\AxiomC{$\Delta,x:U \vdash a : V$}
\RightLabel{\scriptsize $\to_I$}
% TODO
\UnaryInfC{}
\end{prooftree}
\end{proof}


\begin{thm}
Si $\Delta \vdash t : T$, alors $t$ est fortement normalisant.
\end{thm}

\begin{proof}
Deux parties : poser la notation générale, puis l'adapter à $\to_\lambda$.

\begin{enumerate}
\item \emph{Définition générale :} Si $e \in \Lambda$, $e\equiv \lterm{\overline{x}}{\Delta \; \overline{u}}$ avec $|\overline{x}|>0$, $|\overline{u}|>0$ et $\Delta \in \mathcal{X}$ ou bien $\Delta$ est un $\beta$-redex
\begin{itemize}
\item $e$ est en forme normale si $\Delta \in \mathcal{X}$ et chaque $u_i$ est en forme normale
\item $e$ est une forme normale de tête (HNF) si $\Delta \in \mathcal{X}$
\item si $e$ n'est pas en HNF, c'est-à-dire $\Delta$ est un $\beta$-redex, $\Delta$ est appelé \emph{redex de tête}.
\end{itemize}
\end{enumerate}

\begin{defi}
$e \in \Lambda$ est fortement normalisant (SN) s'il n'existe pas de $\beta$-réduction infinie issue de $e$
\end{defi}

\paragraph{Exemple}
\begin{itemize}
\item $\Omega$ n'est pas SN
\item $(\lterm{x}{\lterm{y}{y}}) \; \Omega$ n'est pas SN (il existe une dérivation infinie) $\to$ \emph{attention} : la $\beta$-équivalence n'est pas compatible avec la propriété d'être fortement normalisant.
\end{itemize}

Par contre, si $a=_\beta b$ et $b$ a une NF (resp HNF), alors $a$ a une NF (resp HNF)

De plus :
\begin{itemize}
\item Si $e$ a une NF (resp HNF), $\lterm{x}{e}$ a une NF (resp HNF)
\item Si $e$ est SN, $\lterm{x}{e}$ est SN
\end{itemize}
\bigskip

Soit $\mathcal{N}$ l'ensemble des termes SN, et $\mathcal{N}_0 \equiv \{  x \; \overline{u} \;|\; x \; \overline{u} \in \mathcal{N}\} \subseteq \mathcal{N}$
\paragraph{Notation}
\begin{itemize}
\item $e\in \Lambda$, $Succ(e)=\{e'\in \Lambda \;|\; e \;\beta_0 \; e' \}$, et cet ensemble est \emph{fini} (réduction à branchements fini)
\item Lemme de Koenig : si un arbre est infini et que cet arbre est a branchement fini, alors il existe un chemin infini
\end{itemize}

Si $e\in \mathcal{N}$, $\bigcup_{p\geq 0} Succ^p (e)$ est \emph{fini}, de sorte que la définition suivante est bien fondée :

\begin{defi}
Pour $e\in \mathcal{N}$, $\ell(e)$ désigne la somme des longueurs des chemins de tout réduction issue de $e$.
\end{defi}

\begin{lemma}
Sont immédiats :
\begin{itemize}
\item Si $e\in \mathcal{N}$, alors $\lterm{x}{e} \in \mathcal{N}$
\item Si de plus $e' \in \mathcal{N}$ et $e \; \beta \; e'$, alors $e' \in \mathcal{N}$
\item Si $e \in \Lambda$ tel que $Succ(e)\subseteq \mathcal{N}$, alors $e\in \mathcal{N}$
\end{itemize}
\end{lemma}
\begin{proof}
Pour le troisième point, soit $e\in \Lambda$ tel que $Succ(e)\subseteq \mathcal{N}$, pour tout $e' \in Succ(e)$, $\ell (e') < \ell(e)$ $\to$ une récurrence simple sur $\ell(e)$ permet d'établir $\mathcal{P}(e) \equiv$ ``$Succ(e) \subseteq \mathcal{N}$ implique $e\in \mathcal{N}$''
\end{proof}

\end{proof}

Soit $\mathcal{N}_0\equiv \{ \underbrace{x\;\bar{u}}_{(((x\; u_1)\; u_2)\; ...)\; u_n} \;|\; x\;\bar{u} \in \mathcal{N} \} \subseteq \mathcal{N}$
\begin{lemma}
\begin{enumerate}[label=\roman*)]
\item Si $e\in \mathcal{N}$, alors $\lterm{x}{e}\in \mathcal{N}$
\item Si $e \in \mathcal{N}$, et $e\; \beta'\; e$, alors $e'\in \mathcal{N}$
\item Si $e \in \Lambda$ tel que $Succ(e)\subseteq \mathcal{N}$, alors $e\in \mathcal{N}$
\item Si $e\in \Lambda$ et $x\in \mathcal{V}$, $e\; x \in \mathcal{N}$ implique $e\in \mathcal{N}$ 
\end{enumerate}
\end{lemma}

\begin{lemma}
Si $b\in \mathcal{N}$ et $a\subst{b}{x}\; \bar{u} \in \mathcal{N}$
\end{lemma}
\begin{proof}
Par récurrence sur $l(b) + l(a\subst{b}{x}\; \bar{u})$
Il suffit de montrer que $\underset{\substack{\downarrow\\ \beta_0}}{Suc}((\lterm{x}{a}\; b\; \bar{u}) \subseteq \mathcal{N}$
\end{proof}
%TODO some 4 lines are missing here.

\subsection{Parties saturées de $\Lambda$}
\begin{defi}
Soit $S\subseteq \Lambda$. Ont dit que $S$ est \emph{saturée} si elle satisfait les conditions suivantes:
\begin{align*}
\mathcal{N}_0 \subseteq S \subseteq \mathcal{N} \tag{Sat 1}\\
\text{Si } e\in S \text{ et } e\;\beta_0^*\;e'\tag{Sat 2}\\
\text{Si $e\in \Lambda$ et $e$ n'est pas une \tlambda-abstraction, et si } Succ(e)\subseteq S, e\in S\tag{Sat 3}
\end{align*}

\end{defi}


\begin{prop}
\begin{enumerate}[noitemsep,label=\roman*)]
\item $\mathcal{N}_0$ est saturée
\item $\mathcal{N}$ est saturée
\item Si $X, Y$ sont des parties saturées, alors $X \to Y = \{ e\in \Lambda \;|\; \forall a\in X, e\; a \in Y\}$ est saturée
\end{enumerate}
\end{prop}
\begin{proof}
\begin{enumerate}[noitemsep,label=\roman*)]
\item En exercice
\item Il suffit de vérifier (Sat 3)
En clair, soit $e\in \Lambda$ qui n'est pas une \tlambda-abstraction, et tel que $Succ(e)\in \mathcal{N}$.\\
On procède par induction structurelle sur $e$.
\begin{itemize}[label=*]
\item $e\in \mathcal{X}$ trivial
\item $e\equiv e_0\; a$, avec $Succ (e_0\; a)\in \mathcal{N}$
\[
\left.
\begin{array}{r}
e\;\beta_0\; e_0'\; a, \text{ avec } e_0\;\beta_0\; e_0'\\
e\;\beta_0\; e_0\; a', \text{ avec } a\;\beta_0\; a'\\
e\;\beta_0\; e_1\subst{a}{x}, \text{ si } e_0\equiv\lterm{x}{e_1}\\
\end{array}
\right]
\begin{array}{l}
e_0'\; a \in \mathcal{N}\\
e_0\; a'\in \mathcal{N}\\
e_1\subst{a}{x}\in \mathcal{N}\\
\end{array}
\]
\end{itemize}
\item $X,Y\in Sat(\Lambda)$, montrons que $X\to Y\in Sat(\Lambda)$\\
(Sat 1)
\begin{itemize}[label=*]
\item $\mathcal{N}_0\subseteq X \to Y$ ?\\
Si $x\;\bar{u}\in \mathcal{N}$, et $a\in X \subseteq \mathcal{N}$
\item $X\to Y\subseteq \mathcal{N}$\\
Si $e\in X\to Y$, c'est-à-dire pour tout $a\in X$, $e\; a \in Y \subseteq \mathcal{N}$
\end{itemize}
\smallskip


(Sat 2) facile : $e\in X\to Y$ et $e\; \beta_0\; e'$\\
Si $a \in X$, $(e\; a)\; \beta_0\;(e'\; a)$ donc $e'\; a \in Suc(e\; a)$. Par hypothèse, $e\in X \to Y$, c'est-à-dire $e\; a \in Y$. Par (Sat 2) appliqué à $Y$, $e'\; a\in Y$.
\smallskip

(Sat 3) Soit $e$ qui n'est pas un \tlambda-abstraction et tel que $Succ(e)\subseteq X \to Y$ ; montrons que $e\in X \to Y$. Cela revient à établir que pour tout $a\in X, e\; a \in Y$. On montre ça en appliquant (Sat 3) à $Y$, car $e\; a$ n'est pas un \tlambda-abstraction.

Il suffit de vérifier que $Succ(e\; a)\subseteq Y$ :
\begin{align*}
e \; a \; \beta_0\; e' \; a \text{ avec } e\; \beta_0 \; e'\\
e \; a \; \beta_0\; e' \; a \text{ avec } a\; \beta_0 \; a'\\
\end{align*}
Et c'est tout !
On remarque qu'il suffit de faire une démonstration par récurrence sur $\ell (a)\to$ à faire !
\end{enumerate}
\end{proof}

\subsection{Normalisation pour $\lambda_\to$}
\begin{thm}[SN]
Si $\Delta \vdash_{\lambda_\to} e:T$, alors $e\in \mathcal{N}$
\end{thm}
\begin{proof}
\begin{itemize}[noitemsep,label=*]
\item Une ``interprétation'' des types commes parties saturées
\item Un lemme d'``étiquetage''
\item Le théorème apparaît comme corollaire.
\end{itemize}
\end{proof}

\begin{defi}
Soit $\rho \in \mathcal{V}\to Sat(\Lambda)$. On définit par induction structurelle l'interprétation d'un type $T$ selon $\rho$ notée $[T]_\rho$:
\begin{itemize}[label=*]
\item Si $T\in \mathcal{V}, [T]_\rho \equiv \rho(T)$
\item Si $T\equiv U\to V$, $[T]_\rho \equiv [U]_\rho \to [V]_\rho$
\end{itemize}
\end{defi}
\begin{proof}
D'après la proposition précédente :
\begin{itemize}[noitemsep,label=-]
\item $\rho$ existe
\item $[T]_\rho \in Suc(\Lambda)$
\end{itemize}
\end{proof}

\paragraph{Remarque}
$[T]_\rho$ ne dépend que de $\rho \upharpoonright FT(T)$

\begin{defi}
$x_1,...,x_n \in \mathcal{X}$ et $t_1,...,t_n \in \Lambda$. La substitution $\sigma\equiv \langle t1/x_1,...,t_n/x_n\rangle$ donne lieu à $\sigma(t)$ oú $t\in \Lambda$:
\begin{itemize}[noitemsep,label=-]
\item Si $t\equiv x_i \in \{x_1,...x_n\}$, alors $\sigma(t)=t_i$
\item Si $t\in \mathcal{X}\{x_1,...,x_n\}, \sigma(t) \equiv t$
\item Si $t\equiv \lterm{x}{t_0}$, on peut supposer que 
$x\notin \{x_1,...,x_n\}\cup \{ \cup_{i\leq n}FV(t_i) \}$ 
et $\sigma (t) \equiv \lterm{x}{\sigma (t_0)}$
\item Si $t\equiv a\; b$, $\sigma(t)=\sigma(a)\; \sigma(b)$
\end{itemize}
\end{defi}

\begin{prop}
Soit $\rho$ une interprétation des types et $\Delta\vdash_{\lambda_to} t:T$\\
Pour tout substitution $\sigma$ de domaine $\subseteq FV(\Delta)$ telle que pour tout $x:S\in \Delta, \sigma (x) \in [S]_\rho$, on a:
\[\sigma ( t) \in [T]_\rho\]
\end{prop}
\begin{proof}
Par induction sur ka hauteur de la dérivation, le seul cas intéréssant est $\to_i$
\begin{prooftree}
\AxiomC{$\Delta,x:U \vdash a : V$}
\UnaryInfC{$\Delta \vdash t : U \to V$}
\end{prooftree}
Où $t\equiv \lterm{x}{a}$.

Soit $\sigma$ de domaine $\subseteq FV(\Delta)$ ; montrons que si $\forall x : S \in \Delta, \sigma(x) \subseteq [S]_\rho$, alors $\sigma (\lterm{x}{a})\in [U\to V]_\rho = [U]_rho \to [V]_\rho$.
Par définition, cela revient à montrer que pour tout $b\in [U]_\rho$, $\sigma (\lterm{x}{a})\; b \in [V]_\rho$.
On peut faire en sorte que 
\[x\notin dom(\Delta) \cup \{ \underset{y:S\in \Delta}{\cup} FV(\sigma(y))\}\cup FV(b)\]
Du coup, $\sigma'\equiv \langle t_1/x_1,...,t_n/x_n,b/x\rangle$ où $\sigma \equiv \langle t_1/x_1,...,t_n/x_n\rangle$ avec $\{x_1,...,x_n\}\subseteq dom(\Delta)$

Et $\sigma'$ satisfait :
\[\sigma (\lterm{x}{a})\; b \equiv \sigma'(a)\]
$\sigma'$ satisfait les hypothèses relatives à $\Delta, x: U$ car $\sigma'(x)=b\in [U]_\rho$.
Du coup, par hypothèse d'induction,
\[\sigma'(a) \in [V]_\rho\]
Finalement,
\[(\lterm{x}{\sigma (a)}\; b \; \beta_0 \; \sigma' (a) \in [V]_\rho\]
En résumé, $\lterm{x}{\sigma (a)} \equiv \sigma (\lterm{x}{a}) \in [U\to V]_\rho$
\end{proof}



%TODO : Slides ?


\part{Polymorphisme}
\section{Abstraction des types}

\begin{figure}[h]
\begin{center}
\begin{tabular}{c | c | c}
Programmation & Théorie des types & Raisonnement\\
\hline
\tlambda-calcul & $\lambda_\to$ & NJ ($\Rightarrow$) \\
\tlambda-calcul enrichi & $\lambda_{\to,\times,\top\bot,...}$ & NJ\\
 &  $\lambda_\mu$ ($\sim$ 1990) & NK\\
Calcul de combinateurs (S,K,I) &  & Système de Hilbert (1900)\\
\hline
* & Système F & // \\
\hline // & ++ & //\\
\end{tabular}
\caption{Correspondance programmation - langage de preuve}
\end{center}
\end{figure}


\subsection{Motivations}
Prenons quelques exemples:
\begin{enumerate}
\item L'identité :\\
Dans $\lambda_\to$, $\vdash \lterm{x}{x} : T \to T$ pour n'importe quel type $T$.\\
On voudrait donner à $\lterm{x}{x}$ un type ``polymorphe''
\item Un entier de Church :\\
$\bar{z} = \lterm{x}{\lterm{f}{f (f x)}}$. Dans $\lambda_\to$, $\vdash \lterm{x}{\lterm{f}{f (f x)}} : A \to (A \to A) \to A$\\
$x:A, f:B \vdash f\;\underbrace{(f\;x)}_{:V} : C$
\begin{enumerate}[noitemsep]
\item $f:b \equiv U \to V$ et $x : U$ donc $A \equiv U$
\item $B \equiv V \to W$ et $W \equiv C$ et $U \to V \equiv V \to W$ d'où $U \equiv V$ et $V\equiv W$
\item $\Delta \equiv \lterm{x}{x\;x}$ qui nécessite $W=W\to V$.
\end{enumerate}
\end{enumerate}


\paragraph{Observations}
\begin{itemize}[label=$\star$]
\item On veut donner à un terme ``plusieurs types'' ``d'un coup ''
\begin{description}[noitemsep]
\item[En C] $\to$ template
\item[En O'Caml] $\to$ \texttt{fun x $\to$ x} : $'\alpha \to \alpha$
\end{description}
\item On peut avoir besoin de gérer plusieurs occurences d'une même variable
\item Observation de J.Reunolds (`74)
\item Introduit par JY Girard (`70) (cadre logique)
\end{itemize}

\subsection{Le système F à la Church}
Que veut-on ?
\begin{itemize}[noitemsep, label=$\star$]
\item Une notion de généralisation (d'abstraction) sur les types
\[T ::= X\in \mathcal{V} \;|\; T\to T \;|\; \forall X.T\]
\item Une notion \emph{d'instanciation} de type
\[ \forall X.T \rightsquigarrow T\subst{S}{X}\]
\end{itemize}
La conséquence sur les termes :
\[ t ::= x \in \mathcal{X} \; | \;  \lterm{x^T}{t}  \;|\; t\;t \;|\; \Lterm{X}{t} \;|\; t\; T \]

(Hyp), ($\to_{i}$)\footnote{où $\lterm{x^S}{t}$ précise le type de $x$} et ($\to_E$) de $\lambda_\to$ plus :

\begin{multicols}{2}
\begin{prooftree}
\AxiomC{$\Delta \vdash t:T$}
\AxiomC{$X \notin FT(\delta)$}
\RightLabel{\scriptsize ($\forall_i$)}
\BinaryInfC{$\Delta \vdash \Lterm{X}{t}:\forall X.T$}
\end{prooftree}

\begin{prooftree}
\AxiomC{$\Delta \vdash t: \forall X.T$}
\RightLabel{\scriptsize ($\forall_e$)}
\UnaryInfC{$\Delta \vdash t\subst{S}{X} : T\subst{S}{X}$}
\end{prooftree}
\end{multicols}

\paragraph{Retour sur les exemples}
\begin{enumerate}
\item $\vdash \lterm{x}{x} : X \to X$ devient $\Lterm{X}{\lterm{x^X}{x}} : \forall X\; X\to X$
\item $\bar{z} \equiv \lterm{x}{\lterm{f}{f\;(f\;x)}}$
\begin{prooftree}

\AxiomC{$\vdash \lterm{x^X}{\lterm{f^{X\to X}}{f\;(f\; x)}}: X \to (X\to X)\to X$}
\UnaryInfC{$\vdash\Lterm{X}{\lterm{x}{\lterm{f}{f\; (f\; x)}}}: \forall X.X\to (X\to X)\to X$}
\end{prooftree}

\item $\lterm{x}{x\;x}$
\begin{itemize}[noitemsep,label=$\star$]
\item $x\; U : U \to U$
\item $x\; V : V \to V$\\ avec $V\to V \equiv U$
\end{itemize}
On en déduit que $W\equiv V\to V$.
\begin{prooftree}

\AxiomC{$x\; U : U \to U$}
\UnaryInfC{$x:\forall X.S \vdash x\; U: (V \to V) \to W $}

\AxiomC{$x\; V : V \to V$}
\UnaryInfC{$x:\forall X.S \vdash x\; V : V\to V$}


\BinaryInfC{$x:\forall x :S \vdash x\; U \; (x\;V):W^{V\to V}$}
\UnaryInfC{$\vdash \lterm{x^{\forall X.S}}{x\; U \; (X\;V)}:(\forall X.S)\to W$}
\end{prooftree}

On trouve un type $((\forall X.(X\to X)\to \forall X.(X\to X) )\to ((\forall X.(X\to X)\to \forall X.(X\to X) )$


\subsubsection{Système F à la Curry}
\begin{itemize}[label=$\star$]
\item \tlambda-term pur
\item Même types que dans le système à la Church
\item (Hyp), ($\to_i$), ($\to_e$)
\newpage
\item Plus 
\begin{multicols}{2}
\begin{prooftree}
\AxiomC{$\Delta \vdash t : T$}
\AxiomC{$X\notin FT(\Delta)$}
\RightLabel{\scriptsize ($\forall_i$)}
\BinaryInfC{$\Delta \vdash t : \forall X.T$}
\end{prooftree}

\begin{prooftree}
\AxiomC{$\Delta \vdash t : \forall X.T$}
\RightLabel{\scriptsize ($\forall_e$)}
\UnaryInfC{$\Delta \vdash t : T\subst{S}{X}$}
\end{prooftree}

\end{multicols}
\end{itemize}


\end{enumerate}


\subsubsection{Aspects dynamiques}
\begin{enumerate}[label=\alph*.]
\item A la Church :
\begin{itemize}[label=$\star$]
\item $(\lterm{x^T}{a})\; b \to_\beta a\subst{b}{x}$
\item $(\Lterm{X}{t})\; S \to_\beta t\subst{S}{X}$
\end{itemize}
\item A la Curry :
\end{enumerate}

\section{Propriétés méta du système F}


\begin{figure}[h]
\begin{center}
\begin{tabular}{c|c|c}
& F-Church & F-Cury\\
\hline
Confluence (Church-Rosser) & $\times$ & Déjà fait\\
\hline
SR (Subject Reduction) & Facile & Plus dure\\
\hline
SN (Strong Normalisation)& En TD & A partir de $\lambda_\to$\\
\end{tabular}
\caption{Propriétés de confluence de systèmes à la Curry et à la Church}
\end{center}
\end{figure}


\begin{lemma}[Dans les deux versions du système F]


\begin{enumerate}[label=\roman*)]
\item Si $\Delta \vdash t : T$ et $\Delta'$ contexte de typage tel que $\Delta_{\upharpoonright FV(t)}=\Delta_{\upharpoonright FV(t}$, alors $\Delta'\vdash t:T$
\item Si $\Delta\vdash t:T$, alors pour tout $S$ et pour tout $X\in \mathcal{V}$, $\Delta\subst{S}{X} \vdash \underbrace{t\subst{S}{X}}_{t}:T\subst{S}{X}$
\item Si $\Delta,x : S\vdash t:T$ et $\Delta \vdash s : S$ alors $\Delta\vdash t\subst{s}{x}:T$
\end{enumerate}
\end{lemma}

\subsection{Propriétés de F-Church}
\begin{prop}[Subject reduction]
Si $\Delta \vdash_{Church} t:T$ et $t\to t'$ alors $\Delta \vdash t' : T$
\end{prop}

\begin{proof}
En exercice.

Pour rappel, $\to$ est l'analogue de $\beta_0$:\[\to_\beta \; \subseteq \; \to\]
également dans un terme.
\end{proof}

\begin{prop}[Strong Normalization]
Si $\Delta \vdash_{Church} t : T$, alors $t$ est fortement normalisant.
\end{prop}
\begin{proof}
Fiche d'exercice (semaine 7), basée sur le résultat pour $\lambda_\to$
\end{proof}

\begin{prop}[Confluence faible]
Si $\Delta \vdash_{Church} t:T$ et si $t\to t_1$, $t\to t_2$, il existe $t_3$ tel que $t_1\to^*t_3$ et $t_1\to^*t_3$.
\end{prop}

\begin{coro}[Confluence]
La relation $\to$ est confluente sur les termes bien typés.
\end{coro}

\begin{proof}
On s'appuie sur SN et la confluence faible.
\end{proof}


\subsection{Propriétés de F-Curry}
On veut établir le résultat suivant:
\begin{prop}
Si $\Delta \vdash_{Curry}t:T$ et $t\underset{\beta_0}{\nrightarrow} t'$, alors $\Delta \vdash t' : T$.
\end{prop}
Le résultat est vrai, mais sa démonstration nécessite des détours...


\subsubsection{Système alternatif}
On introduit la notion de \emph{séquence de types}.
\[\Delta \to_n T_0,...,T_n \; \seq\]
avec les règles suivantes :

\begin{multicols}{2}
\begin{prooftree}
\AxiomC{}
\LeftLabel{\scriptsize (Ax)}
\UnaryInfC{$\Delta \vdash_0 T \seq$}
\end{prooftree}

\begin{prooftree}
\AxiomC{$\Delta \to_n T_0,...,T_n \; \seq$}
\AxiomC{$X \notin FT(\Delta)$}
\LeftLabel{\scriptsize (Gen)}
\BinaryInfC{$\Delta \vdash_{n+1} T_0,...,T_n, \forall X.T_n \seq$}
\end{prooftree}

\begin{prooftree}
\AxiomC{$\Delta \to_n T_0,...,T_n \; \seq$}
\AxiomC{$T_n\equiv \forall X.T$}
\LeftLabel{\scriptsize (Inst)}
\BinaryInfC{$\Delta \vdash_{n+1} T_0,...,T_n, T\subst{S}{X}\seq$}
\end{prooftree}

\begin{prooftree}
\AxiomC{$x:T\in \Delta$}
\LeftLabel{\scriptsize (Hyp)}
\UnaryInfC{$\Delta \vdash_0 x : T$}
\end{prooftree}

\begin{prooftree}
\AxiomC{$\Delta,x : S \vdash_n t:T$}
\LeftLabel{\scriptsize ($\to_i$)}
\UnaryInfC{$\Delta \vdash_0 \lterm{x}{t}:S\to T$}
\end{prooftree}

\begin{prooftree}
\AxiomC{$\Delta \vdash_p e:S\to T$}
\AxiomC{$\Delta \vdash_q s:S$}
\LeftLabel{\scriptsize ($\to_e$)}
\BinaryInfC{$\Delta \vdash_0 e\; s : T$}
\end{prooftree}

\begin{prooftree}
\AxiomC{$\Delta \vdash_0 t:T$}
\AxiomC{$\Delta \vdash_n T_0,...,T_n \seq$}
\LeftLabel{\scriptsize (Sub)}
\BinaryInfC{$\Delta \vdash_n t : T_n$}
\end{prooftree}

\end{multicols}

\begin{prop}
$\Delta\vdash_{Curry} t:T$ si et seulement si il existe $n\geq 0$ tel que $\Delta \vdash_n t:T$
\end{prop}
\begin{proof}
Dans chacun des sens, par induction.
\end{proof}

\paragraph{Exo}
Si $\Delta\vdash_p t:T$ et $\Delta \vdash_p T_o,...,T_q \seq$ avec $T_0 \equiv T$, alors
\[\Delta \to_{p+q-1} t : T_q\]

\paragraph{Exo}
Si $\Delta \to_n T_0,...,T_n \seq$ avec $T_0\equiv U\to V$ et $T_n \equiv A\to B$ alors il existe $\bar{X}\subseteq \mathcal{V}$ et $\bar{S}$ de même longueur tel que
\[A\equiv U\subst{\bar{S}}{\bar{X}}, B \equiv V\subst{\bar{S}}{\bar{X}}\]


\begin{lemma}
\begin{enumerate}[label=\roman*)]
\item Si $\Delta\vdash_n t:T$, alors $\Delta\subst{S}{X}\vdash_n t:T{S}{X}$
\item Si $\Delta,x:S\vdash_n t:T$ et $\Delta\to_r s:S$ alors il existe $n'\geq 0$ tel que
\[\Delta \vdash_{n'} t\subst{s}{x}:T\]
\end{enumerate}
\end{lemma}

\begin{proof}
\begin{enumerate}[label=\roman*)]
\item facile
\item Considérons le cas où $t$ est une variable. Soit:
\begin{prooftree}
\AxiomC{}
\UnaryInfC{$\Delta,x:S \vdash_0 t:T$}
\end{prooftree}
\begin{itemize}[label=-]
\item Si $t\equiv x$, $S\equiv T$

Par ailleurs, $\Delta \vdash_r s: T$, du coup $\Delta \vdash_r t\subst{s}{x}: T$
\item Si $t\not\equiv x$, $t:T\in \Delta$ et 
\begin{prooftree}
\AxiomC{$x:T\in \Delta$}
\LeftLabel{\scriptsize (Hyp)}
\UnaryInfC{$\Delta \vdash_0 x : T$}
\end{prooftree}
\end{itemize}
\bigskip
Soit $n\geq 0$:
\begin{prooftree}
\AxiomC{$\Delta \vdash_0 t:T_0$}
\AxiomC{$\Delta \vdash_n T_0,...,T_n \seq$}
\BinaryInfC{$\Delta\vdash_n t:T_n$}
\end{prooftree}
Par hypothèse d'induction, il existe $n''\geq 0$ tel que $\Delta \to_{n''} t : T_0$. On en conclut $\Delta \vdash_{n''+n-1} t : T_n$
\end{enumerate}
\end{proof}

$\to$ La démonstration ``séquence'' est différente !

\begin{lemma}
Si $\Delta \vdash_n (\lterm{x}{a})\;b :T$, alors il existe $n,\geq 0$ tel que $\Delta\vdash_{n'} a\subst{b}{x}:T$
\end{lemma}
\begin{proof}
On se limite ici au cas où $n=0$:
\begin{prooftree}
\AxiomC{$\Delta \vdash_p \lterm{x}{a}:S\to T$}
\AxiomC{$\Delta \vdash_q b :S$}
\BinaryInfC{$\Delta\vdash_0 (\lterm{x}{a})\; b : T$}
\end{prooftree}

\begin{prooftree}
\AxiomC{$\square$}
\UnaryInfC{$\Delta, x: U \vdash_r a : V$}
\UnaryInfC{$\Delta \vdash_0 \lterm{x}{a}:U\to V$}
\AxiomC{$\Delta \vdash_p F_0,...,F_p \seq$}
\BinaryInfC{$\Delta\vdash_p \lterm{x}{a}: S\to T$}
\end{prooftree}

Avec $F_0 \equiv U\to V$ et $F_p \equiv S \to T$, donc il existe $\bar{X},\bar{S}$ tels que $S\equiv U\subst{\bar{S}}{\bar{X}}$ et $T\equiv\subst{\bar{S}}{\bar{X}}$ et $\bar{X}\cap FT(\Delta)=\emptyset$.

Par ailleurs,
\begin{prooftree}
\AxiomC{$\Delta \vdash_o b:S$}
\AxiomC{$\Delta \vdash_q S_0,...,S_q seq$}
\BinaryInfC{$\Delta \vdash_q b:S\qquad (S_q\equiv S)$}
\end{prooftree}

On en déduit :
\begin{itemize}[label=*]
\item $\Delta\subst{\bar{S}}{\bar{X}}, x: U\subst{\bar{S}}{\bar{X}} \vdash_r a : V\subst{\bar{S}}{\bar{X}}$\\
mais, puisque $\bar{X}\cap FV(\Delta)=\emptyset$
\[\Delta, x:S \vdash_r a:T\]
\item Finalement,
\[
\left.\begin{cases}
\Delta,x:S\vdash_r a:T\\
\Delta \vdash_q b : S\\
\end{cases}\right\rbrace \Rightarrow \Delta\vdash_{r'} a\subst{b}{x}:T\]
\end{itemize}
Pour un certain $r'\geq 0$
\end{proof}

\begin{prop}[SR]
Si $\Delta \vdash_n t: T$ et si $t \; \beta_0 \; t'$, alors il existe $n'\geq 0$ tel que $\Delta \vdash_{n'}t':T$
\end{prop}
\begin{proof}
On a vu le cas de base:
\begin{itemize}[label=*,noitemsep]
\item $\to_\beta \subseteq \beta_0$
\item On regarde les autres cas qui définissent $\beta_0$
\end{itemize}
\end{proof}

\begin{thm}[SR]
Si $\Delta\vdash_{Curry} t:T$ et si $t\; \beta_0 \; t'$, alors $\Delta \vdash_{Curry} t': T$
\end{thm}

\paragraph{Remarque} Dans (SU)\footnote{Bouquin de référence}, on utilise $\sigma \preceq \tau$ avec $\sigma,\tau$ types.

\part{Égalité}
\setcounter{section}{-1}
\section{Usage des inductifs dans Coq}
\subsection{Les booléens}
$\mathbf{bool}\equiv \{\mathbf{true}, \mathbf{false}\}$

En Coq, \\
\texttt{Inductive bool : Type :=\\
| true $\hspace{410px} (\star)$\\
| false}\\
correspond à la déclaration de $\mathbf{true}$ et $\mathbf{false}$ comme éléments de type $\mathbf{bool}$.


Ce qui ``correspond'' à l'énoncé d'un principe d'induction sur $\mathbf{bool}$ :\\
Pour tout $P : \mathbf{bool} \to \mathbf{True}$, si $\begin{cases}
(P\; \mathbf{true}) \;\text{est habitable}\\
(P\; \mathbf{false}) \;\text{est habitable}
\end{cases}$, alors pour tout $b$ $\mathbf{bool}$, $P\; b$ est habitable.
\bigskip

Dans Coq, ($\star$) engendre:
\begin{itemize}[noitemsep,label=-]
\item un nouveau type $\mathbf{bool}$
\item Les constructeurs $\mathbf{true}$ et $\mathbf{false}$
\item Le principle d'induction, de type $\Pi \; P : \mathbf{bool} \to Type$
\end{itemize}

\subsection{Les entiers de Peano}
\texttt{
nat : Type\\
0 : nat\\
S : nat $\to$ nat
}
\bigskip

\texttt{
Inductive nat : Type :=\\
| 0 : nat\\
| S : nat $\to$ nat}
\bigskip

Ce qui correspond à l'introduction d'un entier, et au principe de récurrence ``standard''.




\subsection{Le produit $A\times B$}
\texttt{
Inductive prod (A B : Type) : Type :=\\
pair : A $\to$ B $\to$ prod A B\\
}
\bigskip

\texttt{
Inductive sum (A B : Type) : Type :=\\
| inl : A $\to$ sum A B\\
| inr : B $\to$ sum A B
}

\section{Types dépendants}
$\to (\Pi\; x : A)\; B(x)$ ``$\forall x\in A. B(x)$''\\
$ \to (\Sigma \; x.A)\; B(X)$ ``$\exists x\in A . B(x)$''


\paragraph{Terminologie}
\begin{itemize}[label=-]
\item On a des constantes d'\emph{univers} : $Type_i, i\in \mathbb{N}$, noté ``$\mathcal{U}$'' $\to$ Ce sont des types

\item Un objet T est est un \emph{type} s'il existe un univers $\mathcal{U}$ tel que $T : \mathcal{U}$

\item Un objet $t$ est un terme s'il existe un type $T$ tel que $:T$ (en particulier, on a toujours $Type_i : Type_{i+1}$, donc $\forall i \in \mathbb{N}, Type_i : \mathcal{U}$
\end{itemize}

\section{Egalité (prélude)}
Dans Coq : 
\begin{itemize}[label=$\star$]
\item égalité de Leibniz.\\
$a, b : A$\\
``$a=b$'' si pour tout $P: A \to Type$, $P\; a \to P \; b$.
\[ a=b \equiv (\Pi A : A \to W) \underbrace{P\; a}_{\mathcal{U}} \to \underbrace{P\; b}_{\mathcal{U}} \;: \;\mathcal{U} \]
\item Type identité (Martin-Löf)\\
$A\; : \; \mathcal{U}, \qquad a,b \;:\; A\qquad Id_A (a,b)\; :\; \mathcal{U}$\\
Peut être présenté à l'aide des \emph{inductifs} de Coq :\\
\texttt{
Inductive \{Id A : Type\} (x, y :A) := \\
 | Id\_reflexive : forall a : A, Id a a.\\
 \textit{(** Ou @Id A a a pour forcer le premier argument**)}
}\\
Qui fournit (entres autres):
\begin{itemize}
\item Le constructeur $Id\_reflexive$ $\rightsquigarrow$ règle d'introduction
\begin{prooftree}
\AxiomC{$A\; : \; Type$}
\AxiomC{$a\;:\; A$}
\BinaryInfC{$Id\_reflexive \; a\; : \; Id_A(a,a)$}
\end{prooftree}
\item Un principe de raisonnement inductif $\rightsquigarrow$ règle d'élimination (faire ``
\texttt{Print Id\_rect}'')\\
Pour tout $A\;:\; Type$, $C\; : \: \prod_{a,b : A} Id_A(a,b) \to \mathcal{U}$, si pour tout $a\;:\;A, C\; (a,a,Id\_reflexive\; a)$ est prouvable, alors pour tout $a,b\;:\; A$, pour tout $p:Id_A(a,b),C(a,b,p)$ est prouvable.

$ind_{=_A}(C,c,a,b,p)\;:\; C(a,b,p)$ avec $p\;:\; Id_A(a,b)$ vérifie $ind_{=_A}(C,c,x,x,Id\_reflexive\; x)\equiv C\; x$\\
($\beta$-règle, en fait $i$-règle, pour $I$ inductif !
\end{itemize}
\end{itemize}

\paragraph{Remarques}
Si $a\equiv b$ dans $A$, alors $a=_A b$ via $Id\_reflexive$ (i.e. $Id_A(a,b)$ prouvable) ! ``$\equiv$'' est l'égalité définitionnelle, et ``$=_A\; (Id_a(\_,\_))$'' est l'égalité propositionnelle.

\paragraph{Exemple}
soit $f: \left. \begin{cases}
\mathbb{N}\to \mathbb{N}\\
a \mapsto x\star x\\
\end{cases}\right) \lterm{x}{x\star x}$
\begin{align*}
f(3) & \equiv (\lterm{x}{x\star x})\; 3\\
(\lterm{x}{x\star x}) \; 3 \equiv 3 \star 3\\
\end{align*}
Mais on n'a pas toujours $3 \star 3 \equiv 9$ ! Par contre, $3 \star 3 =_{\mathbb{N}} 9$ ? $\to \; Id_{\mathbb{N}} (3\star 3, 9)$ ?

%\paragraph{Idée}
%$x,y : A \vdash Id_a (x,y) : Type$ ``$x=_A y$''
%d'où la règle de formation :
%\begin{multicols}{2}
%\begin{prooftree}
%\AxiomC{$A: Type$}
%\AxiomC{$a: A$}
%\AxiomC{$b:A$}
%\RightLabel{\scriptsize (Id-form)}
%\TrinaryInfC{$Id_A (a,b) : Type$}
%\end{prooftree}
%
%\begin{prooftree}
%\AxiomC{$a: A$}
%\RightLabel{\scriptsize (Id-intro)}
%\UnaryInfC{$refl_a : Id_A(a,a)$}
%\end{prooftree}
%\end{multicols}
%
%Et rien d'autre ! Dans Coq:\\
%\texttt{
%Inductive Id (A : Type) : A $\to$ A $\to$ Type :=\\
%IA\_reflexive : forall x:A Id A x x
%}
%
%
%
%\begin{prooftree}
%\AxiomC{C : ($\Pi$ x,y : A)($\Pi$ p : Id$_A$ x y) Type}
%\AxiomC{x : A $\vdash$ c : C x x (refl$_x$)}
%\AxiomC{a : A}
%\AxiomC{b : A}
%\AxiomC{p : Id$_A$ a b}
%%\RightLabel{\scriptsize (Id-elm)}
%\QuinaryInfC{Ind$_{Id_A}$ C (\tlambda x.c) a b p : C a b p}
%\end{prooftree}

\section{Question d'égalité}
\subsubsection*{En Logique}
Si $P,Q$ sont énoncés logiques,
\[P \Leftrightarrow Q \equiv (P\Rightarrow Q) \land (Q \Rightarrow P)\]
Relation d'équivalence. Est-ce une notion d'égalité ? $\to$ pas toujours en logique intuitionniste.
\subsection*{Isomorphisme en math}
$A \simeq B \qquad f: A \to B \qquad g : B \to A$\\
$g \circ f = id_A \;;\; f\circ g = id_B$

\section{Mise en oeuvre de $Id_A(\_,\_)$}
\subsection{}
\begin{lemma}
$Id_A(\_,\_)$ est une relation d'équivalence.
\end{lemma}

\begin{proof}
\begin{enumerate}[label=\roman*)]
\item Pour tout $a : A$, $Id_A(a,a)$ par application de $Id\_reflexive$ ``par réflexivité''
\item Pour tout $a,b : A$, $Id_A(a,b)\to Id_A(b,a)$
\[ \prod_{a,b : A}\prod_{p:Id_A (a,b)} Id_A(b,c)\]

On raisonne par induction sur le chemin $p:Id_A(a,b)$:

Il suffit de considérer le cas de base, c'est-à-dire lorsque $b$ est $a$, et $p$ est $refl_a$ (alors $Id\_reflexive$). Dans ce cas, on doit prouver 
\[Id_A(a,a)\]
Il suffit de fournir $refl_a$.
\item pour tout $a,b,c : A,\qquad p:Id_a(a,b),\qquad q:Id_A(b,c)$ on a $Id_A(a,c)$

On raisonne par induction sur $p$:

Le cas de base correspond à $b$ est $a$ et $p$ est $refl_a$. Dans ce cas, il faut prouver $\prod_{q:Id_a(a,e)} Id_a (a,e)$. On raisonne par induction sur $q$:

Le cas de base : $c$ est $a$, et $q$ est $refl_a$ : $refl_a\; : \; Id_a(a,a)$
\end{enumerate}
\end{proof}

On interprète $Id_A(a,b)$ comme un chemin de $a$ à $b$ : on obtient une structure de groupoïde.

\begin{prop}
\begin{enumerate}[label=\roman*)]
\item $p=p.refl_y$ et $p=refl_x.p$
\item $p^{-1}.p=refl_x$ et $p.p^{-1}=refl_y$
\item $(p^{-1})^{-1}=p$
\item $p.(q.r)=(p.q).r$
\item $(refl_x)^{-1} = refl_x$
\end{enumerate}
\end{prop}

\subsection{Transport de l'égalité}
\begin{lemma}
$f : A \to B, x,y :A$ et $p:x=_A y$ (elim $Id_A(x,y)$)
Alors il existe un terme
\[ap_f(p) : f(x) =_B f(y)\]
tel que pour tout $x:A, ap_f (refl_x) \equiv refl_{f(x)}$
\end{lemma}

\begin{proof}
Par induction sur la classe $p: x=_A y$, if suffit de considérer le cas de base : $y$ est $x$ et $p$ est $refl_x$. Alors $refl_{f(x)} =_B f(x)$
%TODO


\end{proof}


Dans le cas dépendant:
\begin{lemma}

$f : \prod_{x:A} P(x), \qquad x,y :A$ et $P:x=_A y$. Alors il existe un chemin noté $P_\star : P(x) \to P(y)$ tel que $(refl_x)_\star \equiv refl_{P(x)}$.
\end{lemma}

\end{document}