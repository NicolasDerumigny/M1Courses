\documentclass{article}

\usepackage[utf8]{inputenc}
\usepackage[english]{babel}
 
\usepackage{caption}
%\usepackage{pgfplots}
\usepackage{listings}
\usepackage{graphicx}
\usepackage{footnote}
\usepackage{amsmath}
\usepackage{amsthm}
\usepackage{graphicx}
\usepackage{url}
\usepackage{amssymb}
\usepackage{mathrsfs}
\usepackage{multirow}
\usepackage{amsfonts}
\usepackage[boxed,linesnumbered,noend]{algorithm2e}
\usepackage{qcircuit}
\usepackage{bussproofs}
\usepackage{enumitem}
\usepackage{multicol}

\newtheorem{thm}{Theorem}
\newtheorem{prop}{Propriety}
\newtheorem{lemma}{Lemma}
\newtheorem{defi}{Definition}
\newtheorem{coro}{Corollary}



\setlength{\oddsidemargin}{0pt}
% Marge gauche sur pages impaires
\setlength{\evensidemargin}{0pt}
% Marge gauche sur pages paires
\setlength{\textwidth}{470pt}
% Largeur de la zone de texte 
\setlength{\topmargin}{0pt}
% Pas de marge en haut
\setlength{\headheight}{13pt}
% Haut de page
\setlength{\headsep}{10pt}
% Entre le haut de page et le texte
\setlength{\footskip}{40pt}
% Bas de page + séparation
\setlength{\textheight}{630pt}
% Hauteur de la zone de texte

\title{Proofs and Programs}
\author{Phillipe Audebaud \footnote{https://perso.ens-lyon.fr/philippe.audebaud/PnP/}\\
\small ENS de Lyon}
\date{}

\newcommand\subst[2]{\langle #1\big/#2\rangle}
\newcommand\lterm[2]{\lambda #1. #2}
\newcommand\Lterm[2]{\Lambda #1. #2}
\newcommand\tlambda[0]{$\lambda$}

\begin{document}

\maketitle
\tableofcontents
\newpage

\subsection*{Basis}
\begin{itemize}
\item Lecture: Tue 8h-10h (Philippe Audebaud)
\item Tutorial: We 8h-10h (Aurore)
\end{itemize}
10 Weeks of courses (3x3), which is really low.

\[Final\;mark = 50\% \cdot CC + 50\% \cdot Exam\]
No mid-time exam, but weekly homework.

\paragraph{Warning} Presence at the courses and tutorial will have an impact on the marks.

\paragraph{Prerequisites}
\begin{itemize}
\item L2.2 $\to$ Logical (Natacha P., Chapter 1 \& 2): 
\begin{itemize}
\item Proof theory
\item Formal system for logic inference.
\end{itemize}
\item $\lambda$-calculus
\item Category theory
\end{itemize}

\setcounter{section}{0}
\part{(Pure) $\lambda$-Calculus}
\subsection{Computing with functions ?}
How do we do mathematics ?
\begin{enumerate}[label=\Alph*.]
\item Having \emph{structures:} numbers, spaces (points, vectors, functions) $\to$ Eilenberg-Mac Lane ($\sim$ 1942) Category theory
\item Build, explore, transform structures $\to$ Church ($\sim$ 1930) \emph{$\lambda$-Calculus}
\item Compare "stuff": \emph{equality} $\to$ Voevoski ($\sim$ 2006) Algebraic topology $\to$ search HoTT (Hight order Type Theory)
\item Provide a framework (\emph{rules}) to reasoning on all that! $\to$ 1st point
\end{enumerate}


\subsection{Church $\lambda$-calculus (informally)}
\begin{align*}
f: &A \to B\\
& x \mapsto e
\end{align*}
Given $a\in A$, $f(a)$ is the "replacement of the occurrence of $x$ in $e$ by a"

\begin{align*}
f & \overset{\text{def}}{=} \lambda x.e \tag{$\lambda$-abstraction}\\
f\;a & = (\lambda a.e) \;a \tag{Application}\\
\end{align*}

\paragraph{Notation}
\[e\subst{a}{x}\] is the replacement in $e$ of all the occurrences of $a$ by $x$.

\paragraph{Example}
\begin{enumerate}
\item \begin{align*}
&\lambda x. x\\
&x\mapsto x
\end{align*}
is the identity function
\item \begin{align*}
&\lambda x.y\\
&x\mapsto y
\end{align*}
Here $x$ and $y$ are variables, $x\neq y$. $(\lambda x.y)\; a$ leads to $y\subst{a}{x}\equiv y$

\[ (\lambda x.a) \; b \to_\beta a\subst{b}{x}\]
$\to_\beta$ is a binary relation on lambda-terms $\Rightarrow$ idea of computation on terms.

\end{enumerate}

\paragraph{Notion of $\alpha$-equivalence}
\[\lambda x.a \overset{?}{=}_\alpha \lambda y.b\]
Pick a \emph{fresh} variable, let say $z$,
\[a\subst{z}{x} =_\alpha b\subst{z}{y}\]

\emph{All the results and proofs will be done up to  $\alpha$-equivalence (no difference made between $\lambda x.x$ and $\lambda y.y$).}


\section{A toolbox on $\lambda$-calculus}
\emph{$\lambda$-calculus: Syntax and Semantics}, Herk Barendregt (1977)
\bigskip


Let $\mathcal{X}$ be a measurable set of variables, ranged over by $x,y,z,...$

\begin{defi}
A $\lambda$-term $e$ is generated by the grammar:
\[ a,b,e ... ::= x\in \mathcal{X} \;|\; \lambda x.e \; | \; a \; b\]

The set of $\lambda$-terms is denoted $\Lambda$.
\end{defi}


\begin{defi}[Free variable]
The set of free variables in $e$, denoted $FV(e)$ is defined inductively:
\begin{itemize}
\item if $e\equiv x\in \mathcal{X},\; FV(x)\equiv\{x\}$
\item if $e\equiv \lambda x . a_0, \; FV(\lambda x.a_0) \equiv FV(a_0)\setminus \{x\}$
\item if $e\equiv a_1 \; a_2, \; FV(a_1\;a_2)\equiv FV(a_1) \cup FV(a_2)$
\end{itemize}
A term $e$ is \emph{closed} if $FV(e)=\emptyset$
\end{defi}

\begin{defi}[Substitution]
Given $x \in \mathcal{X}$, $a\in \Lambda$, the substitution of (all the) occurrences of $a$ in $e\in \Lambda$, denoted $e\subst{a}{x}$ is:
\begin{itemize}
\item if $y\in \mathcal{X} \setminus \{x\}, \; y\subst{a}{x}\equiv y $ and $x\subst{a}{x}\equiv a$
\item $(\lambda y.e)\subst{a}{x} = \lambda y.e \subst{a}{x}$
\item $(e\; f)\subst{a}{x} = (e\subst{a}{x})\; f\subst{a}{x} $
\end{itemize}
\end{defi}

\begin{defi}[$\to_\beta$ reduction]
\begin{align*}
\to_\beta & \subseteq \Lambda \times \Lambda\\
& \Big\{ \big( \underbrace{(\lambda x.a)\; b}_{\text{redex}},\underbrace{a\subst{b}{x}}_{\text{contraction}}\big)\big| x\in \mathcal{X},\; a,v \in \Lambda \Big\}
\end{align*}
\end{defi}

\paragraph{Example}
\begin{enumerate}
\item \begin{align*}
(\underbrace{\lambda x. (\lambda y.y)\; a}_{\nrightarrow_\beta (\lambda y.y)\; b})\; b) \to_\beta ((\lambda y.y)\; a)\;\subst{b}{x} & \equiv ((\lambda y.y)\;\subst{b}{x})a\subst{b}{x}\\
& \equiv (\lambda y.y)\; a\subst{b}{x}
\end{align*}

\item \begin{align*}
(\lambda x.y)\; a \to_\beta y\\
\end{align*}

\item \begin{align*}
(\lambda x.x\; x)(\lambda x.x \;x) \to_\beta & (x\; x)\subst{\lambda x.x \; x}{x} \text{ or } (x\; x)\subst{\lambda y.y \; y}{x}\\
& (\lambda x. x\;x)(\lambda x.x\; x)
\end{align*}
Russell paradox: we get an infinite $\beta$-reduction!
\end{enumerate}

\[\to_\beta \; \subseteq \; \beta_0 \; \subseteq \; \underbrace{\beta}_{\beta-\text{reduction}} = \beta_0^* \]
\[\to_\beta^* \text{ is the $\beta$-reduction, noted } \twoheadrightarrow_\beta \]

\begin{defi}[$\beta_0$-contraction]
Let $a,b\in \Lambda$. $a\; \beta_0 \; b$ is defined by cases:
\begin{itemize}
\item $x\; \beta_0 \; x$
\item $(\lambda x.u)v\; \beta_0 \; u\subst{v}{x}$
\item $(\lambda x.u)\; \beta_0 \; (\lambda x.v)$ if $u\; \beta_0 \; v$
\item $(u\; v)\; \beta_0 \; (u'\; v)$ if $u\;\beta_0\;u'$
\item $(u\; v)\; \beta_0 \; (u\;v')$ if $v\; \beta_0 \; v'$
\end{itemize}
\end{defi}

\noindent\textit{Maintenant en français !}

\paragraph{Remarque:} $\beta_0$ est réflexive.

\begin{defi}
La \emph{$\beta$-réduction} est la clôture transitive de $\beta_0$:
\[\beta = \beta_0^* \]
\end{defi}


\paragraph{Remarque}
Si $a,b \in \Lambda$, alors $a \; \beta \; b$ si il existe $n\geq 0$ et $(e_k)_{0\leq k \leq n}$ $\lambda$-termes tels que :
\begin{itemize}
\item $a=e_0$ et $b=e_n$
\item pour tout $k<n$, $e_k \; \beta_0 \; e_{k+1}$
\end{itemize}

\begin{defi}
Soit $\mathcal{R}$ une relation binaire sur $\Lambda$. On dit que $\mathcal{R}$ est \emph{$\lambda$-compatible} si elle satisfait les propriétés suivantes :
\begin{itemize}
\item $x\;\mathcal{R}\; x$
\item si $a \; \mathcal{R} \; b$ et $c \;\mathcal{R}\; d$ alors $a\; c\; \mathcal{R} \; b \; d$
\item si $a\; \mathcal{R}\; b$ alors $\lambda x.a\; \mathcal{R}\; \lambda x.b$
\end{itemize}
\end{defi}

\begin{prop}
La $\beta$-réduction est la plus petite relation $\lambda$-compatible et transitive contenant $\to_{\beta}$
\end{prop}

\begin{proof}
$\star$ On vérifie d'abord :
\[\to_\beta \; \subseteq \beta_0 \subseteq \beta_0^* = \beta \]

D'autre part, $\beta_0$ est $\lambda$-compatible :
\begin{itemize}
\item par réflexivité, $x \; \beta	\; x$
\item soit $a\; \beta \; b$ ; par définition,il existe $n\geq 0$, $(e_k)_{0\leq k \leq n}$ tel que $a=e_0$, $b=e_n$ et pour tout $k<n$, $e_k \; \beta_0 \; e_{k+1}$. Du coup, par définition de $\beta_0$, pour tout $k<n$, $\lterm{x}{e_k} \; \beta_0 \; \lterm{x}{e_{k+1}}$.

Ainsi, $\lterm{x}{a} \; \beta \; \lterm{x}{b}$.
\end{itemize}

$\star$ Soit $\mathcal{R}$ une autre relation $\lambda$-compatible et transitive contenant $\to_\beta$. Montrons que $\beta\subseteq \mathcal{R}$. Il "suffit" de vérifier que $\beta_0 \subseteq \mathcal{R}$ (laissé en exercice).
\end{proof}

\subsection*{Propriétés essentielles de la $\beta$-réduction}
\paragraph{Remarque}
$(\Lambda, \beta_0)$ est un \emph{système de réduction abstrait\footnote{Cf ThPr}}.

\begin{defi}[Forme normale, Relation normalisante]
Soit $\mathcal{R}$ une relation binaire sur $\Lambda$,
\begin{itemize}
\item On dit que $a$ \emph{est} une forme normale (relativement à $\mathcal{R}$) s'il n'existe pas $b\in \Lambda$ tel que $a\; \mathcal{R} \; b$.
\item On dit que $a$ \emph{a} une forme normale (relativement à $\mathcal{R}$) s'il existe $b\in \Lambda$ tel que $b$ \emph{est} une forme normale et $a \; \mathcal{R}^*\;b$
\item On dit que $\mathcal{R}$ est normalisante si tout $a \in \Lambda$ a une forme normale
\end{itemize}
\end{defi}


\paragraph{Exemple}
\begin{itemize}
\item $\lterm{x}{x}$ est une forme normale relativement à $\beta_0$
\item $\beta_0$ n'est pas normalisante !
\begin{align*}
\Omega & \equiv (\lterm{x}{x\; x})\; (\lterm{x}{x\; x})\\
\Omega & \to_\beta \Omega
\end{align*}
\end{itemize}

\begin{defi}[Confluence]
Soit $\mathcal{R}$ une relation binaire sur $\Lambda$. On dit que $\mathcal{R}$ est confluente si pour tout $(a,b,c) \in \Lambda^3$ tel que 
\[a \; \mathcal{R}^* \; b \text{ et } a\; \mathcal{R}^*\; c\]
alors il existe $d\in \Lambda$, tel que
\[b \; \mathcal{R}^* \; d \text{ et } c \; \mathcal{R}^* \; d\]
\end{defi}

\begin{thm}
La $\beta_0$-réduction est confluente.
\end{thm}
\begin{proof}
En semaine 3 ou 4.
\end{proof}

\begin{coro}
Tout $\lambda$-terme admet au plus une forme normale, relativement à $\beta_0$
\end{coro}

\subsection*{Notion d'égalité sur les \tlambda-termes}

\begin{defi}
La $\beta$-équivalence sur $\Lambda$ est la relation binaire notée $=_\beta$, définie comme la clôture réflexive symétrique transitive de $\beta_0$ :\\
$a=_\beta b$ s'il existe $n\geq 0$ et $(e_k)_{0\leq k \leq n}$ tel que $a=e_0$ et $b=e_n$ et $\forall k<n$, soit $e_k \; \beta_0 \; e_{k+1}$ soit $e_{k+1} \; \beta_0 \; e_{k}$
\end{defi}

\begin{defi}[\tlambda-congruence]
Une relation binaire $\mathcal{R}$ (sur $\Lambda$) est une \tlambda-congruence si c'est une relation d'équivalence et qu'elle est \tlambda-compatible.
\end{defi}

\begin{thm}[Church-Rosser]
Pour tout $(a,b)\in \Lambda^2$, $a=_\beta b$ si et seulement si il existe $c\in \Lambda$ tel que $a \; \beta \; b$ et $b \; \beta \; c$
\end{thm}

\begin{proof}
La condition est suffisante
\bigskip

Réciproquement, pour la condition nécessaire, on introduit $R\subseteq \Lambda \times A$ défini par :\\
$a \; \mathcal{R} \; b$ s'il existe $c$ tel que $a\; \beta \; c$ et $b \; \beta \; c$.

On remarque, par définition de $\mathcal{R}$,
\begin{itemize}
\item $\mathcal{R}$ est réflexive et symétrique
\item $\mathcal{R}$ est transitive
\end{itemize}
De plus, $\mathcal{R}$ contient $\beta$ (ou $\beta_0$). Donc, si $a=_\beta b$, alors $a\; R \; b$.
\end{proof}

\begin{thm}
La relation d'équivalence $=_\beta$ est la plus petite \tlambda-congruence contenant $\to_\beta$
\end{thm}
\begin{proof}
En exo.
\end{proof}



\paragraph{Notation} On note $\equiv$ pour une définition ($\overset{\text{def}}{=}$), mais aussi pour l'$\alpha$-équivalence ($=_\alpha$). On peut utiliser la notation de Bruijn (cf références).


\section{Calcul propositionnel et correspondance de Curry-Howard}
\subsection{Éléments de langage (informels)}
\begin{itemize}
\item Théorie de la \emph{démonstration} (prouvabilité)
\item Thème des modèles
\end{itemize}

Quelques "ingrédients" :
\begin{itemize}
\item \emph{énoncés} (logiques) : ici les familles du calcul propositionnel:
\[A ::= x \;|\: \top \:|\; \bot \;|\; A\Rightarrow B \;|\; A\land B \;|\; A \lor B \;|\; \neg A\footnote{$\neg A$ signifie en fait $A \Rightarrow \bot$} \label{eq:star}\tag{$\star$}\]
La notation "$A$ propriété" signifie que $A$ est engendrée par la grammaire (\ref{eq:star})
\item On parle de \emph{jugements} sur ces énoncés : "$A$ true" 
\item On introduit aussi des jugements hypothétiques : 
$A_1$ true, $A_2$ true, ... ,$A_n$ true $\vdash$ $B$ true
\end{itemize}


Commentaires sur les différentes règles de (NJ):
\begin{itemize}
\item Le vrai
\item L'implication (/!\textbackslash: $A \Rightarrow B \neq \neg A \lor B$ dans (NJ))
\item Le faux
\item La négation
\item La disjonction
\end{itemize}


\subsection{Fragments $\lambda_\to$}
On peut associer des règles au typages de \tlambda-termes en raisonnant sur $\lterm{x}{t} : T$


\begin{thm}[Curry-Howard]
Le fragment $NJ_\to$ et $\lambda_\to$ sont en correspondance via:
\begin{enumerate}
\item Si $\Delta \vdash t : T$ dans $\lambda_\to$
%TODO
\end{enumerate}
\end{thm}

\begin{figure}[h]
\begin{center}
\begin{tabular}{| c | c |}
\hline
$\lambda_\to$ & $NJ_\to$\\
\hline
\emph{type} & \emph{proposition}\\
\hline
variable de type & proposition atomique\\
type flèche & implication\\
\hline
\emph{terme} & \emph{dèrivation}\\
\hline
variable de terme & hypothèse\\
\tlambda-abstraction & règle d'introduction\\
application & règle d'élimination\\
$\beta$-redex & coupure\\
$\beta$-réduction & transformation sur les dérivations\\
\hline
\emph{forme normale} & \emph{dérivation sans coupure}\\
\hline
\end{tabular}
\caption{Correspondance de Curry-Howard}
\end{center}
\end{figure}


\subsection{Interprétation BHK}
L'interprétation de Brouwer-Heyting-Kolmogorov consiste à construire un témoin (une preuve) d'une proposition selon le protocole suivant :
\begin{itemize}
\item Un témoin pour $A \land B$ est une paire formée par un témoin pour $A$ et un témoin pour $B$
\item Il y a un témoin unique pour $\top$
\item Un témoin pour $A\lor B$ est soit un témoin pour $A$, soit un témoin pour $B$
\item Il n'y a pas de témoin pour $\bot$
\item Un témoin pour $A\Rightarrow B$ est une application de témoins pour $A$ vers des témoins pour $B$
\item Un témoin pour $\neg A$ est un témoin de $A \Rightarrow \bot$
\end{itemize}
Avec $A, B$ engendrés par la grammaire
\[A ::= X \;|\: A \Rightarrow A \;|\; A \lor A \;|\; A \land A \;|\; \top \;|\; \bot\]

\begin{defi}[Produit (paire)]
Soit A,B. Le produit de $A$ par $B$ est le damier de $A\times B$, et de la propriété universelle suivante :\\
Pour tout $f : D \to A$ et $g : D\to B$, il existe $h : D\to A\times B$ tel que $\pi_1 \circ h = f$ et $\pi_2 \circ h = g$\footnote{Ces égalité correspondent à des $\beta$-réduction dans le \tlambda-calcul}.

De plus, $h$ est \emph{unique} et ne dépend que de $f$ et de $g$,
\[h = \langle f, g \rangle\]
\end{defi}

Par ailleurs, si $e : D \to A \times B$, alors 
\[
\begin{cases}
\pi_1 \circ e : D \to A\\
\pi_2 \circ e : D \to B\\
\end{cases}
\]
Pour ce couple, il existe $\langle \pi_1 \circ e, \pi_2 \circ e\rangle : D \to A \times B$

Du coup, par unicité, on a nécessairement
\[\langle \pi_1 \circ e, \pi_2 \circ e\rangle = e \]

Cette observation donne lieu à :
\begin{itemize}
\item une transformation sur les dérivations
\item une autre forme de réduction sur les \tlambda-termes
\end{itemize}
On parle alors d'$\eta$-réduction.

\bigskip

On rajoute alors les règles de typage du produit ($\times_i$) et ($\times_{E,k}$) pour $k\in \{1,2\}$.


\begin{defi}[Somme (coproduit)]
Soit $A,B$. C'est la donné de $A + B$ avec la propriété universelle suivante :\\
Si $f : A \to C$, et $g : B \to C$, il existe $k : A + B \to C$ unique, ne dépendant que de $f$ et de $g$ noté $\{ f, g \}$, tel que 
\[
\begin{cases}
k \circ in_l = f\\
k \circ in_r = g
\end{cases}
\]

Par ailleurs, si on se donne
\[e : A + B \to C\]
Alors
\[
\begin{cases}
e \circ in_l : A \to C\\
e \circ in_r : B \to C
\end{cases}
\]
Donc 
\[\{ e \circ \in_l, e \circ in_r\} = e\]
\end{defi}

On rajoute alors trois règles : (+Ig), (+Id) et (+E)




\section{$\lambda$-calcul simplement typé}
\subsection*{Quelques Lemmes}

\begin{lemma} 
Si $\Delta \vdash t : T$ clos, $FV(t) \subseteq FV(\Delta)$, où $FV(\emptyset)=\emptyset$, et $FV(\Delta,x : S) = FV(\Delta)\cup \{x\}$.
\medskip

\emph{Attention} : Un contexte de typage $\Delta \equiv x_1 : S, ..., x_p : S_p$ où $p\geq 0$ est valide si les variables $x_1, ... ,x_p$ sont distinctes deux à deux.
\end{lemma}

On peut rajouter des règles sur la validité de $\Delta$ en tant que contexte.

\begin{prooftree}
\AxiomC{}
\AxiomC{}
%\RightLabel{\scriptsize Rule Name}
\BinaryInfC{$\emptyset$ contexte valide}
\end{prooftree}

\begin{prooftree}
\AxiomC{$\Delta$ contexte valide}
\AxiomC{$T$ type}
\AxiomC{$x\notin FV(\Delta)$}
%\RightLabel{\scriptsize Rule Name}
\TrinaryInfC{$\Delta, x : T$ contexte valide}
\end{prooftree}

Et on augmente (Hyp).

\begin{prooftree}
\AxiomC{$\Delta$ contexte valide}
\AxiomC{$x : T \in \Delta$}
\RightLabel{\scriptsize (Hyp)}
\BinaryInfC{$\Delta, x : T$ contexte valide}
\end{prooftree}


\begin{lemma}[Affaiblissement]
Si $\Delta \vdash t : T$ et si $\Delta \subseteq \Delta'$, avec $\Delta'$ contexte valide, alors $\Delta' \vdash t : T$.
\end{lemma}

\begin{proof}
Par induction sur la dérivation principale, c'est-à-dire
$\Delta \vdash t : T$.
Le seul cas ``délicat'' est lorsque
\begin{prooftree}
\AxiomC{$\Delta,x:U \vdash a : V$}
\RightLabel{\scriptsize $\to_I$}
% TODO
\UnaryInfC{}
\end{prooftree}
\end{proof}


\begin{thm}
Si $\Delta \vdash t : T$, alor $t$ est fortement normalisant.
\end{thm}

\begin{proof}
Deux parties : poser la notation générale, puis l'adapter à $\to_\lambda$.

\begin{enumerate}
\item \emph{Définition générale :} Si $e \in \Lambda$, $e\equiv \lterm{\overline{x}}{\Delta \overline{u}}$ avec $|\overline{x}|>0$, $|\overline{u}|>0$ et $\Delta \in \mathcal{X}$ ou bien $\Delta$ est un $\beta$-redex
\begin{itemize}
\item $e$ est en forme normale si $\Delta \in \mathcal{X}$ et chaque $u_i$ est en forme normale
\item $e$ est une forme normale de tête (HNF) si $\Delta \in \mathcal{X}$
\item si $e$ n'est pas en HNF, c'est-à-dire $\Delta$ est un $\beta$-redex, $\Delta$ est appelé \emph{redex de tête}.
\end{itemize}
\end{enumerate}

\begin{defi}
$e \in \Lambda$ est fortement normalisant (SN) s'il n'existe pas de $\beta$-réduction infinie issue de $e$
\end{defi}

\paragraph{Exemple}
\begin{itemize}
\item $\Omega$ n'est pas SN
\item $(\lterm{x}{\lterm{y}{y}} \Omega$ n'est pas SN (il existe une dérivation infinie) $\to$ \emph{attention} : la $\beta$-équivalence n'est pas compatible avec la propriété d'être fortement normalisant.
\end{itemize}

Par contre, si $a=_\beta b$ et $b$ a une NF (resp HNF), alors $a$ a une NF (resp HNF)

De plus :
\begin{itemize}
\item Si $e$ a une NF (resp HNF), $\lterm{x}{e}$ a une NF (resp HNF)
\item Si $e$ est SN, $\lterm{x}{e}$ est SN
\end{itemize}
\bigskip

Soit $\mathcal{N}$ l'ensemble des termes SN, et $\mathcal{N}_0 \equiv \{  x\overline{u} \:|\: x\overline{u} \in \mathcal{N}\} \subseteq \mathcal{N}$
\paragraph{Notation}
\begin{itemize}
\item $e\in \Lambda$, $Succ(e)=\{e'\in \Lambda \:|\: e\beta_0 e' \}$, et cet ensemble est \emph{fini} (réduction à branchements fini)
\item Lemme de Koenig : si un arbre est infini et que cet arbre est a branchement fini, alors il existe un chemin infini
\end{itemize}

Si $e\in \mathcal{N}$, $\bigcup_{p\geq 0} Succ^p (e)$ est \emph{fini}, de sorte que la définition suivante est bien fondée :

\begin{defi}
Pour $e\in \mathcal{N}$, $\ell(e)$ désigne la somme des longueurs des chemins de tout réduction issue de $e$.
\end{defi}

\begin{lemma}
Sont immédiats :
\begin{itemize}
\item Si $e\in \mathcal{N}$, alors $\lterm{x}{e} \in \mathcal{N}$
\item Si de plus $e' \in \mathcal{N}$ et $e \beta e'$, alors $e' \in \mathcal{N}$
\item Si $e \in \Lambda$ tel que $Succ(e)\subseteq \mathcal{N}$, alors $e\in \mathcal{N}$
\end{itemize}
\end{lemma}
\begin{proof}
Pour le troisième point, soit $e\in \Lambda$ tel que $Succ(e)\subseteq \mathcal{N}$, pour tout $e' \in Succ(e)$, $\ell (e') < \ell(e)$ $\to$ une récurrence simple sur $\ell(e)$ permet d'établir $\mathcal{P}(e) \equiv$ ``$Succ(e) \subseteq \mathcal{N}$ implique $e\in \mathcal{N}$''
\end{proof}

\end{proof}




\part{Polymorphisme}
\section{Abstraction des types}

\begin{figure}[h]
\begin{center}
\begin{tabular}{c | c | c}
Programmation & Théorie des types & Raisonnement\\
\hline
\tlambda-calcul & $\lambda_\to$ & NJ($\Rightarrow$) \\
\tlambda-calcul enrichie & $\lambda_{\to,\times,\top\bot,...}$ & NJ\\
 &  $\lambda_\mu$ ($\sim$ 1990) & NK\\
Calcul de combinateurs (S,K,I) &  & Système de Hilbert (1900)\\
\hline
* & Système F & // \\
\hline // & ++ & //\\
\end{tabular}
\caption{Correspondance programmation - langage de preuve}
\end{center}
\end{figure}


\subsection{Motivations}
Prenons quelques exemples:
\begin{enumerate}
\item L'identité :\\
Dans $\lambda_\to$, $\vdash \lterm{x}{x} : T \to T$ pour n'importe quel type $T$.\\
On voudrait donner à $\lterm{x}{x}$ un type ``polymorphe''
\item Un entier de Church :\\
$\bar{z} = \lterm{x}{\lterm{f}{f (f x)}}$. Dans $\lambda_\to$, $\vdash \lterm{x}{\lterm{f}{f (f x)}} : A \to (A \to A) \to A$\\
$x:A, f:B \vdash f\;\underbrace{(f\;x)}_{:V} : C$
\begin{enumerate}[noitemsep]
\item $f:b \equiv U \to V$ et $x : U$ donc $A \equiv U$
\item $B \equiv V \to W$ et $W \equiv C$ et $U \to V \equiv V \to W$ d'où $U \equiv V$ et $V\equiv W$
\item $\Delta \equiv \lterm{x}{x\;x}$ qui nécessite $W=W\to V$.
\end{enumerate}
\end{enumerate}


\paragraph{Observations}
\begin{itemize}[label=$\star$]
\item On veut donner à un terme ``plusieurs types'' ``d'un coup ''
\begin{description}[noitemsep]
\item[En C] $\to$ template
\item[En O'Caml] $\to$ \texttt{fun x $\to$ x} : $'\alpha \to \alpha$
\end{description}
\item On peut avoir besoin de gérer plusieurs occurences d'une même variable
\item Observation de J.Reunolds (`74)
\item Introduit par JY Girard (`70) (cadre logique)
\end{itemize}

\subsection{Le système F à la Church}
Que veut-on ?
\begin{itemize}[noitemsep, label=$\star$]
\item Une notion de généralisation (d'abstraction) sur les types
\[T ::= X\in \mathcal{V} \;|\; T\to T \;|\; \forall X.T\]
\item Une notion \emph{d'instanciation} de type
\[ \forall X.T \rightsquigarrow T\subst{S}{X}\]
\end{itemize}
La conséquence sur les termes :
\[ t ::= x \in \mathcal{X} \; | \;  \lterm{x^T}{t}  \;|\; t\;t \;|\; \Lterm{X}{t} \;|\; t\; T \]

(Hyp), ($\to_{i}$)\footnote{où $\lterm{x^S}{t}$ précise le type de $x$} et ($\to_E$) de $\lambda_\to$ plus :

\begin{multicols}{2}
\begin{prooftree}
\AxiomC{$\Delta \vdash t:T$}
\AxiomC{$X \notin FT(\delta)$}
\RightLabel{\scriptsize ($\forall_i$)}
\BinaryInfC{$\Delta \vdash \Lterm{X}{t}:\forall X.T$}
\end{prooftree}

\begin{prooftree}
\AxiomC{$\Delta \vdash t: \forall X.T$}
\RightLabel{\scriptsize ($\forall_e$)}
\UnaryInfC{$\Delta \vdash t\subst{S}{X} : T\subst{S}{X}$}
\end{prooftree}
\end{multicols}

\paragraph{Retour sur les exemples}
\begin{enumerate}
\item $\vdash \lterm{x}{x} : X \to X$ devient $\Lterm{X}{\lterm{x^X}{x}} : \forall X\; X\to X$
\item $\bar{z} \equiv \lterm{x}{\lterm{f}{f\;(f\;x)}}$
\begin{prooftree}

\AxiomC{$\vdash \lterm{x^X}{\lterm{f^{X\to X}}{f\;(f\; x)}}: X \to (X\to X)\to X$}
\UnaryInfC{$\vdash\Lterm{X}{\lterm{x}{\lterm{f}{f\; (f\; x)}}}: \forall X.X\to (X\to X)\to X$}
\end{prooftree}

\item $\lterm{x}{x\;x}$
\begin{itemize}[noitemsep,label=$\star$]
\item $x\; U : U \to U$
\item $x\; V : V \to V$\\ avec $V\to V \equiv U$
\end{itemize}
On en déduit que $W\equiv V\to V$.
\begin{prooftree}

\AxiomC{$x\; U : U \to U$}
\UnaryInfC{$x:\forall X.S \vdash x\; U: (V \to V) \to W $}

\AxiomC{$x\; V : V \to V$}
\UnaryInfC{$x:\forall X.S \vdash x\; V : V\to V$}


\BinaryInfC{$x:\forall x :S \vdash x\; U \; (x\;V):W^{V\to V}$}
\UnaryInfC{$\vdash \lterm{x^{\forall X.S}}{x\; U \; (X\;V)}:(\forall X.S)\to W$}
\end{prooftree}

On trouve un type $((\forall X.(X\to X)\to \forall X.(X\to X) )\to ((\forall X.(X\to X)\to \forall X.(X\to X) )$


\subsubsection{Système F à la Curry}
\begin{itemize}[label=$\star$]
\item \tlambda-term pur
\item Même types que dans le système à la Church
\item (Hyp), ($\to_i$), ($\to_e$)
\newpage
\item Plus 
\begin{multicols}{2}
\begin{prooftree}
\AxiomC{$\Delta \vdash t : T$}
\AxiomC{$X\notin FT(\Delta)$}
\RightLabel{\scriptsize ($\forall_i$)}
\BinaryInfC{$\Delta \vdash t : \forall X.T$}
\end{prooftree}

\begin{prooftree}
\AxiomC{$\Delta \vdash t : \forall X.T$}
\RightLabel{\scriptsize ($\forall_e$)}
\UnaryInfC{$\Delta \vdash t : T\subst{S}{X}$}
\end{prooftree}

\end{multicols}
\end{itemize}


\end{enumerate}


\subsubsection{Aspects dynamiques}
\begin{enumerate}[label=\alph*.]
\item A la Church :
\begin{itemize}[label=$\star$]
\item $(\lterm{x^T}{a})\; b \to_\beta a\subst{b}{x}$
\item $(\Lterm{X}{t})\; S \to_\beta t\subst{S}{X}$
\end{itemize}
\item A la Cury :
\end{enumerate}



\end{document}